Come anticipato, nell’ambito della sicurezza di una Smart Home ci si riferisce a due campi principali, ovvero quello della sicurezza fisica e della sicurezza informatica. La cosa interessante è che in pratica la sicurezza informatica, o logica, è un qualcosa la quale necessità nasce dal momento in cui la sicurezza fisica interagisce, per merito dell’IoT, con gli altri devices all’interno di una casa, esponendo dati e privacy a un potenziale rischio legato alla vulnerabilità degli oggetti smart che li gestiscono . Questo è il legame che associa un sistema di videosorveglianza, d’allarme o GPS alla necessità di mettere la propria Smart House in una condizione ottimale a livello di cyber security. Si può dunque parlare, secondo quanto espresso in un articolo di elettronica news, non solo di oggetti di sicurezza smart che possono interagire fra di loro ma di sicurezza smart degli oggetti stessi .
La presunta vulnerabilità citata innanzi può essere attribuita, secondo quanto emerge dal recente report in merito alla cyber security rilasciato da Swisscom Security nel maggio del 2018, proprio all’innovazione tecnologica la quale crea una convergenza fra ciò che è fisico e ciò che è virtuale . Una realtà, questa, che è estremamente attuale e veritiera in quelle che sono le Smart Home. Sostanzialmente è una catena che continua nel tempo, che gli hacker sfruttano a loro vantaggio usufruendo anche dell’inesperienza dei consumatori e creando delle possibili nuove minacce .
Come detto le minacce nascono di pari passo con l’evolversi della tecnologia. L’immagine soprastante è un radar, sempre estratto dal report di Swisscom Security, il quale mostra le diverse minacce, assegnandole a uno dei sette domini e, in base all’attualità e alla concretezza di esse sono state assegnate in uno dei quattro livelli: più una minaccia è verso il centro e più è attuale e concreta. 
Quanto emerge, nell’ambito dell’IoT, è che ci sono due diversi tipi di minacce categorizzati differentemente, ma che entrambi fanno suonare un campanello d’allarme, ovvero: IoT devices e IoT-based DDos. È opportuno dapprima spiegare cosa si intende con DDos e qual è la differenza con il primo. Si tratta dell’acronimo del termine inglese “Distribued denial-of-service” (DDoS), consiste nel rendere inutilizzabile e irraggiungibili, potenzialmente, qualsiasi dispositivo legato alla rete, sia computer che dispositivi IoT . Questo è reso possibile dai così detti “Botnet” (rete di bot), ovvero una serie di nodi della rete composti di oggetti compromessi in grado di sferrare attacchi in qualsiasi momento .
Secondo il radar Swisscom, come anticipato, entrambe le minacce fanno suonare un campanello d’allarme, poiché entrambe vengono collocate nei due cerchi più allarmanti. La più preoccupante riguarda gli IoT devices, collocati nel primo cerchio, denominato “temi principali”. Essi sono categorizzabili come attacchi cyber i quali colpiscono poi in seguito a livello fisico. Il pericolo rimane sempre quello: se non adeguatamente protetti gli oggetti legati all’IoT possono essere una minaccia per i dati e per la loro disponibilità in termini di prestazioni . Questo causerebbe, come detto, delle problematiche proprio a livello di sicurezza fisica. 
Per quel che riguarda invece IoT e DDos, questa minaccia viene assegnata al secondo livello di preoccupazione, quindi “allerta precoce”. Si tratta dunque del legame che c’è fra la diffusione dell’IoT e la possibilità di usarne i devices non protetti come “candidati di trasmissione” per Botnet. La categoria è quella della proliferazione, perciò un qualcosa in grado di diffondersi e moltiplicarsi .
La diffusione dell’IoT risulta dunque essere una minaccia, in quanto se non protetto adeguatamente può diventare una vera e propria arma informatica. 
Un esempio reale di quanto detto in merito alla vulnerabilità lo espone il portale online CorriereComunicazioni.it (CorCom), il quale scrive in merito a un report effettuato dall’azienda russa Kaspespery, specializzata in cyber security , il quale mette al corrente del numero sempre più crescente di attacchi rilevati verso oggetti connessi grazie all’IoT . Ciò che è stato evidenziato, grazie a dei sistemi appositi in grado di attirare gli hacker denominati “Honeypot” , è che rispetto al 2017 i numeri di attacchi malware (ovvero di software dannosi ) verso dispositivi IoT è già triplicato solamente nel primo semestre del 2018 . La cosa ancor più preoccupante è che molti di questi tentativi di attacchi, lo conferma anche Fastweb , sono avvenuti soprattutto attraverso router ma anche oggetti di uso comune come le lavatrici e altri dispositivi come ad esempio stampanti e dispositivi DVR, che rientrano nel ramo IoT .
