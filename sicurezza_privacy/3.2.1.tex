Secondo quanto viene definito dall’enciclopedia Treccani la sicurezza ha diverse definizioni, delle quali la seguente riguarda particolarmente il tema di cui verrà trattato in questo capitolo:
“La condizione che rende e fa sentire di essere esente da pericoli, o che dà la possibilità di prevenire, eliminare o rendere meno gravi danni, rischi, difficoltà, evenienze spiacevoli, e simili.” (TRECCANI, Sicurezza, www.treccani.it (ultima consultazione: 23 novembre 2018) 
Ciò che traspare dalla definizione di questo termine è comunque piuttosto generico, poiché la sicurezza così intesa nella definizione soprastante non riguarda un ambito specifico ma si estende su i più disparati settori . Si pensi ad esempio alla sicurezza in un pub o una discoteca, dove oltre a delle norme specifiche solitamente degli uomini sono assunti e incaricati di mantenere l’ordine e la sicurezza di tutti i presenti; oppure alla sicurezza dei lavoratori stessi, la quale deve essere garantita dal datore di lavoro . Altri esempi possono anche essere la sicurezza alimentare o la sicurezza aerea. Questi elencati sono tutti diversi fra di loro e non riguardano in maniera diretta le Smart Home, ma rimangono pur sempre degli aspetti fondamentali della sicurezza e che permettono a tutti di poter vivere la propria quotidianità tranquillamente, il che è essenziale.
Quanto detto precedentemente dimostra che la sicurezza è ampliamente classificabile in diverse sezioni, delle quali solo alcune riguardano le Smart Home e ne garantiscono protezione e prontezza al pericolo rispetto a più attori e avvenimenti. Le principali branchie della sicurezza che il mondo della casa intelligente coinvolge sono: la sicurezza fisica, che comprende ad esempio quelle che possono essere fughe di gas, gli incendi e i furti , e secondariamente la sicurezza informatica (anche conosciuta come sicurezza logica) la quale, al contrario della prima, impedisce l’accesso ai “luoghi” digitali, alle informazioni e ai dati , i quali sono di assoluta rilevanza in un’abitazione smart. Perciò, nel corso del capitolo, verranno sviluppate parallelamente e confrontate, in quanto sono composti da strumenti e processi che li differenziano fra di loro, ma che collaborando in sinergia hanno uno scopo unico, ovvero quello di rendere la quotidianità domestica sicura e vivibile, in questo caso anche con un livello di garanzia ed efficienza maggiore rispetto a quanto può, solitamente, garantire una casa tradizionale. 
