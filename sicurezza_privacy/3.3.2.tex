Antecedentemente all’IoT si può parlare di una fase definibile come “pre-Internet of Things” , visto che esistevano già dei prodotti disponibili in grado di collezionare dati. Essi agivano però grazie alla sensoristica semplice, che era in grado di trasformare in dati digitali delle informazioni . La differenza sostanziale rispetto al concetto odierno di IoT sta nel fatto che, in un secondo momento, è arrivata la connessione in rete che permette ai dispositivi non solo di collezionare ma anche scambiare dati e informazioni .
È importante sottolineare il fatto che i cambiamenti non si sono fermati, ci sono state e ci saranno sempre delle evoluzioni e dei cambiamenti che permetteranno agli smart objects di essere sempre più precisi e efficienti nei loro compiti. È dunque necessario che la società moderna sia pronta e cosciente del fatto che questa è solo una fase di un processo fatto da continui cambiamenti ed evoluzioni, non prive di controversie, problematiche e sfide da affrontare (soprattutto in merito alla privacy) che ci costringeranno a cambiare nuovamente in futuro . 
