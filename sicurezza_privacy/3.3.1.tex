Una definizione in sé è già stata data nel primo capitolo, ma ci sono degli aspetti che meritano di essere approfonditi in quanto strettamente legati con i temi di questo capitolo.
Una sua particolarità è che grazie alla raccolta di dati e informazioni i devices collaborano fra di loro e interagiscono con l’utente con lo scopo di migliorarne la vita quotidiana e di soddisfarne i bisogni a corto e lungo termine. In merito ai temi del capitolo si tratta di mantenere la casa e chi ci vive al sicuro. Perciò l’evoluzione di internet e della rete non ha portato solo a connettere i computer, ma anche dei semplici oggetti che possono avere dimensioni molto piccole. Essi vengono anche definiti come “Smart objects”, e sono solitamente composti da diversi componenti, ovvero: microprocessore, memoria, capacità di archiviazione, modulo di comunicazione e una carica di energia . L’aspetto dell’IoT che più conta, in questo caso, è proprio la possibilità di poter collezionare dati. Un fattore, questo, che come si vedrà in seguito può creare delle problematiche a livello di privacy.
Le categorie di oggetti e cose che usufruiscono della rete possono essere molteplici, come ad esempio: dispositivi, apparecchiature, impianti, sistemi, materiali, prodotti tangibili, opere, beni, macchinari e attrezzature . Ovviamente per poter raccogliere e condividere dati devono avere specifiche caratteristiche e funzionalità che li rendono diversi da oggetti qualunque, come ad esempio identificazione (come un indirizzo IP), localizzazione, connessione, capacità di raccogliere dati e di comunicare con l’esterno  . Nell’ambito delle Smart Homes l’Internet of Things ha diversi scopi, ma principalmente da la possibilità agli utenti di poter controllare da remoto gli oggetti collegati e diversi ambienti della propria abitazione e i devices che ne fanno parte, compresi quelli legati alla sicurezza. 
