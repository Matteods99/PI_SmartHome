Ci sono diversi oggetti legati alla sicurezza fisica che sono di rilievo ma soprattutto innovativi rispetto a quelli a cui siamo abituati in una casa tradizionale. Un esempio può essere quello della mia attuale casa, nella quale c’è un sistema di videosorveglianza composto da due telecamere riposte all’esterno che mirano all’entrata principale e all’entrata secondaria (finestra in giardino) oltre a dei sensori posti sopra tutti gli accessi alla casa dall’esterno (porte e finestre) i quali rilevano l’apertura e la chiusura di esse. Nel caso in cui, ad allarme inserito, venissero aperte scatta automaticamente una sirena molto potente. Il sistema in questione venne installato nel 2003, e infatti il controllo che si ha su di esso è minimo: non si tratta di un sistema moderno il quale permette di essere controllato anche da remoto come accade con i sistemi di sicurezza IoT. Al massimo è possibile vedere in diretta, attraverso la televisione, la visuale delle videocamere. Questo ovviamente non permette di tenere sotto controllo la casa nel momento in cui si è al di fuori di essa. Un altro punto negativo è che rispetto alla maggior parte delle soluzioni moderne i costi dei sistemi d’allarme dell’epoca erano molto elevati: all’incirca tutto insieme il nostro sistema di allarme è costato 10'000 CHF. Soprattutto la possibilità di scegliere era molto ridotta, mentre oggi ai consumatori sono offerte soluzioni che a livello di monitoraggio della casa sono valide ed economiche.
Come si può vedere nell’immagine soprastante, al giorno d’oggi molti prodotti vacillano su dei prezzi nettamente inferiori e accessibili, che possono variare dai 100 CHF fino ai 500 CHF. Ovviamente esistono anche fasce di prezzo più elevate, le quali principali categorie sono:
•	quelle fai-da-te (prezzo basso)
•	standard (prezzo medio)
•	domotica integrata (prezzo alto, comparabile ai prezzi dei sistemi nelle case tradizionali) . 
Un aspetto interessante e vantaggioso degli oggetti smart legati alla sicurezza è, come si vede nell’immagine, che molti di essi sono facilmente installabili e oltretutto non necessitano di un professionista. Inoltre un altro punto a favore è che i devices, anche se prodotti da aziende diverse, sono compatibili con altri. Va comunque tenuto di conto che non sempre è così, non tutti gli oggetti sono compatibili ed è bene, in quanto consumatore, esserne a conoscenza. Ad esempio il sistema “Nest Secure” è ottimo dal momento in cui, nella casa, vi sono già prodotti Nest; ma dal momento in cui si vorrebbe adattarlo a dei prodotti come “Amazon Alexa” questo non sarebbe possibile .
Prodotti piuttosto famosi come “Nest secure” o “Adobe Home Security Starter Kit”, i quali rientrano in quella fascia di prodotti in grado di offrire prestazioni comunque buone ma che tendono a costare meno (anche se in questo caso specifico il costo di Nest è elevato se comparato ad altri prodotti della stessa fascia), senza però tenere conto del fatto che sul lungo periodo potrebbero avere dei costi maggiori rispetto ai sistemi tradizionali. Un motivo potrebbe essere l’aggiunta di parti accessorie che comportano un costo anche se minimo, come accade nel caso di Nest . Un altro costo ulteriore è attribuibile principalmente alla necessità di coprire e proteggere i sistemi IoT dalle minacce legate al mondo dell’hacking. Come citato inizialmente la sicurezza informatica è proprio un qualcosa di opportuno dal momento in cui esistono sistemi IoT facilmente vulnerabili, poiché sono collegati grazie alla rete internet la quale può essere violata facilmente, soprattutto se non protetta. 
Questo risulta essere proprio il punto fondamentale che, rispetto alla sicurezza di una casa tradizionale, va creare dei dubbi proprio sotto l’aspetto della vulnerabilità e della sicurezza della privacy e dei dati di tutti coloro che usufruiscono di queste smart possibilità, un qualcosa che dunque mette in dubbio la sicurezza di una Smart Home proprio sotto questo aspetto e che potrebbe, come verrà approfondito in seguito, creare delle problematiche.
\begin{enumerate}
\item  quelle fai-da-te (prezzo basso)

\item  standard (prezzo medio)

\item  domotica integrata (prezzo alto, comparabile ai prezzi dei sistemi nelle case tradizionali)\footnote{\ DOGIARO,\ Quanto\ Costa\ la\ Domotica:\ Ecco\ i\ 3\ Livelli\ di\ Prezzi\ pi\textrm{\`{u}}\ DIFFUSI\ per\ la\ Smart\ Home,\ www.blog.dogiaro.com\ (ultima\ consultazione:\ 24\ gennaio\ 2019)}. 
\end{enumerate}