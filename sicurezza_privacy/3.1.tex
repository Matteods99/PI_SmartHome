La sicurezza è uno degli aspetti fondamentali di cui le Smart Home e la domotica si preoccupano . Essa va assicurata, e sia nelle case tradizionali che in quelle del futuro vi sarà una varietà e un numero sempre più grande di sistemi, impianti e prodotti che cercheranno di garantirla. Ciò nonostante le Smart Home sono diverse, poiché portano con loro nuove tecnologie e metodi che se da una parte cercano di rendere più sicura una casa dall’altra potrebbero allo stesso tempo creare la necessità di risolvere nuove possibili sfide e controversie legate alla sicurezza, e di conseguenza anche alla privacy e alla protezione dei dati personali, che sono al giorno d’oggi fondamentali per molti individui e aziende, così come per le Smart Home stesse e per il loro funzionamento.
A tal proposito in questo capitolo verranno analizzati questi elementi rispetto alle Smart Home, facendo anche un confronto con il concetto di “abitazione tradizionale”. Verranno fatti degli accertamenti su quel che concerne la legislazione in vigore, svizzera ma anche internazionale, attraverso il regolamento generale della protezione dei dati (GDPR), che risulta essere di primaria importanza soprattutto per quel che concerne i dati personali e il loro sviluppo in futuro. In questo modo potranno essere tratte delle conclusioni che diano delle risposte alle seguenti domande:”Sono effettivamente sicure le Smart Home? Possiamo garantire la nostra privacy senza intaccare l’efficienza delle Smart Home?”
