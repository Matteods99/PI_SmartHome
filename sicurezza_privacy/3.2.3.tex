Per quel che concerne la sicurezza e la privacy delle Smart Home i dati personali sono un aspetto fondamentale, che permette di fare una differenziazione degli aspetti di sicurezza e privacy rispetto alle case tradizionali, poiché rispetto a quest’ultime nelle abitazioni high tech le informazioni e i dati vengono scambiati e utilizzati dai devices interconnessi fra di loro in modo tale da sfruttare al meglio le funzionalità della Smart Home, ma questo verrà approfondito maggiormente successivamente nel paragrafo dedicato all’Internet of Things (IoT). Nell’ambito delle Smart Home i dati risultano essere centrali poiché possono essere valorizzati in diversi modi, ottenendo informazioni sui devices e sulle abitudini di chi ne usufruisce, anche a favore delle aziende ma non solo . Questo si ricollega a quanto detto precedentemente sul fatto che essi debbano essere tutelati a favore degli utenti, in quanto i dati possono essere soggetti a un’elaborazione e un utilizzo improprio, spesso senza che il detentore di questi ne sia a conoscenza. L’evoluzione della sicurezza controllabile da remoto ha quindi dato molte possibilità all’utenza in fatto di videosorveglianza e sicurezza a livello generale, ma secondo quanto emerge da un articolo pubblicato da la Repubblica ha anche automaticamente incrementato le possibilità di attacchi cyber e utilizzo improprio dei dati personali, dunque aspetti legati alla privacy . Non a caso, nonostante oggi le Smart Home siano una realtà in linea di massima piuttosto solida il quale mercato è in crescita (secondo quanto si evince dai dati di Swiss Life nel 2016 in Svizzera il mercato delle Smart Home ha generato un fatturato pari a 58 milioni di franchi e nel mondo di CHF 60 miliardi, ma si prevede un aumento fino a un valore complessivo di 500 miliardi di franchi entro il 2021)  molti consumatori sono ancora restii all’adozione di queste nuove tecnologie legate alle abitazioni, proprio a causa della sicurezza dei dati alla violazione di essi .
I dati personali sono definiti nel seguente modo dal Garante per la protezione dei dati personali in Italia: 
“Sono dati personali le informazioni che identificano o rendono identificabile, direttamente o indirettamente, una persona fisica e che possono fornire informazioni sulle sue caratteristiche, le sue abitudini, il suo stile di vita, le sue relazioni personali, il suo stato di salute, la sua situazione economica, ecc..”(GARANTE PER LA PROTEZIONE DEI DATI PERSONALI, Cosa intendiamo per dati personali?, www.garanteprivay.it, ultima consultazione : 23 novembre 2018) 
Come accennato in precedenza esistono diversi tipi di dati personali, che possono essere distinti in più categorie.
\begin{enumerate}
\item  \textbf{Dati identificativi}
\end{enumerate}
Vengono anche definiti “Personally Identifiable Information” (PII) , e sono in grado di fornire informazioni divisibili tra identificazione diretta e indiretta. Ad esempio possono essere informazioni come il nome, il cognome e in generale i dati anagrafici (identificazione diretta); d’altro canto possono essere considerate informazioni che permettono l’identificazione indiretta i vari numeri di identificazione come ad esempio il numero di targa o il codice fiscale.
\begin{enumerate}
\item  \textbf{Dati soggetti a trattamento speciale}
\end{enumerate}
Sono soggetti a trattamento speciale quei dati che la maggior parte di noi conoscono come dati sensibili, ma che di fatto oggi sarebbe meglio chiamare ex sensibili . Questi sono in linea di massima non trattabili, se non in alcuni casi specifici, come ad esempio nel caso in cui essi possono essere di interesse pubblico o nel caso vi sia un consenso da parte del proprietario dei dati in questione. Sono ad esempio dati sensibili quelli legati al credo religioso o filosofico, ma anche di natura genetica o etnici. 
\begin{enumerate}
\item  \textbf{Dati biometrici}
\end{enumerate}
Secondo quanto pubblicato su Agendadigitale.eu, portale d’informazione in merito allo sviluppo digitale, i dati biometrici sono:
Dati personali ottenuti da un trattamento tecnico specifico, relativi alle caratteristiche fisiche, fisiologiche o comportamentali di una persona fisica e che ne consentono o confermano l’identificazione univoca, quali l’immagine facciale o i dati dattiloscopici (AGENDA DIGITALE EU, Home > Cittadinanza Digitale > GDPR, che si intende per dati personali: natura, tipologie e qualità, www.agendadigitale.eu (ultima consultazione: 23 novembre 2018) 
\begin{enumerate}
\item  \textbf{Dati anonimi e dati pseudonimi}
\end{enumerate}
La differenza fra i due termini è che i primi non sono considerati come personali, a differenza dei secondi. Questo è dovuto dal fatto che per essere considerabile personale un dato deve anche poter identificare l’utente. Se i dati anonimi non sono in grado di risalire all’identità di chi ne è interessato i dati pseudonimi possono, ma in modo meno esplicito e diretto rispetto, ad esempio, ai dati identificativi. Questo è dovuto dal fatto che gli elementi identificativi vengono modificati o comunque codificati, come nel caso di un nickname al posto di un nome reale .
\begin{enumerate}
\item  \textbf{Dati relativi alle comunicazioni elettroniche}
\end{enumerate}
Vengono anche definiti con l’appellativo di “dati emersi” e consistono in tutta quella serie di dati personali che possono essere ottenuti attraverso degli apparecchi elettronici, come smartphones e strumenti legati alla geolocalizzazione; oltre all’utilizzo di internet stesso. Questi dati in particolare possono essere considerati fondamentali nell’ambito delle tematiche proposte, come si vedrà nel paragrafo seguente . 