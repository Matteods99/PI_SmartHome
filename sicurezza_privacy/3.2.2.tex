Così come per la sicurezza anche la privacy è un qualcosa di estremamente variegato e che, soprattutto, è in continua evoluzione in quanto termine. Risulta anche essere molto personale, dato che ognuno può averne una percezione soggettiva e diversa rispetto ad altri individui . Basti pensare che nel 1890 venne già data una definizione di privacy all’interno del “Harvard Law Review – The Right to Privacy”, redatto dagli allora giuristi americani Samuel Warren e Louis Brandeis i quali, molto semplicemente, la descrissero come il diritto di essere lasciati soli . Questa è rimasta come una sorta di base che vale ancora oggi, ma che come detto innanzi non è un qualcosa di immutato nel tempo, in quanto la privacy cambia parallelamente con lo sviluppo sociale, culturale e tecnologico . Può essere quindi che la percezione di privacy di uno svizzero è diversa da quella di un giapponese, per via di culture e sviluppi sociali molto distanti fra di loro; questo si ricollega a quanto detto sulla soggettività della percezione della privacy. Un’altra caratteristica che privacy e sicurezza condividono è la multi-contestualità nella quale possono trovarsi. Oggi, la privacy, si situa in contesti che vanno dal giornalismo all’utilizzo degli smartphone, ma anche nell’informatica in generale, internet e sanità . 
Come introdotto precedentemente il significato di questo concetto si evolve e cambia anche di pari passo con l’evoluzione della tecnologia, ovvero l’aspetto che più conta, fra quelli citati, nell’ambito delle Smart Home. Difatti, secondo quanto emerge da Wikipedia e vari esperti del settore come l’avvocato Bruno Saetta, se un tempo la privacy si rifaceva alla tutela della vita privata oggi si è estesa fino al diritto alla protezione, alla salvaguardia e alla riservatezza dei propri dati personali . Il merito è attribuibile, principalmente, al progresso tecnologico. Ma non solo, va sottolineato che proprio a causa delle nuove tecnologie, le cui principali sono quelle legate alla comunicazione a distanza e all’archiviazione , la privacy è diventata sempre più un concetto sottile, poiché grazie ad esse le nostre informazioni possono essere facilmente ottenute (si pensi ad esempio alla geolocalizzazione).
È importante evidenziare il fatto che, nonostante lo stretto rapporto che esiste fra sicurezza e privacy, questi sono comunque diversi fra di loro e non va fatta confusione a riguardo. Un esempio piuttosto semplice che aiuta a comprendere meglio questo concetto può essere quello dell’e-commerce: effettuando un acquisto online tramite un protocollo, come ad esempio https, i dati inseriti dall’acquirente possono essere considerati al sicuro rispetto a terzi poiché avviene un passaggio diretto dal cliente al server fornitore, ma questo non garantisce la privacy di questi dati . Il rapporto che esiste fra i due termini è comunque fondato, tenendo conto del fatto che la difesa della privacy è presupposta, in una società libera, nel momento in cui è anch’essa sicura . È perciò importante sviluppare questi due concetti al meglio e in sintonia, non solo nella società ma anche nelle nostre abitazioni, in modo da poter soddisfare il nostro bisogno fondamentale, secondo quanto espresse Abraham Maslow attraverso la sua piramide dei bisogni, di sentirsi al sicuro .
