Le Smart Home, come tutte le cose alla fine, necessità di un particolare tipo di manutenzione, volta a rendere efficienti ed efficaci i sistemi il più possibile. 
Come già detto, i sistemi informatici delle Smart Home sono un po’ come dei computer, infatti sono suddivisi come tali: una parte Hardware e una Software.
Per quanto riguarda il lato software, il sistema ovviamente necessita di aggiornamenti. Questi aggiornamenti possono venire effettuati automaticamente con i sistemi più avanzati, oppure devono venir aggiornati tramite terzi, come ad esempio degli elettricisti. Gli aggiornamenti vengono fatti su tutti i dispositivi e sono fondamentali per garantire il migliore funzionamento di tutto l’impianto e anche per poter avere la possibilità di implementare nuove funzionalità.
Invece, il lato hardware non necessita molta manutenzione, ad eccezione quelle volte in cui i dispositivi non si danneggiano o si rovinano. È inoltre possibile aggiornare i vecchi dispositivi con alcuni più moderni. 
In ogni caso non è necessaria una vera e propria manutenzione annuale, dato che comunque sono sistemi che segnalano da soli i casi di malfunzionamento, quindi quando il sistema ha una falla lo si saprà subito
