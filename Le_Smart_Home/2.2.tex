La definizione di Smart Home non è così semplice da dare. La traduzione dall’inglese è letteralmente “casa intelligente” ed è effettivamente ciò che è, dato che si tratta dell’inserimento di determinati devices o elementi elettronici all’interno della propria abitazione. Quest’unione tra tecnologia e immobile ha il compito di migliorare la vita delle persone che ci vivono, correggendo alcuni errori o dimenticanze che i proprietari commettono inconsciamente, magari superflue per l’uomo, che portano ad una riduzione dei consumi, una maggiore comodità e sicurezza nella propria abitazione.
Per avere una casa intelligente si hanno principalmente due opzioni: la prima è quella della personalizzazione, ovvero si può decidere di far costruire la propria abitazione da zero con tutti i relativi elementi tecnologici già installati nella casa, senza devices aggiuntivi e con interfacce utente molto più “concrete” (esteticamente visibili). Ad esempio, in questo tipo di configurazione troviamo ancora i pulsanti al muro o comunque un pannello dei comandi. Secondariamente troviamo un’altra configurazione, quella un po’ più commerciale, ovvero l’acquisto di devices (alla fine una sorta di gadget) che comunicano direttamente con il proprio smartphone.
Le Smart Home aiutano la nostra quotidianità davvero in molteplici modi, tramite vari sensori, fotocellule e luci riescono ad aiutarci dove noi non riusciamo ad intervenire.
Per fare un esempio concreto, la casa riesce a controllare le varie temperature in tutte le stanze e riesce quindi a regolarsi secondo le nostre necessità. Potrebbe decidere di spegnere il riscaldamento in determinate stanze che non utilizziamo mai, oppure il sistema potrebbe addirittura riconoscere le persone che si trovano nella stanza ed adattare di conseguenza la temperatura e l’umidità a seconda delle nostre preferenze.
Invece, per quanto riguarda la sicurezza le Smart Home riescono ad essere molto più efficienti dei normali sistemi di sorveglianza.
Ad esempio, ci sono alcuni devices che, grazie ad alcuni sensori, riescono a rilevare la presenza di fumo, gas, incendi e riescono ad intervenire di conseguenza, ad esempio chiamando automaticamente o il proprietario o direttamente i singoli servizi di assistenza.
Sempre in ambiti di sicurezza, si trovano alcuni devices davvero tanto interessanti quanto semplici e utili. Basti pensare al campanello con videocamera e porta automatica, dove l’utente può vedere chi ha suonato al campanello anche da remoto e decidere se aprire la porta oppure no, anche direttamente dal posto di lavoro.
In generale in una Smart Home troviamo una continua interazione tra vari dispositivi e monitoraggio delle varie parti della casa, tramite videocamere e il controllo di quasi tutti i dispositivi elettronici smart, come ad esempio elementi della cucina, impianto stereo, televisori, tende e finestre. Il tutto controllabile anche con sistema di riconoscimento vocale.
Ciò ovviamente porta ad un vantaggio nella propria routine che non lascia indifferenti, dato che la casa è pienamente capace di svegliare il proprietario la mattina con la propria canzone preferita, aprendo dolcemente le tende, gestendo la luminosità e il colore delle luci secondo le preferenze del proprietario e accendendo magari il camino.
Sicuramente nel frattempo la macchina del caffè sta svolgendo il suo lavoro, cosicché il proprietario possa trovare una tazza di caffè caldo appena fuori dal letto. Successivamente il proprietario potrebbe farsi la doccia mentre il sistema racconta la sua agenda e le cose da svolgere durante la giornata.
Alla fine, l’unico limite è la propria immaginazione.
Oltre a tutti i vantaggi di comfort si ottengono enormi agevolazioni per quanto riguarda i costi, l’impatto ambientale e una notevole facilità di funzionamento. 
Infatti, i sistemi delle Smart Home tendono ad essere davvero estremamente semplici, utilizzabili persino da bambini o anziani. Le interfacce utente sono tendenzialmente chiare e di facile utilizzo, oltre ad essere molto affidabili e quindi durature nel tempo. Infatti, non bisogna dimenticare che gli anziani hanno bisogno di più attenzioni e assistenza permanente, cosa che le Smart Home fanno egregiamente. 
Ad esempio, se un uomo anziano dovesse cadere e non riuscisse più ad alzarsi, il sistema dovrebbe capirlo e chiamare automaticamente assistenza fisica per aiutare l’infortunato. Oppure se dovesse entrare un ladro in casa, il sistema dovrebbe automaticamente chiamare la Polizia e il proprietario, facendo vedere addirittura in che stanza si trova e di conseguenza anche cosa stesse rubando. 
Per quanto riguarda i costi ovviamente l’investimento iniziale può essere un po’ più incisivo di una normale abitazione con sistema tradizionale. Bisogna però dire che, basandoci su una visuale a lungo termine, il risparmio è garantito, 
Ad esempio, basti pensare ai risparmi sulla bolletta della luce dovute ad alcune dimenticanze che succedono nella vita quotidiana.
Inoltre, è imperativo dire che il valore di mercato delle Smart Home è nettamente maggiore a quello di alcune abitazioni con sistema tradizionale, dato che la tecnologia al giorno d’oggi è un vero e proprio valore aggiunto. 
Bisogna però dire che il proprietario deve già mettere in conto che prima o poi i sistemi andranno sostituiti oppure aggiornati. È dunque richiesto un piccolo costo per quelle occasioni in cui bisogna pensare alla manutenzione della propria abitazione.
Forse un possibile aspetto negativo è il fatto che è sempre richiesta una maggiore conoscenza e istruzione tra i comuni elettricisti, i quali sono costretti ad imparare un nuovo tipo di installazione (concetto trasformabile in opportunità per dei nuovi mercati). È anche sì vero che alcuni sistemi sono talmente avanzati che riescono ad aggiornarsi da soli, senza alcun aiuto di terzi, fanno anche delle scansioni regolari al fine di identificare alcune possibili anomalie del sistema. Secondariamente, un ulteriore problema nelle Smart Home è che semplicemente le persone non ne sono a conoscenza o comunque non sufficientemente informate in materia. Le persone tendono a pensare ancora alla casa autonoma come a quegli edifici super futuristici dei film di fantascienza, ma non sono seriamente coscienti di ciò che il mercato concretamente offre dei devices che migliorerebbero il comfort nella vita di molte persone.
Concettualmente secondo le possibilità del mercato odierne, è addirittura possibile far sì che la casa guadagni autonomamente, tramite dei pannelli solari oppure altre fonti di energia alternative. 
La casa può infatti produrre elettricità ed utilizzarla, però, quella che non usa, potrebbe automaticamente venderla allo stato e ricavarne un ricavo, che verrebbe direttamente versato sul conto corrente del proprietario. In conclusione, il proprietario avrebbe una vera e propria fonte di reddito autonoma, con grado di rischio nullo e con introiti (sebbene di vario valore a causa delle condizioni naturali).
