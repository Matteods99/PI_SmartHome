Non è semplice spiegare il (vero e proprio) funzionamento delle Smart Home, dato che rimangono dei veri e propri sistemi informatici, suddivisi, come tutti i computer da una parte concreta e una astratta. Per definizione queste due “parti” vengono dette Software (astratta) e Hardware (concreta). Il Software è il sistema operativo delle Smart Home, il quale è tendenzialmente differente tra i vari brand, anche perché non è sempre presente un vero e proprio Software. Però l’Hardware è sempre presente, quest’ultimo è la parte fisica dei vari devices, ovvero tutto ciò che concretamente si vede. Uno dei sistemi elettronico integrati (personalizzato) su cui la domotica si basa è un sistema chiamato BUS, capace di controllare un insieme integrato di vari tipi di funzionalità, complesse o facili che siano. Il BUS, molto semplicemente, è un sistema doppio, ovvero che riesce a gestire sia l’alimentazione del dispositivo che lo scambio di dati tra tutti i vari devices. Ovviamente tutti questi devices comunicano tra loro tramite la rete di casa oppure una rete dedicata. 
Per quanto riguarda lo scambio di dati del BUS, i dati viaggiano sottoforma di bit codificati, ovvero il buon vecchio codice binario. 
Fisicamente, la persona che preme un pulsante da qualche parte come ad esempio per accendere la luce, trasmette una sorta di impulso contenente l’informazione di accendere la luce. Successivamente questa informazione viaggia verso un device attuatore che a sua volta invia un feedback con la relativa conferma della ricezione del comando.
Il sistema BUS riesce purtroppo a comunicare soltanto con un device per volta. Per evitare problemi o collisioni tra i vari devices ci sono determinati sistemi precauzionali di sicurezza che intervengono per gestire tutti i devices. Per fare un semplice esempio: se qualcuno volesse accendere la luce e contemporaneamente aumentare la luminosità non potrà farlo, dato che può effettuare solo un’azione per volta. Ciò che il sistema concretamente farà, sarà accendere la luce e mettere in coda il comando che abbassa la luminosità. Quindi una volta accesa la luce, il sistema effettuerà in un secondo momento l’azione di regolazione della luce. Il BUS ovviamente funziona soltanto per i sistemi integrati (personalizzati). Per quanto riguarda invece i sistemi “commerciali” che sarebbero i sistemi tradizionali, vi è una comunicazione completamente diversa, ovvero un continuo scambio di informazioni tra i vari dispositivi che sono interconnessi tra loro wireless. Questi dispositivi si potrebbe dire che sono dotati di intelligenza propria, dato che possiedono tutti delle piccole “menti” ovvero i processori che permettono al dispositivo di autogestirsi e comunicare direttamente con lo Smartphone del proprietario. 
Ovviamente ci sono altri sistemi di comunicazione tra devices. Il primo sistema in assoluto che venne inventato negli anni Settanta. L’X10 è tutt’oggi uno dei sistemi più utilizzati, esso venne inventato in Scozia e si basa sulla comunicazione tramite rete elettrica. In questo sistema tutti gli apparecchi sono, bene o male, ricevitori. Invece qualsiasi mezzo di controllo, come ad esempio i telecomandi, sono dei trasmettitori che emettono un segnale direttamente verso il device. 
Purtroppo, anche questo sistema ha dei limiti. Infatti, comunicare su reti elettriche non è la cosa migliore e soprattutto è poco affidabile. Per risolvere questi grossi problemi sono stati inventati nuovi sistemi di tutto rispetto, che utilizzano sistemi di comunicazione molto più innovativi, come comunicazioni via Bluetooth, WiFi o radio. I due principali sistemi principali sono ZigBee e Z-Wave. In realtà questi non sono veri e propri sistemi, ma sono protocolli, ovvero l’insieme di determinate regole che, se lette dai giusti sistemi informatici, vengono utilizzate come forma di linguaggio.
In pratica il protocollo è un vero e proprio linguaggio codificato che viene utilizzato tra più devices per comunicare tra loro, è come se fosse una vera e propria lingua. 
Con questo particolare tipo di sistema è possibile mandare il messaggio a destinazione in modi diversi. Ovviamente ciò ha permesso un ampliamento delle opzioni tra cui la possibilità di scegliere. 
Attualmente, il sistema oggettivamente migliore è quello che funziona con la rete wireless di casa, poiché offre una flessibilità più ampia per ogni singolo device. Esso diventa potenzialmente personalizzabile in ogni aspetto, dato che il completo controllo dei dispositivi proviene solo e unicamente dal proprio Smartphone. 
Vi sono numerosi altri sistemi  di comunicazione nelle Smart Home, alcuni indicati qui di seguito:
\begin{itemize}
    \item{Insteon}
    \item{CoCo}
    \item{Thread}
    \item{WeMo}
    \item{Nest}
    \item{Bacnet}
    \item{Dali}
    \item{Infrarosso}
\end{itemize}
Tutti questi sistemi hanno modi di funzionare differenti tra loro. Insteon ad esempio è un insieme di protocolli che comunicano sia tramite rete elettrica (Powerline) che via wireless. CoCo invece utilizza delle onde radio ed è una variante del sistema X10. Thread invece è un protocollo piuttosto nuovo che funziona wireless. È risultato molto molto promettente ed è stato realizzato da una collaborazione tra Samsung, Google e altre importanti aziende.
Per quanto riguarda WeMo, esso lavora tramite WiFi ed è stato creato da Belkin, è compatibile con tutti gli altri devices di altre case produttrici. Successivamente, Nest è uno dei sistemi più conosciuti, lavorando tramite WiFi. È uno dei sistemi migliori ed è completamente creato da Google, infatti appartiene al kit Google Home.
Bacnet è invece un protocollo comunicativo aperto e neutrale, che può quindi essere adattato a vari sistemi. Successivamente, Dali è uno standard per le interfacce digitali, come ad esempio le schermate che si vedono sui tablet di comando nei sistemi più avanzati. 
Infine, abbiamo il sistema di comunicazione ad infrarossi, uno dei sistemi più utilizzati nei telecomandi delle TV. Questo è un sistema che è diventato piuttosto obsoleto, dato che è unidirezionale, impreciso e a distanza limitata
