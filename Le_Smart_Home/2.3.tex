Sicuramente non è possibile tracciare una vera e propria linea del tempo evolutiva dalla nascita delle Smart Home ad oggi, soprattutto perché non sono passati molti anni dal primo avvenimento delle Smart Home, essendo un prodotto attuale, soprattutto visto che i primi dati storici utilizzabili risalgono a non più di 100 anni fa.
È però possibile cercare di delineare alcuni momenti chiave nella storia per poi trarre una conclusione, o meglio, una linea temporale. Ovviamente si parlerà principalmente della terza rivoluzione industriale e, in alcuni termini, anche della quarta: ovvero due delle più grandi rivoluzioni degli ultimi secoli, periodi in cui vi è stata un vero e proprio progresso nel campo tecnologico o industriale. In ogni caso le Smart Home sarebbe stato possibile farle già molti anni fa a livello di tecnologie esistenti, però non è mai successo a causa della mancanza di una tecnologia abbastanza valida e consolidata che riesca a fare tutti questi automatismi contemporaneamente all’unisono. Esistevano già però determinate funzionalità che rendono in un certo senso “smart” la nostra abitazione, ad esempio il termostato che si alza automaticamente quando la sonda esterna capisce che è freddo.
In particolare, possiamo ricollegare la storia della domotica ad alcuni personaggi di spessore, come ad esempio William Penn Powers, il quale, per primo, creò una compagnia (antenata della Siemens) che inventò i climatizzatori automatici nel 1881. Grazie a quest’invenzione venne costruito il primo Hotel dotato di aria condizionata automatica sempre a Chicago nel 1907.
Durante tutto il ‘900 soprattutto negli USA vengono portate avanti alcune invenzioni che avrebbero poi portato la domotica a quello che è oggi. 
In particolare, nel 1950 circa, venne realizzato un dispositivo che permetteva il totale controllo di determinati impianti. Successivamente nel 1966, l’inventore Jim Sutherland creò il primo dispositivo automatico per la climatizzazione e anche di alcuni elementi elettrici. Negli anni ’70 infine, è stato sviluppato l’X10, il principale standard su cui si basa tutto il funzionamento delle Smart Home, anche odierne: concetto che verrà spiegato successivamente.
Sicuramente è possibile inoltre tracciare la storia della domotica grazie all’avvento dei sistemi informatici o computer, questi ha apportato un grosso cambiamento per quanto riguarda il mercato delle Smart Home, in quanto principalmente i devices avevano fatto un grande passo oltre, attraverso l’Internet delle cose, ovvero l’Internet of Things (IoT) che viene definito come un insieme di dispositivi che possono interconnettersi tra di loro tramite rete internet e senza l’intervento di terze parti. In pratica si tratta di una serie di tecnologie che permettono di collegare alla rete internet qualunque dispositivo.  In ogni caso il concetto verrà approfondito in seguito, insieme ai relativi concetti di sicurezza.
Nei primi anni del 2000 le Smart Home hanno cominciato a prendere piede e a diffondersi su tutto il globo, facendo comparire nel commercio le prime aziende e le prime soluzioni intelligenti, che si sarebbero evolute col tempo.
Oggi siamo in un periodo in cui i grandi brand cominciano a produrre elettrodomestici smart e c’è una vera e propria “corsa all’oro” per decidere chi è il numero uno sul mercato (dominato dai giganti Google, Apple e Amazon). 
Invece per quanto riguarda la diffusione della domotica a livello geografico, non è facile individuare un particolare schema di diffusione; anche perché essendo un “prodotto” molto innovativo e recente, come vediamo tutti i giorni dopotutto, si è diffuso dagli USA ed è successivamente arrivato in città come Londra o Parigi per poi infine diffondersi un po’ dappertutto.
Sul territorio italiano, questo mercato è in continua crescita; fino al 30\% ogni anno. In realtà la domotica è arrivata in Italia già negli anni ’70, però non ha mai preso veramente il via. Però oggi, finalmente, questo commercio ha preso piede e si sta diffondendo su tutto il globo.
Secondo alcune statistiche  effettuate in Italia, nel passaggio dal 2006 al 2008, le Smart Home sono quasi raddoppiate. La tabella si basa su impianti “base”, ovvero impianti piccoli e domestici e impianti “avanzati”, quindi impianti più grandi e quindi dedicati ad edifici di grandezza maggiore.
Riconducendosi alla situazione italiana, sono emerse alcune informazioni inerenti all’evoluzione delle Smart Home negli ultimi anni. In modo particolare, si sottolinea il fatto che, nonostante l’ovvia crescita esponenziale nel settore delle Smart Home, vi sono alcuni fattori che persistono e creano barriere nello sviluppo e nella diffusione, come ad esempio la bassa riconoscibilità di numerose marche che sono presenti sul mercato, ma che non riescono ad emergere. 
Per quanto riguarda il mercato invece, ci sono alcune informazioni indicative sul trand delle vendite dei prodotti “smart” per le case: i prodotti tendenzialmente più venduti in Italia sono quelli con applicazioni dell’Internet of Things per la sicurezza. Tra questi prodotti troviamo i più classici, cominciando con i semplici sistemi di sorveglianza fino ad arrivare alle serrature elettroniche e ai citofoni di nuova generazione in cui ai può vedere chi sta suonando al campanello.
Successivamente, i secondi prodotti più venduti sono tutti quelli che trattano l’ambito clima, come ad esempio i termostati interconnessi e automatici, oppure i termosifoni “smart”. 
Come terzo settore vi è quello incentrato sugli elettrodomestici, quindi qualsiasi elemento controllabile con un’applicazione, come ad esempio i celeberrimi “automatic-cooker” che riescono praticamente a preparare la cena da soli, con pochissimi accorgimenti. 
A livello internazionale gli smart devices più diffusi sono gli altoparlanti, come ad esempio tutti i sistemi multiroom intelligenti, che riescono a comunicare tra loro e riproducono i brani in modo simultaneo sfruttando la rete di casa. Ovviamente il tutto è opzionalmente controllabile con il comando vocale.
Come già detto in precedenza ci sono alcune barriere  o limiti che devono ancora essere superati. Le principali barriere sono tre:
\begin{itemize}
    \item{L’installazione dei prodotti}
    \item{L’integrazione dell’offerta con servizi di valore}
    \item{La presenza dei Brand affermati} 
\end{itemize}
La prima barriera è molto influente, perché nonostante i prodotti siano fatti per risolvere alcuni problemi, possono causarne altri, tra i quali la buona riuscita dell’installazione di tutti i devices. È vero che le case produttrici fanno il meglio che possono nel rendere comprensibile a tutti il funzionamento e soprattutto far si che chiunque acquisti dei devices riesca ovviamente anche ad installarli autonomamente. Tuttavia sorge il problema: tanta gente non ci riesce. 
Nonostante siano davvero dei sistemi elementari, tante persone a quanto pare non riescono ad effettuare un’installazione ottimale nelle proprie abitazioni. Allora queste persone sono costrette a chiedere aiuti a terzi dietro compenso: altri costi!
La seconda barriera è denominata “l’integrazione dell’offerta con servizi di valore”, ciò significa che le persone non ottengono un vero e proprio valore aggiunto con ciò che acquistano, infatti solo pochi sistemi di base offrono dei veri e propri servizi, come ad esempio l’allarme intrusione. 
La terza e ultima barriera è, come ho già detto, la mancata presenza sul mercato di Brand importanti e affermati. Il problema in tutto ciò è che il consumatore finale non si fida a sufficienza dei prodotti, poiché semplicemente non conosce la marca e reputa di conseguenza i prodotti obsoleti e/o inaffidabili, cosa ovviamente non vera.
