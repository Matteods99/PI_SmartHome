Nei capitoli precedenti è già stato accennato il fatto che esistono due principali concetti di SmartHome al momento sul commercio. Ovvero i sistemi integrati (centralizzati) e i sistemi tradizionali (distribuiti), qui denominati “Commerciali” per quanto riguarda quelli tradizionali e Personalizzati per quanto riguarda quelli integrati. 
Con devices commerciali si intendono tutti quei devices cosiddetti “aftermarket” ovvero che vengono acquistati dopo l’acquisto della casa, chiamati anche semplicemente “kit”. Questo particolare tipo di sistema è costituito da vari dispositivi prodotti da Brand famosi e che in un certo senso convertono i vecchi sistemi con quelli nuovi e Smart. 
A differenza dei devices personalizzati essi funzionano principalmente wireless e sono completamente modulabili, ovvero le persone possono decidere quanto mettere e dove in un secondo momento (dopo la costruzione dell’immobile). Questi sistemi comunicano direttamente con il dispositivo mobile del proprietario e devono essere sempre collegati al WiFi di casa. Questi dispositivi vengono anche detti “intelligenti” perché è come se fossero dotati di intelligenza propria, dato che si autoregolano e possono anche comunicare tra loro e ottenere la miglior soluzione possibile.
 
In commercio ci sono una miriade di opzioni, bisogna però capire che in questo caso, l’immaginazione limita molto di più, a causa del fatto che tutti i dispositivi si vedono e tendenzialmente non si effettuano veri e propri “lavori” per quanto riguarda l’istallazione dei dispositivi commerciali. 
Infatti, il concetto che sta alla base di tutti i dispositivi commerciali è il “plug and play”, quindi è voluto che per installare questi componenti non sia necessario alcun tipo di intervento strutturale, basta installare e usarli. 
Ad esempio, le lampadine Smart come le Hue di Philips non necessitano di alcuna modifica all’impianto elettrico: basta avvitare la lampadina, accendere la luce e il gioco è fatto. Questo avviene grazie alla compattezza del sistema. In pratica è la lampadina stessa che al suo interno ha un ricevitore e un Hardware personalizzato e connesso al WiFi, il quale riceve le informazioni dal cellulare. 
Spiegato a livello pratico: il segnale parte dal telefono, arriva al router del WiFi e viene immediatamente reindirizzato alla lampadina che svolgerà le funzioni precedentemente richieste. 
Tra i sistemi più famosi troviamo grandi Brand come Nest che al momento è al primo posto per quanto concerne il sistema di climatizzazione. Però a livello generico i colossi del mercato in questo momento sono due: Amazon Echo e Google Home. Entrambi i sistemi sono controllati tramite comando vocale e sono anche intelligenti, nel senso che riescono a rispondere a determinare domande che gli vengono poste. Un po’ come Siri sull’iPhone in pratica. Di seguito invece vi è la classifica dei migliori otto sistemi  per SmartHome del 2019:
\begin{itemize}
    \item{Amazon Echo}
    \item{Philips Hue}
    \item{TP-Link HS200}
    \item{Ecobee4}
    \item{NetGear Arlo Q}
    \item{Char-Broil Digital Electric Smoker with SmartChef Technology}
    \item{Perfect Bake Pro}
    \item{Ecovacs Deebot N79S}
\end{itemize}
