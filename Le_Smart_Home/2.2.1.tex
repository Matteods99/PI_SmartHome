
Essendo una vera e propria lavorazione, sono stati elencati e spiegati alcuni elementi chiave da Domenico Trisciuoglio  nel suo libro “Introduzione alla Domotica” affinché vi sia una realizzazione del progetto domotico. Le fasi sono le seguenti:
\begin{itemize}
    \item{Analisi delle esigenze del cliente}
    \item{Decisione dei livelli di automazione}
    \item{Progetto di massima e quantificazione del budget (preventivo)}
    \item{Stesura del progetto definitivo}
    \item{Realizzazione dell’impianto}
    \item{Messa in servizio e verifiche}
    \item{Consegna al cliente}
\end{itemize}
Il primo elemento è l’analisi delle esigenze del cliente, fase in cui si decide cosa è più o meno necessario all’interno dell’abitazione. Vengono inoltre valutate le esigenze anche in base a cosa esattamente il cliente vuole rendere automatico e dove.
In secondo luogo, vi è la decisione dei livelli di automazione, la fase in cui si decide letteralmente “quanto” rendere automatiche determinate funzioni della casa. Questa fase della realizzazione di una SmartHome è delineata solamente da due cose: la fantasia e ovviamente il prezzo. Ci sono vari livelli di automazione, ad esempio c’è differenza tra il “tenere fissa la temperatura” e “adattare i vari flussi di corrente di aria calda o fredda e umidità a dipendenza delle preferenze della persona che entra nella stanza, ovviamente con riconoscimento facciale”. Ci sono di conseguenza anche prezzi diversi.
Successivamente, la terza fase è quella in cui si decide il progetto in linea di massima e si qualifica il budget. In modo particolare questa fase permette di stendere la versione di massima del progetto e di conseguenza avere immediatamente un feedback inerente i vari costi. In questa fase, a differenza delle altre, è possibile che, una volta sottoposto il tutto a un determinato cliente, quest’ultimo decida di ripartire dall’inizio poiché non soddisfatto del risultato. 
La quarta fase di realizzazione delle Smart Home vi è la stesura del progetto definitivo, momento in cui, dopo aver ricevuto il via libera del cliente si può procedere con le varie operazioni. Il passo immediatamente successivo è quello della realizzazione dell’impianto, ovvero la parte più interessante di tutto il processo, ma in realtà, data l’estrema facilità di applicazione dei dispositivi nelle abitazioni, questo è anche il passaggio concretamente più semplice. Questa fase a livello di difficoltà è direttamente proporzionale a come sono state fatte le precedenti fasi, nel senso che se le altre fasi sono state effettuate in modo efficace ed efficiente sarà possibile fare un’installazione dei devices del caso in modo molto semplice e veloce. Se nelle precedenti fasi invece sono stati omessi alcuni passaggi sarà di conseguenza più difficile effettuare le varie installazioni all’interno dell’abitazione.
L’ultima fase è la messa in servizio e la relativa verifica degli impianti. Questa fase delinea, in modo molto elementare, la consegna. Infatti, sarà possibile ridiscutere gli ultimi dettagli a lavori finiti, oppure scegliere tra alcune varianti di Sofwares nel caso in cui alcune funzioni implementate non dovessero essere di gradimento del cliente finale.
