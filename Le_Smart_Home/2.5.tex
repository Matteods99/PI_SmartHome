Nelle Smart Home ovviamente la cosa più importante sono i devices e soprattutto le funzionalità che ne derivano. Praticamente ogni stanza della casa è potenzialmente automatizzabile o comunque sfruttabile a livello domotico.
Ci sono alcuni esempi di Smart Homes nel mondo che sono semplicemente belli da togliere il fiato. Come già detto precedentemente alla fine l’unico limite è l’immaginazione. Spetta solamente al proprietario dell’abitazione decidere dove vuole intervenire o meno e soprattutto quanto vuole spendere.
Ci sono molteplici “settori” in cui si può intervenire, per essere più specifici sono una sorta di gestioni differenti tra loro. In ognuna di queste gestioni si situano alcune funzioni Smart anch’esse diverse tra loro. 
Le principali gestioni  sono quattro e sono elencate qui di seguito:
\begin{itemize}
    \item{Gestione ambientale}
    \item{Gestione e controllo carichi}
    \item{Gestione della sicurezza}
    \item{Gestione dell’informazione e della comunicazione}
\end{itemize}
La prima gestione è quella incentrata sul controllo dell’ambiente, infatti questa sezione comprende tutte quelle che sono ad esempio le gestioni climatiche. Come già detto in precedenza questo tipo di automazione si occupa principalmente di modificare a piacimento le temperature e l’umidità all’interno delle varie stanze della casa. In secondo luogo, si occupa del controllo dei consumi, i quali dovrebbero venire ridotti il più possibile. Questo tipo di funzionalità è molto vario a livello di sofisticatezza, ad esempio il sistema può semplicemente svolgere alcune funzioni molto basilari come impostare diverse temperature per le stanze, come può anche essere molto sofisticato tramite le funzioni che includono la regolazione della luce solare come metodo di riscaldamento, regolando di conseguenza le tapparelle, oppure spegnere il riscaldamento automaticamente quando non c’è nessuno in quella stanza o addirittura in tutta la casa. 
Sempre all’interno della gestione ambientale esiste un ventaglio di funzioni dell’illuminazione. Esse sono davvero molto variegate tra loro. Si parte dalla basilare regolazione dei Dimmer della luce (regolazione dell’intensità) e si arriva all’impostazione di determinati colori e luce naturale (o luce artificiale) all’interno delle varie parti della casa, il tutto prontamente scelto secondo i gusti dei vari ospiti. 
La seconda gestione invece è prettamente incentrata sulla gestione dei “carichi” ovvero tutti gli elettrodomestici. Questo particolare tipo di gestione è davvero completamente personalizzabile non solo a livello di sofisticatezza ma anche a livello numerico, perché è il proprietario a decidere quali elettrodomestici mettere in casa e di conseguenza quali automatizzare secondo le sue necessità. Ad esempio, non ha motivo di esistere una culla per bambini smart all’interno dell’abitazione di un soggetto che vive da solo. 
In questa particolare gestione si situano praticamente tutti gli apparecchi di cui conosciamo l’esistenza, però Smart. 
Ad esempio, ci sono i frigoriferi (come quello di Samsung) con un enorme schermo sulla parete esterna del frigorifero: questo schermo può venire utilizzato come lista della spesa che comunica direttamente con il proprio smartphone oppure lo si può utilizzare per vedere la propria agenda della giornata. Inoltre, esistono anche le lavatrici smart, che si regolano da sole e comunicano sempre con il proprio dispositivo cellulare. Successivamente esistono frullatori, forni, lavastoviglie e molti altri apparecchi che comunicano a distanza ciò che sta accadendo in tempo reale. Una delle particolarità di questo tipo di elettrodomestici è che hanno funzioni di auto-diagnosi. Giustifica che esse potranno individuare e tramettere da sole il danno alla casa madre e in alcuni casi addirittura autoripararsi. 
La terza gestione invece è totalmente focalizzata sulla sicurezza. Questo tipo di gestione è fondamentalmente una delle più importanti nella propria abitazione, dato che la sicurezza è forse il motivo per cui a casa quasi tutti si sentono bene. Ad aumentare questo senso di sicurezza anche qui la Smart Home  ha adottato delle soluzioni davvero molto intelligenti. La parola chiave è “videosorveglianza”, infatti questo concetto che è già presente da un po’ nelle case di molte persone, è andato a rinforzarsi grazie all’IoT e alla continua comunicazione tra il dispositivo del proprietario e ciò che i sensori e le videocamere vedono. 
Uno degli esempi più famosi è quello del videocitofono. Grazie a questo dispositivo è possibile vedere a distanza chi sta suonando al campanello in quel preciso momento ed è inoltre possibile comunicare direttamente con l’ospite che ha suonato il campanello direttamente dal proprio dispositivo mobile. 
Un altro esempio di sicurezza sono tutti i sensori interconnessi e che comunicano sempre con il proprietario o direttamente con i soccorsi. Alcuni tipi di sensori sono: quegli degli allagamenti, degli incendi e addirittura i sensori per delle eventuali fughe di gas. Un altro esempio è quello inerente alle persone anziane, le quali hanno ancora più difficoltà a muoversi o a compiere determinate azioni quotidiane. Nel caso in cui una persona anziana dovesse cadere in bagno, il sistema lo capirebbe e chiamerebbe immediatamente i soccorsi. 
La quarta e ultima gestione è basata sulla comunicazione. Si parla di comunicazione sia all’interno della casa che all’esterno. Anche questo tipo di gestione si basa sulle preferenze del proprietario ed esistono vari livelli di sofisticatezza anche in questo caso. Per essere più specifici, molte persone hanno i sistemi di diffusione sonora multiroom, però davvero pochi hanno un sistema che capisce i gusti musicali della persona che è appena entrata nella stanza. 
Alcuni esempi di gestione comunicativa domotica all’interno dell’abitazione sono ad esempio il tipo di televisore, dato che alcuni modelli nuovi di Samsung lavorano assieme al telefonino e riconoscono i gusti del proprietario. Sono inoltre presenti sul mercato sistemi che permettono l’interconnessione tra il dispositivo mobile e le videocamere all’interno delle varie stanze, quindi si può vedere cosa sta succedendo nella propria casa anche mentre non si è fisicamente all’interno di essa.
Per fare un esempio di uno dei sistemi domotici più famosi e meglio strutturati al mondo non può non risaltare la casa Gates, la dimora del celeberrimo fondatore di Microsoft: Bill Gates. 
Casa sua è uno dei migliori esempi di SmartHome che si possano trovare. Giusto per fare un semplice esempio, il suo sistema di diffusione sonora è a dir poco sensazionale, poiché dato che ogni persona della famiglia ha un Chip che lo riconosce, esso è capace di tracciare i movimenti di tutti i componenti che vivono all’interno della casa. Ma la parte più interessante è che questo chip fa si che una persona, man mano che si sposta all’interno della casa, riesca ad ascoltare i suoi brani preferiti. Ciò vuol dire che gli altoparlanti si attivano solamente se una persona gli si avvicina in qualche modo. Ovviamente risulta spontaneo chiedersi: ma se ci sono due persone nella stessa stanza? Il sistema è talmente avanzato che riuscirà a trovare un genere che risulta un compromesso tra i differenti gusti musicali. 
