Quando si chiede ad una persona di immaginare il posto in cui preferirebbe svegliarsi, probabilmente risponderebbe: “nel mio letto”. Magari nella casa dei propri sogni 
È risaputo che ciò che una casa da all’uomo è molto più di un rifugio, infatti il concetto di casa si è molto evoluto nel tempo, cambiando e mutando col passare degli anni.
Oggi l’idea perfetta di casa non è sicuramente la stessa che avevano i nostri antenati prima di noi. 
Le case si sono evolute al fine di aumentare in modo esponenziale il comfort e diminuire al minimo gli sprechi, consumi e l’impatto ambientale.
Negli ultimi anni si è cominciato a sentire parlare di “queste” famose Smart Home oppure, più generalmente, di domotica (domus: casa e ticos: applicazione)
Ma cosa sono le Smart Home? Come funzionano? Come si sono diffuse?
Queste sono le domande alle quali si cercherà di dare risposta, unite ad altri approfondimenti sul tema, tra cui i sistemi “personalizzati” o “commerciali”, per concludere delineando alcuni aspetti chiave sulla manutenzione, senza tralasciare alcuni esempi fondamentali che aiuteranno a comprendere a fondo in che modo le Smart Home riescono a migliorare la vita di tutti i giorni.
