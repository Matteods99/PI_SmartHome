Per concludere il capitolo, risulta piuttosto ragionevole fare delle considerazioni finali su quanto detto finora. Dovrebbe essere chiaro ora cosa sono le Smart Home, come funzionano e come di sono diffuse. Inoltre, dovrebbe essere chiaro il fatto che esistono due principali tipi di sistemi: “personalizzati” o “commerciali”. Oltre a ciò sono stati spiegati alcuni esempi di applicazioni concrete di SmartHome, oltre al concetto di manutenzione. È stato inoltre possibile delineare alcuni processi chiave nella costituzione di una SmartHome.
Infine, ora dovrebbero essere acquisite tutte le possibili gestioni e funzioni dei devices che sono presenti nelle SmartHome.
Tutto considerato passare ai sistemi SmartHome, nonostante l’investimento iniziale più ingente, risulta, a lungo termine, piuttosto interessante e soprattutto utile nella vita quotidiana.
