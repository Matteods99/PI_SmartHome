Se prima si parlava di sistemi Commerciali (o distribuiti) ora si parla di sistemi Personalizzati (integrati). 
Questo particolare sistema è stato il primo ad essere inventato, dato che utilizza un tipo di sistema diverso, anche per quanto riguarda la comunicazione.
Questo tipo di sistema è davvero completamente personalizzabile secondo ogni minimo desiderio del consumatore, dato che può venire installato solo quando viene costruita la casa. È possibile anche installarlo dopo, ma nessuno lo fa perché non conviene e soprattutto perché comporterebbe l’intervento di terzi per l’installazione. 
Non c’è nulla di semplice o “plug and play” nei sistemi integrati, perché è un sistema che viene installato come se fossero i cavi del telefono per intenderci. È un tipo di sistema che funziona con una centralina, la quale è collegata (tendenzialmente grazie alla power line) con tutti i vari dispositivi, ognuno dei quali deve venir preimpostato nel momento dell’installazione in una determinata parte della casa. 
Per intenderci, l’esempio di prima della lampadina in questo caso non va più bene, perché se il proprietario decidesse di rendere una luce Smart, sarebbe per forza necessario installare un nuovo sistema dedicato all’uso di quella specifica lampadina, infatti il sistema non sarebbe più nella lampadina stessa, ma sarebbe nel muro. 
Il sistema personalizzato però lascia davvero carta bianca al proprietario al momento della costruzione dell’immobile. Bisogna inoltre dire che essendo un sistema completamente installato da professionisti, rimane un impianto molto efficiente e soprattutto praticamente privo di malfunzionamenti. Nonostante ciò rimane comunque il fatto che è un sistema molto sofisticato e tecnico. 
Invece si sa quanto è stabile la rete wireless, potrebbe sempre crashare il sistema. Poi bisogna pensare che nei sistemi commerciali, ogni volta che si riavvia il router tutti i dispositivi smettono di funzionare fino a riavvio completo, cosa che con i sistemi integrati non succede.
In questo caso la concorrenza è meno agguerrita, ma sicuramente molto presente rispetto ai sistemi più commerciali. Tra le marche più famose ci sono AMX, Vimar e anche B-Ticino.
