Le abitazioni intelligenti puntano a promuovere e sostenere questo modello di economia.  “La green economy, in italiano economia verde, è una forma di economia in cui gli investimenti mirano a ridurre le emissioni di carbonio e l’inquinamento, ad aumentare l’efficienza energetica e delle risorse, a evitare la perdita di biodiversità e conservare l’ecosistema ”. Si può dire quindi che è un concetto volto a migliorare il benessere umano e l’equità sociale riducendo significativamente i rischi ambientali e le scarsità ecologiche .
I temi trattati dall’economia e quindi dalla crescita sostenibile sono al giorno d’oggi argomenti di forte attualità poiché tutti i consumatori e cittadini sono sempre più attenti alla qualità dell’ambiente. Anche se possono sembrare due elementi che non hanno niente in comune, economia e rispetto per l’ambiente non devono assolutamente essere visti come antagonisti bensì come obiettivi da perseguire in ugual maniera nel senso che un’abitazione che risulta maggiormente efficace per l’ambiente potrebbe incentivare i clienti a preferirla al posto che un’abitazione tradizionale che non può ancora automatizzare il suo funzionamento e ridurre i consumi. Questa politica economica può quindi scaturire un riscontro positivo per i risultati economici delle imprese del settore edile. È un concetto molto sentito soprattutto dove sono presenti molte abitazioni poiché vi è un incremento esponenziale di inquinamento, consumo delle risorse energetiche e naturali .
