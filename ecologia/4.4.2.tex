Il concetto di sviluppo sostenibile è definito come “sviluppo in grado di garantire il soddisfacimento dei bisogni attuali senza compromettere la possibilità delle generazioni future di far fronte ai loro bisogni ”.
Il concetto di “edilizia sostenibile” è stato introdotto e sviluppato già a partire dal 1970 per rispondere alla crisi energetica e rispettare aspetti sociali molto presenti e significativi da alcuni decenni. Si tratta dei problemi ambientali che la Terra sta subendo a causa di svariati fenomeni. Questa preoccupazione ha dato infatti inizio ad un nuovo modo di concepire il settore edile, prestando sempre più attenzione agli aspetti ecologici. Pur essendo un settore in rapida e costante evoluzione, solo oggi dopo quasi 50 anni, questo concetto si sta diffondendo diventando una realtà quotidiana. 
Oggigiorno la maggior parte delle persone si sofferma sulla qualità degli edifici in termini di involucro ed impianti senza studiare la parte più importante ossia la gestione della struttura stessa. Elemento questo strettamente legato allo stile di vita degli occupanti dell’abitazione .
Una delle sfide più importanti da affrontare a livello globale è proprio la riduzione del consumo di energia infatti negli ultimi anni si sta andando sempre più nella giusta direzione poiché la maggior parte dei nuovi edifici è già in partenza dotato di un livello di intelligenza maggiore rispetto alle abitazioni progettate e costruite in precedenza. Da non dimenticare il fatto che una Smart Home può essere in qualche maniera introdotta attraverso vari dispositivi in una casa già esistente da anni. Questo è concepibile per merito della tecnologia e l’Internet of Things.
Contemporaneamente all’aumento del livello di sostenibilità ambientale vi è anche quella economica .
Paradossalmente sono considerati quindi gli edifici “intelligenti” quelli più compatibili con l’ambiente.
Infatti il fattore dell’automazione si sposa perfettamente con la sostenibilità poiché sfruttando l’energia solare per riscaldare gli ambienti e anche grazie al fatto che quest’intelligenza artificiale inserita nei vari dispositivi è in grado di elaborare continuamente e in tempo reale un’elevatissima quantità di dati, consente ai proprietari delle Smart Home di tenere sempre sotto controllo i consumi e rispettivamente i costi, evitando cosi sprechi eccessivi .
L’edilizia sta puntando ininterrottamente sulle Smart Home. Smart è di fatto la parola chiave che caratterizza ed identifica le abitazioni in cui si abiterà in un breve futuro. Questo termine ci ricollega a più fattori perché uno stile di vita più facile per merito di un comfort maggiore e l’aumento della sicurezza non sono abbastanza per poter definire una casa Smart poiché quest’ultima rivolge anche un particolare interesse all’ottimizzazione dei consumi energetici .
