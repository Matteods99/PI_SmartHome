L’energia è un elemento fondamentale quando si parla di case intelligenti. I vari accorgimenti legati ad un impiego più razionale delle fonti energetiche sono strettamente correlate con un minor impatto ecologico. Hans-Peter Burkhard, Direttore del Centro per una politica aziendale ed economia sostenibile (CCRS) dell’Università di Zurigo, afferma che “per edifici sostenibili si interndono le case energeticamente efficienti” .
È un elemento rilevante poiché è sempre tra i primi motivi, oltre  alla possibilità di avere la propria abitazione sotto controllo ed una maggiore comodità, che spinge una persona a costruire proprio una Smart Home per il suo risparmio energetico che porta di conseguenza, come citato dal direttore dell’Università di Zurigo, ad un efficacia ed un rendimento energetico notevole.
L’utilizzo dell’energia che troviamo in queste case domotiche si sta integrando sempre più con il concetto di intelligenza artificiale. Si sta infatti evolvendo grazie alle nuove tecnologie ed il suo percorso è legato al fatto che determinati dispositivi sono in grado di misurare ed ottimizzare l’ambiente in cui viviamo.
Il termine utilizzato per determinare quest’energia è digital energy . Essa è stata sottoposta ad un’analisi dall’Energy e Strategy group del Politecnico di Milano ed indica “l’uso di tecnologie digitali sempre più avanzate lungo la filiera dell’energia ”. È un concetto basato su due fattori principali: la distribuzione e la misurazione. Entrambi gli aspetti passano attraverso lo sfruttamente dei sistemi di hardware e software, sistemi in grado di elaborare dati, monitorare l’elaborazione di questi dati che riguardano i consumi energetici. Questi sono defiiti i big data. I big data sono come “un ingente insieme di dati digitali che possono essere rapidamente processati da banche dati centralizzate ”. Grazie alle tecnologie digitali presenti al giorno d’oggi è possibile gestire questi dati reali immediatamente ed in ogni luogo. Ci sono cinque principali motivi che spingono a favore dell’utilizzo della digital energy :
\begin{enumerate}
\item  Decentralizzazione: attraverso reti interconnesse si possono offrire sistemi energetici pi\`{u} flessibili e soggetti a meno perdite.

\item  Comportamento dei clienti: i clienti sempre pi\`{u} esigenti, vogliono essere in grado di poter monitorare i propri consumi.

\item  Rischi tecnici: si pu\`{o} ridurre la durata di eventuali interruzioni di corrente elettrica o altre problematiche relative a questo problema in quanto la gestione \`{e} molto pi\`{u} efficace e veloce

\item  \textit{Cyber security: }questa digitalizzazione richiede attente valutazioni ai rischi legati alla sicurezza in termini informatici, importante quindi il fatto che siano reti protette con le tecnologie pi\`{u} affidabili e recenti in termini di innovazione.\textit{}

\item \textit{ }Norme e incentivi: le politiche energetiche insistono sulla sostenibilit\`{a} di tipo ambientale; aumento della produzione dalle fonti rinnovabili e riduzione di emissioni di sostanze nocive.
\end{enumerate}
Attravero le Smart Home e le nuove opportunità offerte dal continuo sviluppo della tecnologia, del web e dell’Internet of Things tutti gli elettrodomestici presenti in un’abitazione sono collegati alla rete per una migliore gestione degli ambienti domestici. Questi oggetti Smart, sempre più presenti nelle abitazioni, possiedono una potenzialità di calcolo incredibile che in maniera completamente automatica si personalizza e diventa maggiormente efficiente a secondo delle esigenze e delle preferenze di ogni abitante della casa. Grazie a questo, è fondamentale il fatto che nelle Smart Home, il consumo di energia può essere misurato durante l’arco di tutta la giornata  . I dati che si ricavano da queste analisi sono utili per chi abita nella casa e permette loro di essere più consapevoli dei propri consumi e limitare nel caso il consumo stesso, specialmente se questo dovesse essere smisurato. Ancora meglio sarebbe se i consumatori fossero consapevoli dei danni ambientali che la produzione dell’energia che la popolazione utilizza per svolgere le attivita quotidiane provoca.