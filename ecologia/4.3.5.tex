Il fattore principale è rivolto al risparmio sui consumi domestici grazie ad un monitoraggio costante degli apparecchi presenti nell’abitazione. 
Infatti all’interno delle Smart Home è presente un nuovo dispositivo che consente di monitorare i consumi domestici. L’efficienza energetica passa infatti attraverso un controllo dei consumi degli elettrodomestici casalinghi grazie ad una gestione che è integrata direttamente nei sistemi. Questo monitoraggio è garantito dallo sviluppo del NED, questo dispositivo è uno smart meter comunemente chiamato “contatore telegestito” . Sviluppato e creato dalla PMI Midori dell’Incubatore 13P del Politecnico di Torino.Attraverso dei sensori è in grado di tenere sempre sotto controllo ed informare costantemente i proprietari della Smart Home dei consumi. Esso presuppone una maggiore valutazione della proprio spesa in quanto il consumatore sarà più informato e consapevole di ciò che accade. Questo contatore è in grado di connettersi automaticamente con tutti i dispositivi di casa senza l’obbligo di dover installare numerosi, costosi ed invasivi strumenti di misura . Questo apparecchio quindi segnala i consumi eccessivi e le eventuali anomalie elettriche, per poterle controllare meglio. Una volta collegato al quadro elettrico, tramite lo smartphone, più precisamente con l’uso di un’applicazione, è possibile scoprire in tempo reale l’utilizzo dell’energia e imparare a risparmiare sulla bolletta fino al 20\% ogni anno. Si è in grado finalmente di fornire informazioni rilevanti riguardanti tutti gli elettrodomestici presenti nell’abitazione .
