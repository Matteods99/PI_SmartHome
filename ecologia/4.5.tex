In conclusione le Smart Home permettono una riduzione del consumo energetico con conseguente diminuzione dell’impatto ecologico. Malgrado l’impossibilità di reperire dati concreti e precisi sulla differenza numerica del risparmio energetico tra una casa tradizionale ed una Smart Home, possiamo affermare con certezza che grazie al suo funzionamento interno automatizzato, all’utilizzo di energie rinnovabili ed al costante monitoraggio, analisi e controllo dei consumi, le case intelligenti sotto efficienti ed ecologiche sotto tutti gli aspetti. Lo studio e lo sviluppo dell’intelligenza artificiale sarà sicuramente ampliato nei decenni a venire.  L’efficienza ed il risparmio energetico rappresentano un chiaro e sostanziale valore per la società. L’accuratezza e la progettazione di un design moderno, che si riallaccia anche alla sostenibilità ambientale, può quindi fruttare dei grandi progetti. In questo caso si parte sempre però dall’approccio bioclimatico che la costruzione di queste abitazioni può avere poiché si presta molta attenzione all’orientamento dell’abitazione rispetto al sole e alla ventilazione naturale, all’attenzione per la riduzione del fabbisogno energetico per riscaldare e raffreddare i locali in cui la gente abita, il ricorso alle fonti rinnovabili citate in precedenza come il solare fotovoltaico ed infine all’efficienza degli impianti che permette la riduzione dei consumi .
L’estrema attualità e crescita dei temi legati all’inquinamento atmosferico e dall’efficienza energetica, stanno sensibilizzando sempre una fetta maggiore di cittadini che, diventando più attenti e sostenibili prestano maggiore cautela nella costruzione delle proprie abitazioni creando costantemente maggiore interesse nei confronti delle soluzioni abitative capaci di rispettare l’ambiente e migliorare in maniera rilevante la qualità della vita. “Il futuro è Smart ”.
