Durante il 2010 la Svizzera ha generato complessivamente 66 TWh di energia elettrica. Per quanto riguarda questa produzione, hanno contribuito per il 57\% le centrali idroelettirche, per il 38\% quelle nucleari e per il 5\% le restanti centrali elettriche. Se viene consierato al netto, il consumo energetico di tale produzione è stato di 60 TWh. L’equivalente di 1TWh corrisponde ad 1 bilione di Wh e per arrivare ad 1 kWh occorrono infine 1000Wh. Il consumo energetico, tematica di grande attualità,  è in continuo aumento in Svizzera come in qualsiasi altra parte del mondo. Se volessimo rendere più chiara l’idea delle immense dimensioni di cui stiamo parlando, l’Ufficio federale dell’energia ha usato l’immagine di un lago di petrolio spiegando che il consumo energetico annuo in Svizzera avrebbe la superficie del Lago di Neuchâtel che corrisponde a 218km2 e una profondità di 70m . 
Negli ultimi 40 anni, questo consumo è raddoppiato e vedendo l’andamento generale non è prevista nessuna diminuzione bensì un continuo aumento.
In questo grafico si trova il consumo di energia svizzero per le categorie di utilizzazione dal 2000 al 2009. Si può vedere quali sono i consumatori che più ne hanno usufruito e si nota più chiaramente che le abitazioni, intese nell’immagine come economie domestiche, consumano il 29.8\% dell’energia complessiva superando dell’11\% le industrie .
Le abitazioni consumano con le aziende insustriali e artigianale ne utilizzano circa 1/3 a testa mentre il settore dei servizi 1/4.
Dal 2005, grazie ai molteplici avanzamenti tecnologici, il consumo energetico degli apparecchi domestici è in continua diminuzione anche se in miura molto contenuta rispetto al potenziale fornito dall’imponente miglioramente tecnico.
Per quanto rigurda la Svizzera, grazie ai dati fornitici dall’Ufficio federale dell’energia (UFE), si conosce che l’aliquota dell’energia elettrica totale che fluisce desplicitamente alle economie domestiche raggiunge il 31\%. 
Mediamente il consumo energetico annuale per una famiglia risulta pari a 5400kWh .
Nel 2017 il consumo energetico di tutte le abitazioni Svizzere è stato complessivamente di 3.428GWh. Sapendo che 1 gigawattora corrisponde esattamente a 1000000 chilowattora sappiamo che questo consumo è stato di 3428000 kWh.
In generale, nelle abitazioni, l’80\% dell’energia è necessario per riscaldare l’edificio stesso mentre il restante 20\% viene utilizzato per gli altri apparecchi elettrici e l’illuminazione. 
Con l’avvento delle nuove tecnologie l’orientamento è sempre più indirizzato su vettori energetici sempre più sostenibili.
L’elettricità risulta comunque un elemento indispensabile nelle nostre case. Questa deve però sottostare correttamente ad impianti sicuri in modo da evitare sprechi di energia ma anche per diminuire in maniera considerevole la possibilità che si sviluppi un incidente. Emerge quindi l’importanza di costruire degli impianti sempre più moderni e tecnologicamente avanzati come quelli presenti nelle Smart Home così che siano persino più efficienti. Un punto essenziale si trova nell’utilizzo dell’energia rinnovabile che oltre che avere dei costi sul lungo termine enormemente inferiori,  tutelano maggiormente l’ambiente. L’evoluzione tecnologica unita ad un impiego razionale ed intelligente degli elettrodomestici di casa, oltre che ad avere i benefici spiegati in precedenza, fanno in modo che ci sia una quantità maggiore di energia per tutta la popolazione.
