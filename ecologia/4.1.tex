“La terra è ferita, serve una conversione ecologica”. Questa frase è stata inclusa nell’enciclica “Laudato sì” di Papa Francesco .
Partendo da questa considerazione, e sapendo che tutti i processi causati dall’uomo si ripercuotono sull’ambiente, con questo lavoro si analizzerà l’impatto ambientale delle Smart Home rispetto alle abitazioni tradizionali.
L’impatto ambientale è definito come “alterazione da un punto di vista qualitativo e quantitativo dell’ambiente, considerato come l’insieme delle risorse naturali e delle attività umane a esse collegate, conseguente a realizzazioni di rilevante entità .”
Pur sapendo che le Smart Home non nascono espressamente per risolvere problemi ecologici bensì per aumentare il comfort e vivere la quotidianità con meno stress e più tranquillità attraverso soluzioni personalizzate  , ci domandiamo se queste case intelligenti hanno anche un minor impatto ambientale rispetto alle abitazioni usuali. Partendo da questa domanda di ricerca si approfondirà se la domotica, nata per migliorare il comfort di vita, possa diventare uno strumento per ridurre in modo significativo il consumo di energia poiché questa riduzione è strettamente correlata con la preservazione dell’ambiente. La Smart Home, oltre che essere sotto un certo punto di vista più sicure, confortevoli, funzionali, flessibili e di semplice utilizzo, permettono infatti di avere un rendimento energetico più efficiente che punta sul risparmio di energia e quindi su una vita più sostenibile .
Essendo ben noto il fatto che il tenore di vita di un individuo è strettamente correlato con il fattore energetico, con questo lavoro ci concentreremo principalmente proprio su questo aspetto poiché il problema dell’elettricità è strettamente correlato alla tutela dell’ambiente in quanto la sua produzione non solo utilizza grandi quantitativi di risorse naturali che mettono così in pericolo l’ambiente in cui viviamo, ma emette ulteriori sostanze che lo influenzano negativamente .
4.2.	Il consumo energetico di un’abitazione 
