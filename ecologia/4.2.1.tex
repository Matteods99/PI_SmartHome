L’energia elettrica può essere prodotta in vari modi. Nelle centrali termoelettriche si ottiene bruciando olio combustibile, gas o carbone, nelle centrali idroelettriche si ottiene utilizzando l’acqua. Grandi quantità di energia vengono invece prodotte dalle centrali nucleari che sfruttano l’energia racchiusa nel nucleo degli atomi .
La combustione dell’olio combustibile, del gas e del carbone ha un impatto negativo diretto sulla qualità dell’aria presente nell’atmosfera poiché il processo necessario per la sua realizzazione grava direttamente sull’ambiente, danneggiandolo a causa di emissioni di sostanze tossiche quali: CO2, zolfo e mercurio.
L’unità di misura dell’energia elettrica è il chilowattora (kWh) che corrisponde al lavoro compiuto da una macchina che sviluppi una potenza costante di 1 chilowatt (=100watt) per una durata di un’ora. È quindi il risultato della moltiplicazione della potenza di un tale apparecchio per il suo tempo di funzionamento . 
L’energia presente negli edifici in cui viviamo è prevalentemente l’energia elettrica. Essa è per definizione l’energia associata all’elettricità, in particolare l’energia di una corrente elettrica. Questo risultato deriva da processi di trasformazione di altri tipi di energia . 
La disponibilità di tali risorse è un fattore fondamentale nello sviluppo di un paese poiché la trasformazione dell’energia cinetica in energia elettrica permette energia luminosa, termina e meccanica indispensabile per il funzionamento di un’abitazione . 
Al giorno d’oggi nella società in cui viviamo è indispensabile usufruirne per poter avere una vita “normale” poiché essendoci abituati a tale comfort risulta difficile viverne senza.
Gran parte dell’energia che viene prodotta globalmente viene usata proprio all’interno delle case e viene impiegata per svolgere diverse funzioni quali: riscaldare l’ambiente e l’acqua, erogare elettricità e per mettere il funzionamento di tutti i dispositivi elettrici e gli elettrodomestici .
In Svizzera sarebbe praticamente impossibile vivere senza energia elettrica in casa. Prova ne è che quando per un guasto alla rete elettrica un’abitazione non può più ricevere la corrente gli occupanti sono “persi”. Buona parte delle attività della vita quotidiana, così come siamo abituati a viverle, diventano inattuabili. Tutte le attività non possono più essere svolte, dall’accendere la luce, al guardare la televisione, al cucinare. Ecco che il ruolo dell’energia nelle case tradizionali diventa essenziale.
