Siamo a conoscenza del fatto che uomini e donne, per svolgere la vita quotidiana in un paese industrializzato, necessitano svariate risorse ma quella che gioca un ruolo fondmentale è l’energia. L’uomo necessita sempre più energia per svolgere le sue attività, difatti, da circa due secoli a questa parte il fabbisogno energetico è cresciuto enormemente. Questo bisogno eccessivo risulta appunto una delle problematiche più attuali. Lo sfruttamento energetico smisurato sta portando ad una scarsità delle fonti energetiche, che si stanno lentamente esaurendo. Il 90\% dell’energia primaria proviene dai combustibili fossili. Sono una risorsa formidabile che si sta di fatto esaurendo 
Questo aspetto ha creato una priorità nell’essere umano: riuscire da una parte a risparmiare energia e dall’altra a poterla riutilizzare. 
Oltre al fatto che le fonti energetiche possono essere rinnovate, si parla sempre più frequentemente di salvaguardare l’ambiente e ridurre gli sprechi. 
La scarsità di risorse naturali in contrasto con l’incremento demografico ha fatto prendere coscienza all’uomo della problematica. La società sta cercando di diminuire lo spreco di tali risorse e di implementare lo sviluppo e l’utilizzo di energie rinnovabili.
La domotica rappresenta realmente una vera chiave di volta per poter ridurre i consumi. L’obiettivo di questa nuova tecnologia è infatti quella di ottimizzare, in questo caso gli apparecchi domestici, in modo tale da consumare esclusivamente lo stretto necessario, senza portare smisurati sprechi di risorse naturali che, se dovessero esaurirsi, metterebbero in grave pericolo il futuro dell’intera società.
