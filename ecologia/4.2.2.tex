La politica energetica della Confederazione Svizzera, come sottoscritto nell’articolo 89 della Costituzione federale della Confederazione Svizzera, con i Cantoni “si adopera per un approvvigionamento energetico sufficiente, diversificato, sicuro, economico ed ecologico, nonché per un consumo energetico parsimonioso e razionale. La Confederazione emana principi per l’utilizzazione delle energie rinnovabili sempre per un consumo enegetico parsimonioso e razionale .”
Nello specifico degli edifici, l’articolo 44 della Costituzione federale della Confederazione Svizzera dice: “Nell’ambito della loro legislazione, i Cantoni creano condizioni quadro volte a favorire l’impiego parsimonioso ed efficiente dell’energia nonché l’impiego di energie rinnovabili. Sostengono l’attuazione di standard di consumo per l’impego parsimonioso ed efficiente dell’energia. Al riguardo evitano ingiustificati ostacoli tecnici al commercio .” 
Questa legge permette agli abitanti di poter utilizzare determinati dispositivi nelle proprie abitazioni, che applicati all’intelligenza artificiale possono rendere Smart un edificio, in grado di risparmiare energia per poterla riutilizzare nel suo utilizzo stesso recuperando il calore residuo.
Il consumo d’energia degli apparecchi elettrici presenti nelle abitazioni deve essere classificato. Esistono infatti in Europa delle classi di consumo energetico che misurano propriamente l’efficienza energetica degli elettrodomestici ad uso casalingo.
Per valutare l’efficienta dei vari elettrodometisici utilizzati, è presente per legge dal 2011 un’etichetta energetica che giustifichi realmente questa efficienza informando i consumatori su elementi importanti riferiti al determinato apparecchio. Come esposto dall’Associazione settoriale Svizzera per gli Apparecchi elettrici per la Casa e l’Industria, per i venditori ed il commercio in genrale, questo strumento è unicamente oggetto di marketing che influneza la decisione   d’acquisto favorendo la vendita di determinati elettrodomenstici che, messi a confronto con altri che svolgono la stessa funzione, risultano meno efficienti e con un consumo maggiore. Ciò nonostante i produttori sono obbligati a metterla su tutti i prodotti che hanno intenzione di mettere sul commercio .
Quest’etichetta energetica viene inserita nel Certificato degli edifici (CECE) che, identifica la qualità energetica di un determinato edificio suddividendola in efficienza e fabbisogno sulla base di una scala di sette livelli. Esso mostra in aggiunta il potenziale miglioramento energetico di un involucro, fa in modo da poter individuare i suoi punti deboli, offre informazioni relative ad un possibile acquisto ed è unificato a livello svizzero . 
I sette livelli sono esposti attraverso lettere e colori. L’etichetta verde indica la classe con il minor consumo energetico rispettivamente contrassegnato con la lettera A+++ mentre quella rossa, lettera D, rappresenta il consumo maggiore. Le tipologie di apparecchi che devono essere così riconosciuti sono: frigoriferi e congelatori, lavapiatti, lavatrici, asciugatrici, apparecchi per l’illuminazione, televisori, forni e climatizzatori. 
