Uscendo dal contesto svizzero, poiché l’utilizzo della domotica nelle abitazioni non è ancora così implementato, verrà riportata questa norma sull’efficienza energetica a livello europeo. La Commissione europea ha infatti avviato una consultazione per una revisione sulla legge riguardante il concetto di efficienza energetica. Si vuole infatti prestare la massima attenzione alle politiche energetiche-climatiche in sinergia con quelle di rilancio economico. Questa norma europea UNI EN 15232 “Prestazione energetica degli edifici – Incidenza dell’automazione, della regolazione e della gestione tecnica degli edifici” evidenzia come l’introduzione nelle abitazioni sistemi di controllo e di automazione comporta la riduzione del consumo energetico .
Entro il 2020 infatti gli stati dell’UE si impegnano a rispettare la “strategia europea 2020” che si pone principalmente tre grandi obiettivi.
\begin{enumerate}
\item  Diminuire del 20\% le emissioni di gas serra

\item  Aumentare del 20\% l'efficienza energetica

\item  Aumentare del 20\% l'utilizzo dell'energie rinnovabili
\end{enumerate}
Questo progetto viene incentivato grazie all’incoraggiamento dell’efficienza energetica e l’uso razionale delle risorse energetiche, la promozione delle fonti di energia innovative e rinnovabili che incoraggiano la diversificazione energetica.