Dagli anni 2000 è diventato fondamentale riuscire a risparmiare energia per poterla riutilizzare. La tecnologia in continuo sviluppo e l’utilizzo di internet in maniera sempre più prorompente ci fa intuire che l’intera industria sta cambiando in questa direzione. 
Stiamo infatti attraversando una fase che possiamo chiamare rivoluzione energetica. L’europa potrebbe infatti riuscire a rendere prioritarie le fonti rinnvabili e l’efficienza energetica in modo tale che  preventivamente entro il 2020 non si utilizzino più il nucleare e le fonti fossibili. Come spiegato in precedenza, visto che le riserve di combustibili fossili si stanno esaurendo lentamente, garantiscono un approvvigionamente per circa ancora 50 anni . Risulta utile essere a conoscenza delle fonti energetiche che sono rinnovabili e parlare dei loro potenziali di riutilizzo poiché risultano essere fondamentali per l’intero funzionamento di un’abitazione, soprattutto di una Smart Home . La Svizzera dispone di un vettore energetico molto importante poiché molto ricco. Se si guarda ciò che potrebbe accadere sul lungo termine, la Svizzera appare sens’altro un paese con un grande potenziale di sviluppo per quanto riguarda le fonti energetiche rinnovabili. Esistono in realtà grandi prospettive della produzione di energia elettrica .
Abitando in una Smart Home, questo processo di poter riutilizzare l’energia avviene in maniera automatizzata. Per esempio se si parla di un impianto fotovoltaico, ossia un impianto che grazie alla tecnologia fotovoltaica presente al suo interno permette di produrre l’energia trafonrmando le radiazioni solari in elettricità senza l’utilizzo di nessuno combustibile e quindi riducendo le emissioni di sostanze tossiche , si può considerare un nuovo modo, inseguito negli ultimi anni, per affrontare il problema dell’approvvigionamento energetico senza danneggiare l’ambiente. Questa energia può essere usata per riscaldare o conseguentemente raffreddare gli ambienti.
L’energia solare è considerata una delle tecnologie rinnovabili più pulite e soprattutto più sicure. È un impianto che consente ad un proprietario di poter approvvgionarsi da se ed essere quindi autosufficiente senza dover ricorrere a terzi. Oltre che avere agevolazioni fiscali per l’installazione, un aumento del valore dell’intero immobile e risparmio economico, il fatto di installare un impianto fotovoltaico nella propria abitazione permette una riduzione delle emissioni nocive come anidride carbonica nell’aria poiché l’intero funzionamento avviene tramite la luce naturale del sole.
