\documentclass{article}
\usepackage[utf8]{inputenc}
\usepackage[italian]{varioref,babel}
\usepackage{lipsum}
\usepackage{fancyhdr}
\usepackage{graphicx}
\usepackage{booktabs} 
\usepackage{caption} 
\usepackage{tabularx}

\pagestyle{fancy}

\lhead{}
\chead{}
\rhead{}
\lfoot{Matteo Della Santa}
\cfoot{Smart Home}
\rfoot{\thepage}
\renewcommand{\headrulewidth}{0pt}
\renewcommand{\footrulewidth}{1pt}

\begin{document}
\begin{titlepage}
	
	\raggedleft 
	
	\rule{1pt}{\textheight} 
	\hspace{0.05\textwidth}
	\parbox[b]{0.75\textwidth}{ 
		
		{\Huge\bfseries Smart Home: \\[0.5\baselineskip] ~la casa del futuro}\\[2\baselineskip]
		{\large\textit{Scuola Cantonale di Commercio di Bellinzona \\
Progetto interdisciplinare \\ 
Anno scolastico 2018-2019 \\
}}\\[4\baselineskip] 
		{\Large\textsc{Rocco Apolloni 4A \\
Matteo Della Santa 4J \\
Chiara Calatti 4B \\
Ariel Signorotti 4C 
}} 
		\vspace{0.5\textheight} 
		{\noindent~~}\\[\baselineskip]}

    \end{titlepage}
    
\newpage
\tableofcontents
\thispagestyle{empty}
\newpage

\section{Introduzione}
Il lavoro svolto vuole mettere sotto luce alcuni elementi fondamentali che circondano il concetto di Smart Home, al fine di comprendere a fondo gli aspetti positivi e negativi ad essa correlati. In particolar modo, si tenterà di rispondere alla domanda se le Smart Home sono davvero, oppure no, le case del futuro. Per rispondere a quest’ostica domanda è stato necessario analizzare in modo completamente trasversale alcuni principali aspetti delle case Smart. Nello specifico, gli aspetti analizzati sono quattro: 
Il primo è ovviamente quello inerente alle spiegazioni basiche necessarie per comprendere il funzionamento, il mercato, lo sviluppo, la diffusione e altri aspetti più tecnici.
La sezione successiva invece tratta tutto quello che è il conflitto tra sicurezza e privacy all’interno delle Smart Home, delineando, inoltre, alcuni fattori chiave tramite le leggi.
Il terzo capitolo invece, tratterà in modo specifico l’impatto ecologico con il quale proprietari delle Smart Home hanno quotidianamente a che fare. 
L’ultimo capitolo invece, vede coinvolti alcuni casi di studio sulle Smart Home, i quali aiuteranno a capire meglio il mercato ticinese e svizzero al momento attuale, oltre a vedere dal lato pratico ciò che è stato detto nei capitoli precedenti. 

\newpage
\section{Le Smart Home}
A cura di Matteo Della Santa
\subsection{Premessa}
Quando si chiede ad una persona di immaginare il posto in cui preferirebbe svegliarsi, probabilmente risponderebbe: “nel mio letto”. Magari nella casa dei propri sogni 
È risaputo che ciò che una casa da all’uomo è molto più di un rifugio, infatti il concetto di casa si è molto evoluto nel tempo, cambiando e mutando col passare degli anni.
Oggi l’idea perfetta di casa non è sicuramente la stessa che avevano i nostri antenati prima di noi. 
Le case si sono evolute al fine di aumentare in modo esponenziale il comfort e diminuire al minimo gli sprechi, consumi e l’impatto ambientale.
Negli ultimi anni si è cominciato a sentire parlare di “queste” famose Smart Home oppure, più generalmente, di domotica (domus: casa e ticos: applicazione)
Ma cosa sono le Smart Home? Come funzionano? Come si sono diffuse?
Queste sono le domande alle quali si cercherà di dare risposta, unite ad altri approfondimenti sul tema, tra cui i sistemi “personalizzati” o “commerciali”, per concludere delineando alcuni aspetti chiave sulla manutenzione, senza tralasciare alcuni esempi fondamentali che aiuteranno a comprendere a fondo in che modo le Smart Home riescono a migliorare la vita di tutti i giorni.

\subsection{La definizione di Smart Home}
La definizione di Smart Home non è così semplice da dare. La traduzione dall’inglese è letteralmente “casa intelligente” ed è effettivamente ciò che è, dato che si tratta dell’inserimento di determinati devices o elementi elettronici all’interno della propria abitazione. Quest’unione tra tecnologia e immobile ha il compito di migliorare la vita delle persone che ci vivono, correggendo alcuni errori o dimenticanze che i proprietari commettono inconsciamente, magari superflue per l’uomo, che portano ad una riduzione dei consumi, una maggiore comodità e sicurezza nella propria abitazione.
Per avere una casa intelligente si hanno principalmente due opzioni: la prima è quella della personalizzazione, ovvero si può decidere di far costruire la propria abitazione da zero con tutti i relativi elementi tecnologici già installati nella casa, senza devices aggiuntivi e con interfacce utente molto più “concrete” (esteticamente visibili). Ad esempio, in questo tipo di configurazione troviamo ancora i pulsanti al muro o comunque un pannello dei comandi. Secondariamente troviamo un’altra configurazione, quella un po’ più commerciale, ovvero l’acquisto di devices (alla fine una sorta di gadget) che comunicano direttamente con il proprio smartphone.
Le Smart Home aiutano la nostra quotidianità davvero in molteplici modi, tramite vari sensori, fotocellule e luci riescono ad aiutarci dove noi non riusciamo ad intervenire.
Per fare un esempio concreto, la casa riesce a controllare le varie temperature in tutte le stanze e riesce quindi a regolarsi secondo le nostre necessità. Potrebbe decidere di spegnere il riscaldamento in determinate stanze che non utilizziamo mai, oppure il sistema potrebbe addirittura riconoscere le persone che si trovano nella stanza ed adattare di conseguenza la temperatura e l’umidità a seconda delle nostre preferenze.
Invece, per quanto riguarda la sicurezza le Smart Home riescono ad essere molto più efficienti dei normali sistemi di sorveglianza.
Ad esempio, ci sono alcuni devices che, grazie ad alcuni sensori, riescono a rilevare la presenza di fumo, gas, incendi e riescono ad intervenire di conseguenza, ad esempio chiamando automaticamente o il proprietario o direttamente i singoli servizi di assistenza.
Sempre in ambiti di sicurezza, si trovano alcuni devices davvero tanto interessanti quanto semplici e utili. Basti pensare al campanello con videocamera e porta automatica, dove l’utente può vedere chi ha suonato al campanello anche da remoto e decidere se aprire la porta oppure no, anche direttamente dal posto di lavoro.
In generale in una Smart Home troviamo una continua interazione tra vari dispositivi e monitoraggio delle varie parti della casa, tramite videocamere e il controllo di quasi tutti i dispositivi elettronici smart, come ad esempio elementi della cucina, impianto stereo, televisori, tende e finestre. Il tutto controllabile anche con sistema di riconoscimento vocale.
Ciò ovviamente porta ad un vantaggio nella propria routine che non lascia indifferenti, dato che la casa è pienamente capace di svegliare il proprietario la mattina con la propria canzone preferita, aprendo dolcemente le tende, gestendo la luminosità e il colore delle luci secondo le preferenze del proprietario e accendendo magari il camino.
Sicuramente nel frattempo la macchina del caffè sta svolgendo il suo lavoro, cosicché il proprietario possa trovare una tazza di caffè caldo appena fuori dal letto. Successivamente il proprietario potrebbe farsi la doccia mentre il sistema racconta la sua agenda e le cose da svolgere durante la giornata.
Alla fine, l’unico limite è la propria immaginazione.
Oltre a tutti i vantaggi di comfort si ottengono enormi agevolazioni per quanto riguarda i costi, l’impatto ambientale e una notevole facilità di funzionamento. 
Infatti, i sistemi delle Smart Home tendono ad essere davvero estremamente semplici, utilizzabili persino da bambini o anziani. Le interfacce utente sono tendenzialmente chiare e di facile utilizzo, oltre ad essere molto affidabili e quindi durature nel tempo. Infatti, non bisogna dimenticare che gli anziani hanno bisogno di più attenzioni e assistenza permanente, cosa che le Smart Home fanno egregiamente. 
Ad esempio, se un uomo anziano dovesse cadere e non riuscisse più ad alzarsi, il sistema dovrebbe capirlo e chiamare automaticamente assistenza fisica per aiutare l’infortunato. Oppure se dovesse entrare un ladro in casa, il sistema dovrebbe automaticamente chiamare la Polizia e il proprietario, facendo vedere addirittura in che stanza si trova e di conseguenza anche cosa stesse rubando. 
Per quanto riguarda i costi ovviamente l’investimento iniziale può essere un po’ più incisivo di una normale abitazione con sistema tradizionale. Bisogna però dire che, basandoci su una visuale a lungo termine, il risparmio è garantito, 
Ad esempio, basti pensare ai risparmi sulla bolletta della luce dovute ad alcune dimenticanze che succedono nella vita quotidiana.
Inoltre, è imperativo dire che il valore di mercato delle Smart Home è nettamente maggiore a quello di alcune abitazioni con sistema tradizionale, dato che la tecnologia al giorno d’oggi è un vero e proprio valore aggiunto. 
Bisogna però dire che il proprietario deve già mettere in conto che prima o poi i sistemi andranno sostituiti oppure aggiornati. È dunque richiesto un piccolo costo per quelle occasioni in cui bisogna pensare alla manutenzione della propria abitazione.
Forse un possibile aspetto negativo è il fatto che è sempre richiesta una maggiore conoscenza e istruzione tra i comuni elettricisti, i quali sono costretti ad imparare un nuovo tipo di installazione (concetto trasformabile in opportunità per dei nuovi mercati). È anche sì vero che alcuni sistemi sono talmente avanzati che riescono ad aggiornarsi da soli, senza alcun aiuto di terzi, fanno anche delle scansioni regolari al fine di identificare alcune possibili anomalie del sistema. Secondariamente, un ulteriore problema nelle Smart Home è che semplicemente le persone non ne sono a conoscenza o comunque non sufficientemente informate in materia. Le persone tendono a pensare ancora alla casa autonoma come a quegli edifici super futuristici dei film di fantascienza, ma non sono seriamente coscienti di ciò che il mercato concretamente offre dei devices che migliorerebbero il comfort nella vita di molte persone.
Concettualmente secondo le possibilità del mercato odierne, è addirittura possibile far sì che la casa guadagni autonomamente, tramite dei pannelli solari oppure altre fonti di energia alternative. 
La casa può infatti produrre elettricità ed utilizzarla, però, quella che non usa, potrebbe automaticamente venderla allo stato e ricavarne un ricavo, che verrebbe direttamente versato sul conto corrente del proprietario. In conclusione, il proprietario avrebbe una vera e propria fonte di reddito autonoma, con grado di rischio nullo e con introiti (sebbene di vario valore a causa delle condizioni naturali).

\subsubsection{Le fasi della realizzazione di un sistema per case intelligenti}

Essendo una vera e propria lavorazione, sono stati elencati e spiegati alcuni elementi chiave da Domenico Trisciuoglio  nel suo libro “Introduzione alla Domotica” affinché vi sia una realizzazione del progetto domotico. Le fasi sono le seguenti:
\begin{itemize}
    \item{Analisi delle esigenze del cliente}
    \item{Decisione dei livelli di automazione}
    \item{Progetto di massima e quantificazione del budget (preventivo)}
    \item{Stesura del progetto definitivo}
    \item{Realizzazione dell’impianto}
    \item{Messa in servizio e verifiche}
    \item{Consegna al cliente}
\end{itemize}
Il primo elemento è l’analisi delle esigenze del cliente, fase in cui si decide cosa è più o meno necessario all’interno dell’abitazione. Vengono inoltre valutate le esigenze anche in base a cosa esattamente il cliente vuole rendere automatico e dove.
In secondo luogo, vi è la decisione dei livelli di automazione, la fase in cui si decide letteralmente “quanto” rendere automatiche determinate funzioni della casa. Questa fase della realizzazione di una SmartHome è delineata solamente da due cose: la fantasia e ovviamente il prezzo. Ci sono vari livelli di automazione, ad esempio c’è differenza tra il “tenere fissa la temperatura” e “adattare i vari flussi di corrente di aria calda o fredda e umidità a dipendenza delle preferenze della persona che entra nella stanza, ovviamente con riconoscimento facciale”. Ci sono di conseguenza anche prezzi diversi.
Successivamente, la terza fase è quella in cui si decide il progetto in linea di massima e si qualifica il budget. In modo particolare questa fase permette di stendere la versione di massima del progetto e di conseguenza avere immediatamente un feedback inerente i vari costi. In questa fase, a differenza delle altre, è possibile che, una volta sottoposto il tutto a un determinato cliente, quest’ultimo decida di ripartire dall’inizio poiché non soddisfatto del risultato. 
La quarta fase di realizzazione delle Smart Home vi è la stesura del progetto definitivo, momento in cui, dopo aver ricevuto il via libera del cliente si può procedere con le varie operazioni. Il passo immediatamente successivo è quello della realizzazione dell’impianto, ovvero la parte più interessante di tutto il processo, ma in realtà, data l’estrema facilità di applicazione dei dispositivi nelle abitazioni, questo è anche il passaggio concretamente più semplice. Questa fase a livello di difficoltà è direttamente proporzionale a come sono state fatte le precedenti fasi, nel senso che se le altre fasi sono state effettuate in modo efficace ed efficiente sarà possibile fare un’installazione dei devices del caso in modo molto semplice e veloce. Se nelle precedenti fasi invece sono stati omessi alcuni passaggi sarà di conseguenza più difficile effettuare le varie installazioni all’interno dell’abitazione.
L’ultima fase è la messa in servizio e la relativa verifica degli impianti. Questa fase delinea, in modo molto elementare, la consegna. Infatti, sarà possibile ridiscutere gli ultimi dettagli a lavori finiti, oppure scegliere tra alcune varianti di Sofwares nel caso in cui alcune funzioni implementate non dovessero essere di gradimento del cliente finale.

\subsection{Nascita e diffusione delle Smart Home}
Sicuramente non è possibile tracciare una vera e propria linea del tempo evolutiva dalla nascita delle Smart Home ad oggi, soprattutto perché non sono passati molti anni dal primo avvenimento delle Smart Home, essendo un prodotto attuale, soprattutto visto che i primi dati storici utilizzabili risalgono a non più di 100 anni fa.
È però possibile cercare di delineare alcuni momenti chiave nella storia per poi trarre una conclusione, o meglio, una linea temporale. Ovviamente si parlerà principalmente della terza rivoluzione industriale e, in alcuni termini, anche della quarta: ovvero due delle più grandi rivoluzioni degli ultimi secoli, periodi in cui vi è stata un vero e proprio progresso nel campo tecnologico o industriale. In ogni caso le Smart Home sarebbe stato possibile farle già molti anni fa a livello di tecnologie esistenti, però non è mai successo a causa della mancanza di una tecnologia abbastanza valida e consolidata che riesca a fare tutti questi automatismi contemporaneamente all’unisono. Esistevano già però determinate funzionalità che rendono in un certo senso “smart” la nostra abitazione, ad esempio il termostato che si alza automaticamente quando la sonda esterna capisce che è freddo.
In particolare, possiamo ricollegare la storia della domotica ad alcuni personaggi di spessore, come ad esempio William Penn Powers, il quale, per primo, creò una compagnia (antenata della Siemens) che inventò i climatizzatori automatici nel 1881. Grazie a quest’invenzione venne costruito il primo Hotel dotato di aria condizionata automatica sempre a Chicago nel 1907.
Durante tutto il ‘900 soprattutto negli USA vengono portate avanti alcune invenzioni che avrebbero poi portato la domotica a quello che è oggi. 
In particolare, nel 1950 circa, venne realizzato un dispositivo che permetteva il totale controllo di determinati impianti. Successivamente nel 1966, l’inventore Jim Sutherland creò il primo dispositivo automatico per la climatizzazione e anche di alcuni elementi elettrici. Negli anni ’70 infine, è stato sviluppato l’X10, il principale standard su cui si basa tutto il funzionamento delle Smart Home, anche odierne: concetto che verrà spiegato successivamente.
Sicuramente è possibile inoltre tracciare la storia della domotica grazie all’avvento dei sistemi informatici o computer, questi ha apportato un grosso cambiamento per quanto riguarda il mercato delle Smart Home, in quanto principalmente i devices avevano fatto un grande passo oltre, attraverso l’Internet delle cose, ovvero l’Internet of Things (IoT) che viene definito come un insieme di dispositivi che possono interconnettersi tra di loro tramite rete internet e senza l’intervento di terze parti. In pratica si tratta di una serie di tecnologie che permettono di collegare alla rete internet qualunque dispositivo.  In ogni caso il concetto verrà approfondito in seguito, insieme ai relativi concetti di sicurezza.
Nei primi anni del 2000 le Smart Home hanno cominciato a prendere piede e a diffondersi su tutto il globo, facendo comparire nel commercio le prime aziende e le prime soluzioni intelligenti, che si sarebbero evolute col tempo.
Oggi siamo in un periodo in cui i grandi brand cominciano a produrre elettrodomestici smart e c’è una vera e propria “corsa all’oro” per decidere chi è il numero uno sul mercato (dominato dai giganti Google, Apple e Amazon). 
Invece per quanto riguarda la diffusione della domotica a livello geografico, non è facile individuare un particolare schema di diffusione; anche perché essendo un “prodotto” molto innovativo e recente, come vediamo tutti i giorni dopotutto, si è diffuso dagli USA ed è successivamente arrivato in città come Londra o Parigi per poi infine diffondersi un po’ dappertutto.
Sul territorio italiano, questo mercato è in continua crescita; fino al 30\% ogni anno. In realtà la domotica è arrivata in Italia già negli anni ’70, però non ha mai preso veramente il via. Però oggi, finalmente, questo commercio ha preso piede e si sta diffondendo su tutto il globo.
Secondo alcune statistiche  effettuate in Italia, nel passaggio dal 2006 al 2008, le Smart Home sono quasi raddoppiate. La tabella si basa su impianti “base”, ovvero impianti piccoli e domestici e impianti “avanzati”, quindi impianti più grandi e quindi dedicati ad edifici di grandezza maggiore.
Riconducendosi alla situazione italiana, sono emerse alcune informazioni inerenti all’evoluzione delle Smart Home negli ultimi anni. In modo particolare, si sottolinea il fatto che, nonostante l’ovvia crescita esponenziale nel settore delle Smart Home, vi sono alcuni fattori che persistono e creano barriere nello sviluppo e nella diffusione, come ad esempio la bassa riconoscibilità di numerose marche che sono presenti sul mercato, ma che non riescono ad emergere. 
Per quanto riguarda il mercato invece, ci sono alcune informazioni indicative sul trand delle vendite dei prodotti “smart” per le case: i prodotti tendenzialmente più venduti in Italia sono quelli con applicazioni dell’Internet of Things per la sicurezza. Tra questi prodotti troviamo i più classici, cominciando con i semplici sistemi di sorveglianza fino ad arrivare alle serrature elettroniche e ai citofoni di nuova generazione in cui ai può vedere chi sta suonando al campanello.
Successivamente, i secondi prodotti più venduti sono tutti quelli che trattano l’ambito clima, come ad esempio i termostati interconnessi e automatici, oppure i termosifoni “smart”. 
Come terzo settore vi è quello incentrato sugli elettrodomestici, quindi qualsiasi elemento controllabile con un’applicazione, come ad esempio i celeberrimi “automatic-cooker” che riescono praticamente a preparare la cena da soli, con pochissimi accorgimenti. 
A livello internazionale gli smart devices più diffusi sono gli altoparlanti, come ad esempio tutti i sistemi multiroom intelligenti, che riescono a comunicare tra loro e riproducono i brani in modo simultaneo sfruttando la rete di casa. Ovviamente il tutto è opzionalmente controllabile con il comando vocale.
Come già detto in precedenza ci sono alcune barriere  o limiti che devono ancora essere superati. Le principali barriere sono tre:
\begin{itemize}
    \item{L’installazione dei prodotti}
    \item{L’integrazione dell’offerta con servizi di valore}
    \item{La presenza dei Brand affermati} 
\end{itemize}
La prima barriera è molto influente, perché nonostante i prodotti siano fatti per risolvere alcuni problemi, possono causarne altri, tra i quali la buona riuscita dell’installazione di tutti i devices. È vero che le case produttrici fanno il meglio che possono nel rendere comprensibile a tutti il funzionamento e soprattutto far si che chiunque acquisti dei devices riesca ovviamente anche ad installarli autonomamente. Tuttavia sorge il problema: tanta gente non ci riesce. 
Nonostante siano davvero dei sistemi elementari, tante persone a quanto pare non riescono ad effettuare un’installazione ottimale nelle proprie abitazioni. Allora queste persone sono costrette a chiedere aiuti a terzi dietro compenso: altri costi!
La seconda barriera è denominata “l’integrazione dell’offerta con servizi di valore”, ciò significa che le persone non ottengono un vero e proprio valore aggiunto con ciò che acquistano, infatti solo pochi sistemi di base offrono dei veri e propri servizi, come ad esempio l’allarme intrusione. 
La terza e ultima barriera è, come ho già detto, la mancata presenza sul mercato di Brand importanti e affermati. Il problema in tutto ciò è che il consumatore finale non si fida a sufficienza dei prodotti, poiché semplicemente non conosce la marca e reputa di conseguenza i prodotti obsoleti e/o inaffidabili, cosa ovviamente non vera.

\subsection{Funzionamento delle Smart Home}
Non è semplice spiegare il (vero e proprio) funzionamento delle Smart Home, dato che rimangono dei veri e propri sistemi informatici, suddivisi, come tutti i computer da una parte concreta e una astratta. Per definizione queste due “parti” vengono dette Software (astratta) e Hardware (concreta). Il Software è il sistema operativo delle Smart Home, il quale è tendenzialmente differente tra i vari brand, anche perché non è sempre presente un vero e proprio Software. Però l’Hardware è sempre presente, quest’ultimo è la parte fisica dei vari devices, ovvero tutto ciò che concretamente si vede. Uno dei sistemi elettronico integrati (personalizzato) su cui la domotica si basa è un sistema chiamato BUS, capace di controllare un insieme integrato di vari tipi di funzionalità, complesse o facili che siano. Il BUS, molto semplicemente, è un sistema doppio, ovvero che riesce a gestire sia l’alimentazione del dispositivo che lo scambio di dati tra tutti i vari devices. Ovviamente tutti questi devices comunicano tra loro tramite la rete di casa oppure una rete dedicata. 
Per quanto riguarda lo scambio di dati del BUS, i dati viaggiano sottoforma di bit codificati, ovvero il buon vecchio codice binario. 
Fisicamente, la persona che preme un pulsante da qualche parte come ad esempio per accendere la luce, trasmette una sorta di impulso contenente l’informazione di accendere la luce. Successivamente questa informazione viaggia verso un device attuatore che a sua volta invia un feedback con la relativa conferma della ricezione del comando.
Il sistema BUS riesce purtroppo a comunicare soltanto con un device per volta. Per evitare problemi o collisioni tra i vari devices ci sono determinati sistemi precauzionali di sicurezza che intervengono per gestire tutti i devices. Per fare un semplice esempio: se qualcuno volesse accendere la luce e contemporaneamente aumentare la luminosità non potrà farlo, dato che può effettuare solo un’azione per volta. Ciò che il sistema concretamente farà, sarà accendere la luce e mettere in coda il comando che abbassa la luminosità. Quindi una volta accesa la luce, il sistema effettuerà in un secondo momento l’azione di regolazione della luce. Il BUS ovviamente funziona soltanto per i sistemi integrati (personalizzati). Per quanto riguarda invece i sistemi “commerciali” che sarebbero i sistemi tradizionali, vi è una comunicazione completamente diversa, ovvero un continuo scambio di informazioni tra i vari dispositivi che sono interconnessi tra loro wireless. Questi dispositivi si potrebbe dire che sono dotati di intelligenza propria, dato che possiedono tutti delle piccole “menti” ovvero i processori che permettono al dispositivo di autogestirsi e comunicare direttamente con lo Smartphone del proprietario. 
Ovviamente ci sono altri sistemi di comunicazione tra devices. Il primo sistema in assoluto che venne inventato negli anni Settanta. L’X10 è tutt’oggi uno dei sistemi più utilizzati, esso venne inventato in Scozia e si basa sulla comunicazione tramite rete elettrica. In questo sistema tutti gli apparecchi sono, bene o male, ricevitori. Invece qualsiasi mezzo di controllo, come ad esempio i telecomandi, sono dei trasmettitori che emettono un segnale direttamente verso il device. 
Purtroppo, anche questo sistema ha dei limiti. Infatti, comunicare su reti elettriche non è la cosa migliore e soprattutto è poco affidabile. Per risolvere questi grossi problemi sono stati inventati nuovi sistemi di tutto rispetto, che utilizzano sistemi di comunicazione molto più innovativi, come comunicazioni via Bluetooth, WiFi o radio. I due principali sistemi principali sono ZigBee e Z-Wave. In realtà questi non sono veri e propri sistemi, ma sono protocolli, ovvero l’insieme di determinate regole che, se lette dai giusti sistemi informatici, vengono utilizzate come forma di linguaggio.
In pratica il protocollo è un vero e proprio linguaggio codificato che viene utilizzato tra più devices per comunicare tra loro, è come se fosse una vera e propria lingua. 
Con questo particolare tipo di sistema è possibile mandare il messaggio a destinazione in modi diversi. Ovviamente ciò ha permesso un ampliamento delle opzioni tra cui la possibilità di scegliere. 
Attualmente, il sistema oggettivamente migliore è quello che funziona con la rete wireless di casa, poiché offre una flessibilità più ampia per ogni singolo device. Esso diventa potenzialmente personalizzabile in ogni aspetto, dato che il completo controllo dei dispositivi proviene solo e unicamente dal proprio Smartphone. 
Vi sono numerosi altri sistemi  di comunicazione nelle Smart Home, alcuni indicati qui di seguito:
\begin{itemize}
    \item{Insteon}
    \item{CoCo}
    \item{Thread}
    \item{WeMo}
    \item{Nest}
    \item{Bacnet}
    \item{Dali}
    \item{Infrarosso}
\end{itemize}
Tutti questi sistemi hanno modi di funzionare differenti tra loro. Insteon ad esempio è un insieme di protocolli che comunicano sia tramite rete elettrica (Powerline) che via wireless. CoCo invece utilizza delle onde radio ed è una variante del sistema X10. Thread invece è un protocollo piuttosto nuovo che funziona wireless. È risultato molto molto promettente ed è stato realizzato da una collaborazione tra Samsung, Google e altre importanti aziende.
Per quanto riguarda WeMo, esso lavora tramite WiFi ed è stato creato da Belkin, è compatibile con tutti gli altri devices di altre case produttrici. Successivamente, Nest è uno dei sistemi più conosciuti, lavorando tramite WiFi. È uno dei sistemi migliori ed è completamente creato da Google, infatti appartiene al kit Google Home.
Bacnet è invece un protocollo comunicativo aperto e neutrale, che può quindi essere adattato a vari sistemi. Successivamente, Dali è uno standard per le interfacce digitali, come ad esempio le schermate che si vedono sui tablet di comando nei sistemi più avanzati. 
Infine, abbiamo il sistema di comunicazione ad infrarossi, uno dei sistemi più utilizzati nei telecomandi delle TV. Questo è un sistema che è diventato piuttosto obsoleto, dato che è unidirezionale, impreciso e a distanza limitata

\subsection{I devices utilizzati e alcuni esempi di “quotidianità”}
Nelle Smart Home ovviamente la cosa più importante sono i devices e soprattutto le funzionalità che ne derivano. Praticamente ogni stanza della casa è potenzialmente automatizzabile o comunque sfruttabile a livello domotico.
Ci sono alcuni esempi di Smart Homes nel mondo che sono semplicemente belli da togliere il fiato. Come già detto precedentemente alla fine l’unico limite è l’immaginazione. Spetta solamente al proprietario dell’abitazione decidere dove vuole intervenire o meno e soprattutto quanto vuole spendere.
Ci sono molteplici “settori” in cui si può intervenire, per essere più specifici sono una sorta di gestioni differenti tra loro. In ognuna di queste gestioni si situano alcune funzioni Smart anch’esse diverse tra loro. 
Le principali gestioni  sono quattro e sono elencate qui di seguito:
\begin{itemize}
    \item{Gestione ambientale}
    \item{Gestione e controllo carichi}
    \item{Gestione della sicurezza}
    \item{Gestione dell’informazione e della comunicazione}
\end{itemize}
La prima gestione è quella incentrata sul controllo dell’ambiente, infatti questa sezione comprende tutte quelle che sono ad esempio le gestioni climatiche. Come già detto in precedenza questo tipo di automazione si occupa principalmente di modificare a piacimento le temperature e l’umidità all’interno delle varie stanze della casa. In secondo luogo, si occupa del controllo dei consumi, i quali dovrebbero venire ridotti il più possibile. Questo tipo di funzionalità è molto vario a livello di sofisticatezza, ad esempio il sistema può semplicemente svolgere alcune funzioni molto basilari come impostare diverse temperature per le stanze, come può anche essere molto sofisticato tramite le funzioni che includono la regolazione della luce solare come metodo di riscaldamento, regolando di conseguenza le tapparelle, oppure spegnere il riscaldamento automaticamente quando non c’è nessuno in quella stanza o addirittura in tutta la casa. 
Sempre all’interno della gestione ambientale esiste un ventaglio di funzioni dell’illuminazione. Esse sono davvero molto variegate tra loro. Si parte dalla basilare regolazione dei Dimmer della luce (regolazione dell’intensità) e si arriva all’impostazione di determinati colori e luce naturale (o luce artificiale) all’interno delle varie parti della casa, il tutto prontamente scelto secondo i gusti dei vari ospiti. 
La seconda gestione invece è prettamente incentrata sulla gestione dei “carichi” ovvero tutti gli elettrodomestici. Questo particolare tipo di gestione è davvero completamente personalizzabile non solo a livello di sofisticatezza ma anche a livello numerico, perché è il proprietario a decidere quali elettrodomestici mettere in casa e di conseguenza quali automatizzare secondo le sue necessità. Ad esempio, non ha motivo di esistere una culla per bambini smart all’interno dell’abitazione di un soggetto che vive da solo. 
In questa particolare gestione si situano praticamente tutti gli apparecchi di cui conosciamo l’esistenza, però Smart. 
Ad esempio, ci sono i frigoriferi (come quello di Samsung) con un enorme schermo sulla parete esterna del frigorifero: questo schermo può venire utilizzato come lista della spesa che comunica direttamente con il proprio smartphone oppure lo si può utilizzare per vedere la propria agenda della giornata. Inoltre, esistono anche le lavatrici smart, che si regolano da sole e comunicano sempre con il proprio dispositivo cellulare. Successivamente esistono frullatori, forni, lavastoviglie e molti altri apparecchi che comunicano a distanza ciò che sta accadendo in tempo reale. Una delle particolarità di questo tipo di elettrodomestici è che hanno funzioni di auto-diagnosi. Giustifica che esse potranno individuare e tramettere da sole il danno alla casa madre e in alcuni casi addirittura autoripararsi. 
La terza gestione invece è totalmente focalizzata sulla sicurezza. Questo tipo di gestione è fondamentalmente una delle più importanti nella propria abitazione, dato che la sicurezza è forse il motivo per cui a casa quasi tutti si sentono bene. Ad aumentare questo senso di sicurezza anche qui la Smart Home  ha adottato delle soluzioni davvero molto intelligenti. La parola chiave è “videosorveglianza”, infatti questo concetto che è già presente da un po’ nelle case di molte persone, è andato a rinforzarsi grazie all’IoT e alla continua comunicazione tra il dispositivo del proprietario e ciò che i sensori e le videocamere vedono. 
Uno degli esempi più famosi è quello del videocitofono. Grazie a questo dispositivo è possibile vedere a distanza chi sta suonando al campanello in quel preciso momento ed è inoltre possibile comunicare direttamente con l’ospite che ha suonato il campanello direttamente dal proprio dispositivo mobile. 
Un altro esempio di sicurezza sono tutti i sensori interconnessi e che comunicano sempre con il proprietario o direttamente con i soccorsi. Alcuni tipi di sensori sono: quegli degli allagamenti, degli incendi e addirittura i sensori per delle eventuali fughe di gas. Un altro esempio è quello inerente alle persone anziane, le quali hanno ancora più difficoltà a muoversi o a compiere determinate azioni quotidiane. Nel caso in cui una persona anziana dovesse cadere in bagno, il sistema lo capirebbe e chiamerebbe immediatamente i soccorsi. 
La quarta e ultima gestione è basata sulla comunicazione. Si parla di comunicazione sia all’interno della casa che all’esterno. Anche questo tipo di gestione si basa sulle preferenze del proprietario ed esistono vari livelli di sofisticatezza anche in questo caso. Per essere più specifici, molte persone hanno i sistemi di diffusione sonora multiroom, però davvero pochi hanno un sistema che capisce i gusti musicali della persona che è appena entrata nella stanza. 
Alcuni esempi di gestione comunicativa domotica all’interno dell’abitazione sono ad esempio il tipo di televisore, dato che alcuni modelli nuovi di Samsung lavorano assieme al telefonino e riconoscono i gusti del proprietario. Sono inoltre presenti sul mercato sistemi che permettono l’interconnessione tra il dispositivo mobile e le videocamere all’interno delle varie stanze, quindi si può vedere cosa sta succedendo nella propria casa anche mentre non si è fisicamente all’interno di essa.
Per fare un esempio di uno dei sistemi domotici più famosi e meglio strutturati al mondo non può non risaltare la casa Gates, la dimora del celeberrimo fondatore di Microsoft: Bill Gates. 
Casa sua è uno dei migliori esempi di SmartHome che si possano trovare. Giusto per fare un semplice esempio, il suo sistema di diffusione sonora è a dir poco sensazionale, poiché dato che ogni persona della famiglia ha un Chip che lo riconosce, esso è capace di tracciare i movimenti di tutti i componenti che vivono all’interno della casa. Ma la parte più interessante è che questo chip fa si che una persona, man mano che si sposta all’interno della casa, riesca ad ascoltare i suoi brani preferiti. Ciò vuol dire che gli altoparlanti si attivano solamente se una persona gli si avvicina in qualche modo. Ovviamente risulta spontaneo chiedersi: ma se ci sono due persone nella stessa stanza? Il sistema è talmente avanzato che riuscirà a trovare un genere che risulta un compromesso tra i differenti gusti musicali. 

\subsubsection{Devices commerciali}
Nei capitoli precedenti è già stato accennato il fatto che esistono due principali concetti di SmartHome al momento sul commercio. Ovvero i sistemi integrati (centralizzati) e i sistemi tradizionali (distribuiti), qui denominati “Commerciali” per quanto riguarda quelli tradizionali e Personalizzati per quanto riguarda quelli integrati. 
Con devices commerciali si intendono tutti quei devices cosiddetti “aftermarket” ovvero che vengono acquistati dopo l’acquisto della casa, chiamati anche semplicemente “kit”. Questo particolare tipo di sistema è costituito da vari dispositivi prodotti da Brand famosi e che in un certo senso convertono i vecchi sistemi con quelli nuovi e Smart. 
A differenza dei devices personalizzati essi funzionano principalmente wireless e sono completamente modulabili, ovvero le persone possono decidere quanto mettere e dove in un secondo momento (dopo la costruzione dell’immobile). Questi sistemi comunicano direttamente con il dispositivo mobile del proprietario e devono essere sempre collegati al WiFi di casa. Questi dispositivi vengono anche detti “intelligenti” perché è come se fossero dotati di intelligenza propria, dato che si autoregolano e possono anche comunicare tra loro e ottenere la miglior soluzione possibile.
 
In commercio ci sono una miriade di opzioni, bisogna però capire che in questo caso, l’immaginazione limita molto di più, a causa del fatto che tutti i dispositivi si vedono e tendenzialmente non si effettuano veri e propri “lavori” per quanto riguarda l’istallazione dei dispositivi commerciali. 
Infatti, il concetto che sta alla base di tutti i dispositivi commerciali è il “plug and play”, quindi è voluto che per installare questi componenti non sia necessario alcun tipo di intervento strutturale, basta installare e usarli. 
Ad esempio, le lampadine Smart come le Hue di Philips non necessitano di alcuna modifica all’impianto elettrico: basta avvitare la lampadina, accendere la luce e il gioco è fatto. Questo avviene grazie alla compattezza del sistema. In pratica è la lampadina stessa che al suo interno ha un ricevitore e un Hardware personalizzato e connesso al WiFi, il quale riceve le informazioni dal cellulare. 
Spiegato a livello pratico: il segnale parte dal telefono, arriva al router del WiFi e viene immediatamente reindirizzato alla lampadina che svolgerà le funzioni precedentemente richieste. 
Tra i sistemi più famosi troviamo grandi Brand come Nest che al momento è al primo posto per quanto concerne il sistema di climatizzazione. Però a livello generico i colossi del mercato in questo momento sono due: Amazon Echo e Google Home. Entrambi i sistemi sono controllati tramite comando vocale e sono anche intelligenti, nel senso che riescono a rispondere a determinare domande che gli vengono poste. Un po’ come Siri sull’iPhone in pratica. Di seguito invece vi è la classifica dei migliori otto sistemi  per SmartHome del 2019:
\begin{itemize}
    \item{Amazon Echo}
    \item{Philips Hue}
    \item{TP-Link HS200}
    \item{Ecobee4}
    \item{NetGear Arlo Q}
    \item{Char-Broil Digital Electric Smoker with SmartChef Technology}
    \item{Perfect Bake Pro}
    \item{Ecovacs Deebot N79S}
\end{itemize}

\subsubsection{Devices personalizzati}
Se prima si parlava di sistemi Commerciali (o distribuiti) ora si parla di sistemi Personalizzati (integrati). 
Questo particolare sistema è stato il primo ad essere inventato, dato che utilizza un tipo di sistema diverso, anche per quanto riguarda la comunicazione.
Questo tipo di sistema è davvero completamente personalizzabile secondo ogni minimo desiderio del consumatore, dato che può venire installato solo quando viene costruita la casa. È possibile anche installarlo dopo, ma nessuno lo fa perché non conviene e soprattutto perché comporterebbe l’intervento di terzi per l’installazione. 
Non c’è nulla di semplice o “plug and play” nei sistemi integrati, perché è un sistema che viene installato come se fossero i cavi del telefono per intenderci. È un tipo di sistema che funziona con una centralina, la quale è collegata (tendenzialmente grazie alla power line) con tutti i vari dispositivi, ognuno dei quali deve venir preimpostato nel momento dell’installazione in una determinata parte della casa. 
Per intenderci, l’esempio di prima della lampadina in questo caso non va più bene, perché se il proprietario decidesse di rendere una luce Smart, sarebbe per forza necessario installare un nuovo sistema dedicato all’uso di quella specifica lampadina, infatti il sistema non sarebbe più nella lampadina stessa, ma sarebbe nel muro. 
Il sistema personalizzato però lascia davvero carta bianca al proprietario al momento della costruzione dell’immobile. Bisogna inoltre dire che essendo un sistema completamente installato da professionisti, rimane un impianto molto efficiente e soprattutto praticamente privo di malfunzionamenti. Nonostante ciò rimane comunque il fatto che è un sistema molto sofisticato e tecnico. 
Invece si sa quanto è stabile la rete wireless, potrebbe sempre crashare il sistema. Poi bisogna pensare che nei sistemi commerciali, ogni volta che si riavvia il router tutti i dispositivi smettono di funzionare fino a riavvio completo, cosa che con i sistemi integrati non succede.
In questo caso la concorrenza è meno agguerrita, ma sicuramente molto presente rispetto ai sistemi più commerciali. Tra le marche più famose ci sono AMX, Vimar e anche B-Ticino.

\subsubsection{Manutenzione}
Le Smart Home, come tutte le cose alla fine, necessità di un particolare tipo di manutenzione, volta a rendere efficienti ed efficaci i sistemi il più possibile. 
Come già detto, i sistemi informatici delle Smart Home sono un po’ come dei computer, infatti sono suddivisi come tali: una parte Hardware e una Software.
Per quanto riguarda il lato software, il sistema ovviamente necessita di aggiornamenti. Questi aggiornamenti possono venire effettuati automaticamente con i sistemi più avanzati, oppure devono venir aggiornati tramite terzi, come ad esempio degli elettricisti. Gli aggiornamenti vengono fatti su tutti i dispositivi e sono fondamentali per garantire il migliore funzionamento di tutto l’impianto e anche per poter avere la possibilità di implementare nuove funzionalità.
Invece, il lato hardware non necessita molta manutenzione, ad eccezione quelle volte in cui i dispositivi non si danneggiano o si rovinano. È inoltre possibile aggiornare i vecchi dispositivi con alcuni più moderni. 
In ogni caso non è necessaria una vera e propria manutenzione annuale, dato che comunque sono sistemi che segnalano da soli i casi di malfunzionamento, quindi quando il sistema ha una falla lo si saprà subito

\subsection{Considerazioni finali}
Per concludere il capitolo, risulta piuttosto ragionevole fare delle considerazioni finali su quanto detto finora. Dovrebbe essere chiaro ora cosa sono le Smart Home, come funzionano e come di sono diffuse. Inoltre, dovrebbe essere chiaro il fatto che esistono due principali tipi di sistemi: “personalizzati” o “commerciali”. Oltre a ciò sono stati spiegati alcuni esempi di applicazioni concrete di SmartHome, oltre al concetto di manutenzione. È stato inoltre possibile delineare alcuni processi chiave nella costituzione di una SmartHome.
Infine, ora dovrebbero essere acquisite tutte le possibili gestioni e funzioni dei devices che sono presenti nelle SmartHome.
Tutto considerato passare ai sistemi SmartHome, nonostante l’investimento iniziale più ingente, risulta, a lungo termine, piuttosto interessante e soprattutto utile nella vita quotidiana.

\newpage
\section{Analisi della sicurezza e della privacy delle Smart Home}
A cura di Rocco Apolloni
\subsection{Premessa}
La sicurezza è uno degli aspetti fondamentali di cui le Smart Home e la domotica si preoccupano . Essa va assicurata, e sia nelle case tradizionali che in quelle del futuro vi sarà una varietà e un numero sempre più grande di sistemi, impianti e prodotti che cercheranno di garantirla. Ciò nonostante le Smart Home sono diverse, poiché portano con loro nuove tecnologie e metodi che se da una parte cercano di rendere più sicura una casa dall’altra potrebbero allo stesso tempo creare la necessità di risolvere nuove possibili sfide e controversie legate alla sicurezza, e di conseguenza anche alla privacy e alla protezione dei dati personali, che sono al giorno d’oggi fondamentali per molti individui e aziende, così come per le Smart Home stesse e per il loro funzionamento.
A tal proposito in questo capitolo verranno analizzati questi elementi rispetto alle Smart Home, facendo anche un confronto con il concetto di “abitazione tradizionale”. Verranno fatti degli accertamenti su quel che concerne la legislazione in vigore, svizzera ma anche internazionale, attraverso il regolamento generale della protezione dei dati (GDPR), che risulta essere di primaria importanza soprattutto per quel che concerne i dati personali e il loro sviluppo in futuro. In questo modo potranno essere tratte delle conclusioni che diano delle risposte alle seguenti domande:”Sono effettivamente sicure le Smart Home? Possiamo garantire la nostra privacy senza intaccare l’efficienza delle Smart Home?”

\subsection{Concetti chiave}
Per rispondere a queste domande e approfondire quanto illustrato precedentemente è però necessario dare prima una definizione alle tre parole chiave di questo capitolo, ovvero: sicurezza, privacy e dati (i quali sono suddivisibili in più categorie). 
\subsubsection{La sicurezza}
Secondo quanto viene definito dall’enciclopedia Treccani la sicurezza ha diverse definizioni, delle quali la seguente riguarda particolarmente il tema di cui verrà trattato in questo capitolo:
“La condizione che rende e fa sentire di essere esente da pericoli, o che dà la possibilità di prevenire, eliminare o rendere meno gravi danni, rischi, difficoltà, evenienze spiacevoli, e simili.” (TRECCANI, Sicurezza, www.treccani.it (ultima consultazione: 23 novembre 2018) 
Ciò che traspare dalla definizione di questo termine è comunque piuttosto generico, poiché la sicurezza così intesa nella definizione soprastante non riguarda un ambito specifico ma si estende su i più disparati settori . Si pensi ad esempio alla sicurezza in un pub o una discoteca, dove oltre a delle norme specifiche solitamente degli uomini sono assunti e incaricati di mantenere l’ordine e la sicurezza di tutti i presenti; oppure alla sicurezza dei lavoratori stessi, la quale deve essere garantita dal datore di lavoro . Altri esempi possono anche essere la sicurezza alimentare o la sicurezza aerea. Questi elencati sono tutti diversi fra di loro e non riguardano in maniera diretta le Smart Home, ma rimangono pur sempre degli aspetti fondamentali della sicurezza e che permettono a tutti di poter vivere la propria quotidianità tranquillamente, il che è essenziale.
Quanto detto precedentemente dimostra che la sicurezza è ampliamente classificabile in diverse sezioni, delle quali solo alcune riguardano le Smart Home e ne garantiscono protezione e prontezza al pericolo rispetto a più attori e avvenimenti. Le principali branchie della sicurezza che il mondo della casa intelligente coinvolge sono: la sicurezza fisica, che comprende ad esempio quelle che possono essere fughe di gas, gli incendi e i furti , e secondariamente la sicurezza informatica (anche conosciuta come sicurezza logica) la quale, al contrario della prima, impedisce l’accesso ai “luoghi” digitali, alle informazioni e ai dati , i quali sono di assoluta rilevanza in un’abitazione smart. Perciò, nel corso del capitolo, verranno sviluppate parallelamente e confrontate, in quanto sono composti da strumenti e processi che li differenziano fra di loro, ma che collaborando in sinergia hanno uno scopo unico, ovvero quello di rendere la quotidianità domestica sicura e vivibile, in questo caso anche con un livello di garanzia ed efficienza maggiore rispetto a quanto può, solitamente, garantire una casa tradizionale. 

\subsubsection{La privacy}
Così come per la sicurezza anche la privacy è un qualcosa di estremamente variegato e che, soprattutto, è in continua evoluzione in quanto termine. Risulta anche essere molto personale, dato che ognuno può averne una percezione soggettiva e diversa rispetto ad altri individui . Basti pensare che nel 1890 venne già data una definizione di privacy all’interno del “Harvard Law Review – The Right to Privacy”, redatto dagli allora giuristi americani Samuel Warren e Louis Brandeis i quali, molto semplicemente, la descrissero come il diritto di essere lasciati soli . Questa è rimasta come una sorta di base che vale ancora oggi, ma che come detto innanzi non è un qualcosa di immutato nel tempo, in quanto la privacy cambia parallelamente con lo sviluppo sociale, culturale e tecnologico . Può essere quindi che la percezione di privacy di uno svizzero è diversa da quella di un giapponese, per via di culture e sviluppi sociali molto distanti fra di loro; questo si ricollega a quanto detto sulla soggettività della percezione della privacy. Un’altra caratteristica che privacy e sicurezza condividono è la multi-contestualità nella quale possono trovarsi. Oggi, la privacy, si situa in contesti che vanno dal giornalismo all’utilizzo degli smartphone, ma anche nell’informatica in generale, internet e sanità . 
Come introdotto precedentemente il significato di questo concetto si evolve e cambia anche di pari passo con l’evoluzione della tecnologia, ovvero l’aspetto che più conta, fra quelli citati, nell’ambito delle Smart Home. Difatti, secondo quanto emerge da Wikipedia e vari esperti del settore come l’avvocato Bruno Saetta, se un tempo la privacy si rifaceva alla tutela della vita privata oggi si è estesa fino al diritto alla protezione, alla salvaguardia e alla riservatezza dei propri dati personali . Il merito è attribuibile, principalmente, al progresso tecnologico. Ma non solo, va sottolineato che proprio a causa delle nuove tecnologie, le cui principali sono quelle legate alla comunicazione a distanza e all’archiviazione , la privacy è diventata sempre più un concetto sottile, poiché grazie ad esse le nostre informazioni possono essere facilmente ottenute (si pensi ad esempio alla geolocalizzazione).
È importante evidenziare il fatto che, nonostante lo stretto rapporto che esiste fra sicurezza e privacy, questi sono comunque diversi fra di loro e non va fatta confusione a riguardo. Un esempio piuttosto semplice che aiuta a comprendere meglio questo concetto può essere quello dell’e-commerce: effettuando un acquisto online tramite un protocollo, come ad esempio https, i dati inseriti dall’acquirente possono essere considerati al sicuro rispetto a terzi poiché avviene un passaggio diretto dal cliente al server fornitore, ma questo non garantisce la privacy di questi dati . Il rapporto che esiste fra i due termini è comunque fondato, tenendo conto del fatto che la difesa della privacy è presupposta, in una società libera, nel momento in cui è anch’essa sicura . È perciò importante sviluppare questi due concetti al meglio e in sintonia, non solo nella società ma anche nelle nostre abitazioni, in modo da poter soddisfare il nostro bisogno fondamentale, secondo quanto espresse Abraham Maslow attraverso la sua piramide dei bisogni, di sentirsi al sicuro .

\subsection{Internet of Things e sicurezza delle Smart Home}
\subsubsection{I dati}
Per quel che concerne la sicurezza e la privacy delle Smart Home i dati personali sono un aspetto fondamentale, che permette di fare una differenziazione degli aspetti di sicurezza e privacy rispetto alle case tradizionali, poiché rispetto a quest’ultime nelle abitazioni high tech le informazioni e i dati vengono scambiati e utilizzati dai devices interconnessi fra di loro in modo tale da sfruttare al meglio le funzionalità della Smart Home, ma questo verrà approfondito maggiormente successivamente nel paragrafo dedicato all’Internet of Things (IoT). Nell’ambito delle Smart Home i dati risultano essere centrali poiché possono essere valorizzati in diversi modi, ottenendo informazioni sui devices e sulle abitudini di chi ne usufruisce, anche a favore delle aziende ma non solo . Questo si ricollega a quanto detto precedentemente sul fatto che essi debbano essere tutelati a favore degli utenti, in quanto i dati possono essere soggetti a un’elaborazione e un utilizzo improprio, spesso senza che il detentore di questi ne sia a conoscenza. L’evoluzione della sicurezza controllabile da remoto ha quindi dato molte possibilità all’utenza in fatto di videosorveglianza e sicurezza a livello generale, ma secondo quanto emerge da un articolo pubblicato da la Repubblica ha anche automaticamente incrementato le possibilità di attacchi cyber e utilizzo improprio dei dati personali, dunque aspetti legati alla privacy . Non a caso, nonostante oggi le Smart Home siano una realtà in linea di massima piuttosto solida il quale mercato è in crescita (secondo quanto si evince dai dati di Swiss Life nel 2016 in Svizzera il mercato delle Smart Home ha generato un fatturato pari a 58 milioni di franchi e nel mondo di CHF 60 miliardi, ma si prevede un aumento fino a un valore complessivo di 500 miliardi di franchi entro il 2021)  molti consumatori sono ancora restii all’adozione di queste nuove tecnologie legate alle abitazioni, proprio a causa della sicurezza dei dati alla violazione di essi .
I dati personali sono definiti nel seguente modo dal Garante per la protezione dei dati personali in Italia: 
“Sono dati personali le informazioni che identificano o rendono identificabile, direttamente o indirettamente, una persona fisica e che possono fornire informazioni sulle sue caratteristiche, le sue abitudini, il suo stile di vita, le sue relazioni personali, il suo stato di salute, la sua situazione economica, ecc..”(GARANTE PER LA PROTEZIONE DEI DATI PERSONALI, Cosa intendiamo per dati personali?, www.garanteprivay.it, ultima consultazione : 23 novembre 2018) 
Come accennato in precedenza esistono diversi tipi di dati personali, che possono essere distinti in più categorie.
\begin{enumerate}
\item  \textbf{Dati identificativi}
\end{enumerate}
Vengono anche definiti “Personally Identifiable Information” (PII) , e sono in grado di fornire informazioni divisibili tra identificazione diretta e indiretta. Ad esempio possono essere informazioni come il nome, il cognome e in generale i dati anagrafici (identificazione diretta); d’altro canto possono essere considerate informazioni che permettono l’identificazione indiretta i vari numeri di identificazione come ad esempio il numero di targa o il codice fiscale.
\begin{enumerate}
\item  \textbf{Dati soggetti a trattamento speciale}
\end{enumerate}
Sono soggetti a trattamento speciale quei dati che la maggior parte di noi conoscono come dati sensibili, ma che di fatto oggi sarebbe meglio chiamare ex sensibili . Questi sono in linea di massima non trattabili, se non in alcuni casi specifici, come ad esempio nel caso in cui essi possono essere di interesse pubblico o nel caso vi sia un consenso da parte del proprietario dei dati in questione. Sono ad esempio dati sensibili quelli legati al credo religioso o filosofico, ma anche di natura genetica o etnici. 
\begin{enumerate}
\item  \textbf{Dati biometrici}
\end{enumerate}
Secondo quanto pubblicato su Agendadigitale.eu, portale d’informazione in merito allo sviluppo digitale, i dati biometrici sono:
Dati personali ottenuti da un trattamento tecnico specifico, relativi alle caratteristiche fisiche, fisiologiche o comportamentali di una persona fisica e che ne consentono o confermano l’identificazione univoca, quali l’immagine facciale o i dati dattiloscopici (AGENDA DIGITALE EU, Home > Cittadinanza Digitale > GDPR, che si intende per dati personali: natura, tipologie e qualità, www.agendadigitale.eu (ultima consultazione: 23 novembre 2018) 
\begin{enumerate}
\item  \textbf{Dati anonimi e dati pseudonimi}
\end{enumerate}
La differenza fra i due termini è che i primi non sono considerati come personali, a differenza dei secondi. Questo è dovuto dal fatto che per essere considerabile personale un dato deve anche poter identificare l’utente. Se i dati anonimi non sono in grado di risalire all’identità di chi ne è interessato i dati pseudonimi possono, ma in modo meno esplicito e diretto rispetto, ad esempio, ai dati identificativi. Questo è dovuto dal fatto che gli elementi identificativi vengono modificati o comunque codificati, come nel caso di un nickname al posto di un nome reale .
\begin{enumerate}
\item  \textbf{Dati relativi alle comunicazioni elettroniche}
\end{enumerate}
Vengono anche definiti con l’appellativo di “dati emersi” e consistono in tutta quella serie di dati personali che possono essere ottenuti attraverso degli apparecchi elettronici, come smartphones e strumenti legati alla geolocalizzazione; oltre all’utilizzo di internet stesso. Questi dati in particolare possono essere considerati fondamentali nell’ambito delle tematiche proposte, come si vedrà nel paragrafo seguente . 
\subsection{Internet of Things e sicurezza delle Smart Home}
\input{sicurezza_privacy/3.3.tex}
\subsubsection{Definizione dell'Internet of Things nell'ambito della sicurezza}
Una definizione in sé è già stata data nel primo capitolo, ma ci sono degli aspetti che meritano di essere approfonditi in quanto strettamente legati con i temi di questo capitolo.
Una sua particolarità è che grazie alla raccolta di dati e informazioni i devices collaborano fra di loro e interagiscono con l’utente con lo scopo di migliorarne la vita quotidiana e di soddisfarne i bisogni a corto e lungo termine. In merito ai temi del capitolo si tratta di mantenere la casa e chi ci vive al sicuro. Perciò l’evoluzione di internet e della rete non ha portato solo a connettere i computer, ma anche dei semplici oggetti che possono avere dimensioni molto piccole. Essi vengono anche definiti come “Smart objects”, e sono solitamente composti da diversi componenti, ovvero: microprocessore, memoria, capacità di archiviazione, modulo di comunicazione e una carica di energia . L’aspetto dell’IoT che più conta, in questo caso, è proprio la possibilità di poter collezionare dati. Un fattore, questo, che come si vedrà in seguito può creare delle problematiche a livello di privacy.
Le categorie di oggetti e cose che usufruiscono della rete possono essere molteplici, come ad esempio: dispositivi, apparecchiature, impianti, sistemi, materiali, prodotti tangibili, opere, beni, macchinari e attrezzature . Ovviamente per poter raccogliere e condividere dati devono avere specifiche caratteristiche e funzionalità che li rendono diversi da oggetti qualunque, come ad esempio identificazione (come un indirizzo IP), localizzazione, connessione, capacità di raccogliere dati e di comunicare con l’esterno  . Nell’ambito delle Smart Homes l’Internet of Things ha diversi scopi, ma principalmente da la possibilità agli utenti di poter controllare da remoto gli oggetti collegati e diversi ambienti della propria abitazione e i devices che ne fanno parte, compresi quelli legati alla sicurezza. 

\subsubsection{Prima dell' IoT}
Antecedentemente all’IoT si può parlare di una fase definibile come “pre-Internet of Things” , visto che esistevano già dei prodotti disponibili in grado di collezionare dati. Essi agivano però grazie alla sensoristica semplice, che era in grado di trasformare in dati digitali delle informazioni . La differenza sostanziale rispetto al concetto odierno di IoT sta nel fatto che, in un secondo momento, è arrivata la connessione in rete che permette ai dispositivi non solo di collezionare ma anche scambiare dati e informazioni .
È importante sottolineare il fatto che i cambiamenti non si sono fermati, ci sono state e ci saranno sempre delle evoluzioni e dei cambiamenti che permetteranno agli smart objects di essere sempre più precisi e efficienti nei loro compiti. È dunque necessario che la società moderna sia pronta e cosciente del fatto che questa è solo una fase di un processo fatto da continui cambiamenti ed evoluzioni, non prive di controversie, problematiche e sfide da affrontare (soprattutto in merito alla privacy) che ci costringeranno a cambiare nuovamente in futuro . 

\subsubsection{Casa tradizionle e Smart Home a confronto sul piano della Sicurezza}
Ci sono diversi oggetti legati alla sicurezza fisica che sono di rilievo ma soprattutto innovativi rispetto a quelli a cui siamo abituati in una casa tradizionale. Un esempio può essere quello della mia attuale casa, nella quale c’è un sistema di videosorveglianza composto da due telecamere riposte all’esterno che mirano all’entrata principale e all’entrata secondaria (finestra in giardino) oltre a dei sensori posti sopra tutti gli accessi alla casa dall’esterno (porte e finestre) i quali rilevano l’apertura e la chiusura di esse. Nel caso in cui, ad allarme inserito, venissero aperte scatta automaticamente una sirena molto potente. Il sistema in questione venne installato nel 2003, e infatti il controllo che si ha su di esso è minimo: non si tratta di un sistema moderno il quale permette di essere controllato anche da remoto come accade con i sistemi di sicurezza IoT. Al massimo è possibile vedere in diretta, attraverso la televisione, la visuale delle videocamere. Questo ovviamente non permette di tenere sotto controllo la casa nel momento in cui si è al di fuori di essa. Un altro punto negativo è che rispetto alla maggior parte delle soluzioni moderne i costi dei sistemi d’allarme dell’epoca erano molto elevati: all’incirca tutto insieme il nostro sistema di allarme è costato 10'000 CHF. Soprattutto la possibilità di scegliere era molto ridotta, mentre oggi ai consumatori sono offerte soluzioni che a livello di monitoraggio della casa sono valide ed economiche.
Come si può vedere nell’immagine soprastante, al giorno d’oggi molti prodotti vacillano su dei prezzi nettamente inferiori e accessibili, che possono variare dai 100 CHF fino ai 500 CHF. Ovviamente esistono anche fasce di prezzo più elevate, le quali principali categorie sono:
•	quelle fai-da-te (prezzo basso)
•	standard (prezzo medio)
•	domotica integrata (prezzo alto, comparabile ai prezzi dei sistemi nelle case tradizionali) . 
Un aspetto interessante e vantaggioso degli oggetti smart legati alla sicurezza è, come si vede nell’immagine, che molti di essi sono facilmente installabili e oltretutto non necessitano di un professionista. Inoltre un altro punto a favore è che i devices, anche se prodotti da aziende diverse, sono compatibili con altri. Va comunque tenuto di conto che non sempre è così, non tutti gli oggetti sono compatibili ed è bene, in quanto consumatore, esserne a conoscenza. Ad esempio il sistema “Nest Secure” è ottimo dal momento in cui, nella casa, vi sono già prodotti Nest; ma dal momento in cui si vorrebbe adattarlo a dei prodotti come “Amazon Alexa” questo non sarebbe possibile .
Prodotti piuttosto famosi come “Nest secure” o “Adobe Home Security Starter Kit”, i quali rientrano in quella fascia di prodotti in grado di offrire prestazioni comunque buone ma che tendono a costare meno (anche se in questo caso specifico il costo di Nest è elevato se comparato ad altri prodotti della stessa fascia), senza però tenere conto del fatto che sul lungo periodo potrebbero avere dei costi maggiori rispetto ai sistemi tradizionali. Un motivo potrebbe essere l’aggiunta di parti accessorie che comportano un costo anche se minimo, come accade nel caso di Nest . Un altro costo ulteriore è attribuibile principalmente alla necessità di coprire e proteggere i sistemi IoT dalle minacce legate al mondo dell’hacking. Come citato inizialmente la sicurezza informatica è proprio un qualcosa di opportuno dal momento in cui esistono sistemi IoT facilmente vulnerabili, poiché sono collegati grazie alla rete internet la quale può essere violata facilmente, soprattutto se non protetta. 
Questo risulta essere proprio il punto fondamentale che, rispetto alla sicurezza di una casa tradizionale, va creare dei dubbi proprio sotto l’aspetto della vulnerabilità e della sicurezza della privacy e dei dati di tutti coloro che usufruiscono di queste smart possibilità, un qualcosa che dunque mette in dubbio la sicurezza di una Smart Home proprio sotto questo aspetto e che potrebbe, come verrà approfondito in seguito, creare delle problematiche.
\begin{enumerate}
\item  quelle fai-da-te (prezzo basso)

\item  standard (prezzo medio)

\item  domotica integrata (prezzo alto, comparabile ai prezzi dei sistemi nelle case tradizionali)\footnote{\ DOGIARO,\ Quanto\ Costa\ la\ Domotica:\ Ecco\ i\ 3\ Livelli\ di\ Prezzi\ pi\textrm{\`{u}}\ DIFFUSI\ per\ la\ Smart\ Home,\ www.blog.dogiaro.com\ (ultima\ consultazione:\ 24\ gennaio\ 2019)}. 
\end{enumerate}
\subsubsection{Applicazione e problematiche dell’IoT nelle Smart Home nell’ambito della sicurezza}
Come anticipato, nell’ambito della sicurezza di una Smart Home ci si riferisce a due campi principali, ovvero quello della sicurezza fisica e della sicurezza informatica. La cosa interessante è che in pratica la sicurezza informatica, o logica, è un qualcosa la quale necessità nasce dal momento in cui la sicurezza fisica interagisce, per merito dell’IoT, con gli altri devices all’interno di una casa, esponendo dati e privacy a un potenziale rischio legato alla vulnerabilità degli oggetti smart che li gestiscono . Questo è il legame che associa un sistema di videosorveglianza, d’allarme o GPS alla necessità di mettere la propria Smart House in una condizione ottimale a livello di cyber security. Si può dunque parlare, secondo quanto espresso in un articolo di elettronica news, non solo di oggetti di sicurezza smart che possono interagire fra di loro ma di sicurezza smart degli oggetti stessi .
La presunta vulnerabilità citata innanzi può essere attribuita, secondo quanto emerge dal recente report in merito alla cyber security rilasciato da Swisscom Security nel maggio del 2018, proprio all’innovazione tecnologica la quale crea una convergenza fra ciò che è fisico e ciò che è virtuale . Una realtà, questa, che è estremamente attuale e veritiera in quelle che sono le Smart Home. Sostanzialmente è una catena che continua nel tempo, che gli hacker sfruttano a loro vantaggio usufruendo anche dell’inesperienza dei consumatori e creando delle possibili nuove minacce .
Come detto le minacce nascono di pari passo con l’evolversi della tecnologia. L’immagine soprastante è un radar, sempre estratto dal report di Swisscom Security, il quale mostra le diverse minacce, assegnandole a uno dei sette domini e, in base all’attualità e alla concretezza di esse sono state assegnate in uno dei quattro livelli: più una minaccia è verso il centro e più è attuale e concreta. 
Quanto emerge, nell’ambito dell’IoT, è che ci sono due diversi tipi di minacce categorizzati differentemente, ma che entrambi fanno suonare un campanello d’allarme, ovvero: IoT devices e IoT-based DDos. È opportuno dapprima spiegare cosa si intende con DDos e qual è la differenza con il primo. Si tratta dell’acronimo del termine inglese “Distribued denial-of-service” (DDoS), consiste nel rendere inutilizzabile e irraggiungibili, potenzialmente, qualsiasi dispositivo legato alla rete, sia computer che dispositivi IoT . Questo è reso possibile dai così detti “Botnet” (rete di bot), ovvero una serie di nodi della rete composti di oggetti compromessi in grado di sferrare attacchi in qualsiasi momento .
Secondo il radar Swisscom, come anticipato, entrambe le minacce fanno suonare un campanello d’allarme, poiché entrambe vengono collocate nei due cerchi più allarmanti. La più preoccupante riguarda gli IoT devices, collocati nel primo cerchio, denominato “temi principali”. Essi sono categorizzabili come attacchi cyber i quali colpiscono poi in seguito a livello fisico. Il pericolo rimane sempre quello: se non adeguatamente protetti gli oggetti legati all’IoT possono essere una minaccia per i dati e per la loro disponibilità in termini di prestazioni . Questo causerebbe, come detto, delle problematiche proprio a livello di sicurezza fisica. 
Per quel che riguarda invece IoT e DDos, questa minaccia viene assegnata al secondo livello di preoccupazione, quindi “allerta precoce”. Si tratta dunque del legame che c’è fra la diffusione dell’IoT e la possibilità di usarne i devices non protetti come “candidati di trasmissione” per Botnet. La categoria è quella della proliferazione, perciò un qualcosa in grado di diffondersi e moltiplicarsi .
La diffusione dell’IoT risulta dunque essere una minaccia, in quanto se non protetto adeguatamente può diventare una vera e propria arma informatica. 
Un esempio reale di quanto detto in merito alla vulnerabilità lo espone il portale online CorriereComunicazioni.it (CorCom), il quale scrive in merito a un report effettuato dall’azienda russa Kaspespery, specializzata in cyber security , il quale mette al corrente del numero sempre più crescente di attacchi rilevati verso oggetti connessi grazie all’IoT . Ciò che è stato evidenziato, grazie a dei sistemi appositi in grado di attirare gli hacker denominati “Honeypot” , è che rispetto al 2017 i numeri di attacchi malware (ovvero di software dannosi ) verso dispositivi IoT è già triplicato solamente nel primo semestre del 2018 . La cosa ancor più preoccupante è che molti di questi tentativi di attacchi, lo conferma anche Fastweb , sono avvenuti soprattutto attraverso router ma anche oggetti di uso comune come le lavatrici e altri dispositivi come ad esempio stampanti e dispositivi DVR, che rientrano nel ramo IoT .

\subsubsection{Utilizzo dei dati e obiettivi}
Gli obiettivi che portano gli hacker ad azioni di questo tipo sono per scopi illegali come spiare, ma soprattutto rubare dati e informazioni personali come ad esempio le password , che a loro volta possono anche essere a protezione di strumenti estremamente privati come i conti e-banking di una persona.
\subsection{Legislazione relativa alla sicurezza e alla privacy}
\subsubsection{Legislazione internazionale}
In questo punto verranno analizzate le regolamentazioni in vigore, principalmente europee, come ad esempio il già citato GDPR.
\subsubsection{Legislazione svizzera}
Lo stesso verrà svolto anche per la legislazione svizzera, e successivamente verranno confrontate cercando di tro vare delle possibili differenze.
\subsection{Considerazioni finali}
Considerando i dati raccolti e la legge verranno date delle conclusioni in merito alle domande di ricerca e, nel caso, verranno ipotizzati possibili soluzioni/strategie per il futuro
\newpage
\section{L'impatto ecologico delle Smart Home}
A cura di Chiara Calatti
\subsection{Premessa}
“La terra è ferita, serve una conversione ecologica”. Questa frase è stata inclusa nell’enciclica “Laudato sì” di Papa Francesco .
Partendo da questa considerazione, e sapendo che tutti i processi causati dall’uomo si ripercuotono sull’ambiente, con questo lavoro si analizzerà l’impatto ambientale delle Smart Home rispetto alle abitazioni tradizionali.
L’impatto ambientale è definito come “alterazione da un punto di vista qualitativo e quantitativo dell’ambiente, considerato come l’insieme delle risorse naturali e delle attività umane a esse collegate, conseguente a realizzazioni di rilevante entità .”
Pur sapendo che le Smart Home non nascono espressamente per risolvere problemi ecologici bensì per aumentare il comfort e vivere la quotidianità con meno stress e più tranquillità attraverso soluzioni personalizzate  , ci domandiamo se queste case intelligenti hanno anche un minor impatto ambientale rispetto alle abitazioni usuali. Partendo da questa domanda di ricerca si approfondirà se la domotica, nata per migliorare il comfort di vita, possa diventare uno strumento per ridurre in modo significativo il consumo di energia poiché questa riduzione è strettamente correlata con la preservazione dell’ambiente. La Smart Home, oltre che essere sotto un certo punto di vista più sicure, confortevoli, funzionali, flessibili e di semplice utilizzo, permettono infatti di avere un rendimento energetico più efficiente che punta sul risparmio di energia e quindi su una vita più sostenibile .
Essendo ben noto il fatto che il tenore di vita di un individuo è strettamente correlato con il fattore energetico, con questo lavoro ci concentreremo principalmente proprio su questo aspetto poiché il problema dell’elettricità è strettamente correlato alla tutela dell’ambiente in quanto la sua produzione non solo utilizza grandi quantitativi di risorse naturali che mettono così in pericolo l’ambiente in cui viviamo, ma emette ulteriori sostanze che lo influenzano negativamente .
4.2.	Il consumo energetico di un’abitazione 

\subsection{Il consumo energetico di un’abitazione tradizionale}
In questo capitolo si andrà ad analizzare l’utilizzo dell’energia elettrica in una casa tradizionale, perché il suo consumo consentirà di poter verificare se con le Smart Home questo consumo risulta minore. 
\subsection{Il ruolo dell’energia nelle abitazioni tradizionali}
L’energia elettrica può essere prodotta in vari modi. Nelle centrali termoelettriche si ottiene bruciando olio combustibile, gas o carbone, nelle centrali idroelettriche si ottiene utilizzando l’acqua. Grandi quantità di energia vengono invece prodotte dalle centrali nucleari che sfruttano l’energia racchiusa nel nucleo degli atomi .
La combustione dell’olio combustibile, del gas e del carbone ha un impatto negativo diretto sulla qualità dell’aria presente nell’atmosfera poiché il processo necessario per la sua realizzazione grava direttamente sull’ambiente, danneggiandolo a causa di emissioni di sostanze tossiche quali: CO2, zolfo e mercurio.
L’unità di misura dell’energia elettrica è il chilowattora (kWh) che corrisponde al lavoro compiuto da una macchina che sviluppi una potenza costante di 1 chilowatt (=100watt) per una durata di un’ora. È quindi il risultato della moltiplicazione della potenza di un tale apparecchio per il suo tempo di funzionamento . 
L’energia presente negli edifici in cui viviamo è prevalentemente l’energia elettrica. Essa è per definizione l’energia associata all’elettricità, in particolare l’energia di una corrente elettrica. Questo risultato deriva da processi di trasformazione di altri tipi di energia . 
La disponibilità di tali risorse è un fattore fondamentale nello sviluppo di un paese poiché la trasformazione dell’energia cinetica in energia elettrica permette energia luminosa, termina e meccanica indispensabile per il funzionamento di un’abitazione . 
Al giorno d’oggi nella società in cui viviamo è indispensabile usufruirne per poter avere una vita “normale” poiché essendoci abituati a tale comfort risulta difficile viverne senza.
Gran parte dell’energia che viene prodotta globalmente viene usata proprio all’interno delle case e viene impiegata per svolgere diverse funzioni quali: riscaldare l’ambiente e l’acqua, erogare elettricità e per mettere il funzionamento di tutti i dispositivi elettrici e gli elettrodomestici .
In Svizzera sarebbe praticamente impossibile vivere senza energia elettrica in casa. Prova ne è che quando per un guasto alla rete elettrica un’abitazione non può più ricevere la corrente gli occupanti sono “persi”. Buona parte delle attività della vita quotidiana, così come siamo abituati a viverle, diventano inattuabili. Tutte le attività non possono più essere svolte, dall’accendere la luce, al guardare la televisione, al cucinare. Ecco che il ruolo dell’energia nelle case tradizionali diventa essenziale.

\subsection{La politica energetica della Confederazione Svizzera}
La politica energetica della Confederazione Svizzera, come sottoscritto nell’articolo 89 della Costituzione federale della Confederazione Svizzera, con i Cantoni “si adopera per un approvvigionamento energetico sufficiente, diversificato, sicuro, economico ed ecologico, nonché per un consumo energetico parsimonioso e razionale. La Confederazione emana principi per l’utilizzazione delle energie rinnovabili sempre per un consumo enegetico parsimonioso e razionale .”
Nello specifico degli edifici, l’articolo 44 della Costituzione federale della Confederazione Svizzera dice: “Nell’ambito della loro legislazione, i Cantoni creano condizioni quadro volte a favorire l’impiego parsimonioso ed efficiente dell’energia nonché l’impiego di energie rinnovabili. Sostengono l’attuazione di standard di consumo per l’impego parsimonioso ed efficiente dell’energia. Al riguardo evitano ingiustificati ostacoli tecnici al commercio .” 
Questa legge permette agli abitanti di poter utilizzare determinati dispositivi nelle proprie abitazioni, che applicati all’intelligenza artificiale possono rendere Smart un edificio, in grado di risparmiare energia per poterla riutilizzare nel suo utilizzo stesso recuperando il calore residuo.
Il consumo d’energia degli apparecchi elettrici presenti nelle abitazioni deve essere classificato. Esistono infatti in Europa delle classi di consumo energetico che misurano propriamente l’efficienza energetica degli elettrodomestici ad uso casalingo.
Per valutare l’efficienta dei vari elettrodometisici utilizzati, è presente per legge dal 2011 un’etichetta energetica che giustifichi realmente questa efficienza informando i consumatori su elementi importanti riferiti al determinato apparecchio. Come esposto dall’Associazione settoriale Svizzera per gli Apparecchi elettrici per la Casa e l’Industria, per i venditori ed il commercio in genrale, questo strumento è unicamente oggetto di marketing che influneza la decisione   d’acquisto favorendo la vendita di determinati elettrodomenstici che, messi a confronto con altri che svolgono la stessa funzione, risultano meno efficienti e con un consumo maggiore. Ciò nonostante i produttori sono obbligati a metterla su tutti i prodotti che hanno intenzione di mettere sul commercio .
Quest’etichetta energetica viene inserita nel Certificato degli edifici (CECE) che, identifica la qualità energetica di un determinato edificio suddividendola in efficienza e fabbisogno sulla base di una scala di sette livelli. Esso mostra in aggiunta il potenziale miglioramento energetico di un involucro, fa in modo da poter individuare i suoi punti deboli, offre informazioni relative ad un possibile acquisto ed è unificato a livello svizzero . 
I sette livelli sono esposti attraverso lettere e colori. L’etichetta verde indica la classe con il minor consumo energetico rispettivamente contrassegnato con la lettera A+++ mentre quella rossa, lettera D, rappresenta il consumo maggiore. Le tipologie di apparecchi che devono essere così riconosciuti sono: frigoriferi e congelatori, lavapiatti, lavatrici, asciugatrici, apparecchi per l’illuminazione, televisori, forni e climatizzatori. 

\subsection{Il consumo energetico in Svizzera e nelle abitazioni}
Durante il 2010 la Svizzera ha generato complessivamente 66 TWh di energia elettrica. Per quanto riguarda questa produzione, hanno contribuito per il 57\% le centrali idroelettirche, per il 38\% quelle nucleari e per il 5\% le restanti centrali elettriche. Se viene consierato al netto, il consumo energetico di tale produzione è stato di 60 TWh. L’equivalente di 1TWh corrisponde ad 1 bilione di Wh e per arrivare ad 1 kWh occorrono infine 1000Wh. Il consumo energetico, tematica di grande attualità,  è in continuo aumento in Svizzera come in qualsiasi altra parte del mondo. Se volessimo rendere più chiara l’idea delle immense dimensioni di cui stiamo parlando, l’Ufficio federale dell’energia ha usato l’immagine di un lago di petrolio spiegando che il consumo energetico annuo in Svizzera avrebbe la superficie del Lago di Neuchâtel che corrisponde a 218km2 e una profondità di 70m . 
Negli ultimi 40 anni, questo consumo è raddoppiato e vedendo l’andamento generale non è prevista nessuna diminuzione bensì un continuo aumento.
In questo grafico si trova il consumo di energia svizzero per le categorie di utilizzazione dal 2000 al 2009. Si può vedere quali sono i consumatori che più ne hanno usufruito e si nota più chiaramente che le abitazioni, intese nell’immagine come economie domestiche, consumano il 29.8\% dell’energia complessiva superando dell’11\% le industrie .
Le abitazioni consumano con le aziende insustriali e artigianale ne utilizzano circa 1/3 a testa mentre il settore dei servizi 1/4.
Dal 2005, grazie ai molteplici avanzamenti tecnologici, il consumo energetico degli apparecchi domestici è in continua diminuzione anche se in miura molto contenuta rispetto al potenziale fornito dall’imponente miglioramente tecnico.
Per quanto rigurda la Svizzera, grazie ai dati fornitici dall’Ufficio federale dell’energia (UFE), si conosce che l’aliquota dell’energia elettrica totale che fluisce desplicitamente alle economie domestiche raggiunge il 31\%. 
Mediamente il consumo energetico annuale per una famiglia risulta pari a 5400kWh .
Nel 2017 il consumo energetico di tutte le abitazioni Svizzere è stato complessivamente di 3.428GWh. Sapendo che 1 gigawattora corrisponde esattamente a 1000000 chilowattora sappiamo che questo consumo è stato di 3428000 kWh.
In generale, nelle abitazioni, l’80\% dell’energia è necessario per riscaldare l’edificio stesso mentre il restante 20\% viene utilizzato per gli altri apparecchi elettrici e l’illuminazione. 
Con l’avvento delle nuove tecnologie l’orientamento è sempre più indirizzato su vettori energetici sempre più sostenibili.
L’elettricità risulta comunque un elemento indispensabile nelle nostre case. Questa deve però sottostare correttamente ad impianti sicuri in modo da evitare sprechi di energia ma anche per diminuire in maniera considerevole la possibilità che si sviluppi un incidente. Emerge quindi l’importanza di costruire degli impianti sempre più moderni e tecnologicamente avanzati come quelli presenti nelle Smart Home così che siano persino più efficienti. Un punto essenziale si trova nell’utilizzo dell’energia rinnovabile che oltre che avere dei costi sul lungo termine enormemente inferiori,  tutelano maggiormente l’ambiente. L’evoluzione tecnologica unita ad un impiego razionale ed intelligente degli elettrodomestici di casa, oltre che ad avere i benefici spiegati in precedenza, fanno in modo che ci sia una quantità maggiore di energia per tutta la popolazione.

\subsection{Il consumo energetico di una Smart Home}
In questo capitolo si analizzerà il metodo di razionalizzazione energetica di una Smart Home ed il suo impatto ecologico. Essendo un tema relativamente innovativo e non ancora così sviluppato, soprattutto in Svizzera, le informazioni reperite non sono ancora così cospicue. Malgrado ciò le informazioni a disposizione permettono comunque di poter affermare che il consumo energetico di una Smart Home è inferiore rispetto al consumo di una casa tradizionale e per quanto esposto in precedenza l’ottimizzazione del consumo energetico salvaguarda l’ambiente.
\subsubsection{Lo sfruttamento energetico}
Siamo a conoscenza del fatto che uomini e donne, per svolgere la vita quotidiana in un paese industrializzato, necessitano svariate risorse ma quella che gioca un ruolo fondmentale è l’energia. L’uomo necessita sempre più energia per svolgere le sue attività, difatti, da circa due secoli a questa parte il fabbisogno energetico è cresciuto enormemente. Questo bisogno eccessivo risulta appunto una delle problematiche più attuali. Lo sfruttamento energetico smisurato sta portando ad una scarsità delle fonti energetiche, che si stanno lentamente esaurendo. Il 90\% dell’energia primaria proviene dai combustibili fossili. Sono una risorsa formidabile che si sta di fatto esaurendo 
Questo aspetto ha creato una priorità nell’essere umano: riuscire da una parte a risparmiare energia e dall’altra a poterla riutilizzare. 
Oltre al fatto che le fonti energetiche possono essere rinnovate, si parla sempre più frequentemente di salvaguardare l’ambiente e ridurre gli sprechi. 
La scarsità di risorse naturali in contrasto con l’incremento demografico ha fatto prendere coscienza all’uomo della problematica. La società sta cercando di diminuire lo spreco di tali risorse e di implementare lo sviluppo e l’utilizzo di energie rinnovabili.
La domotica rappresenta realmente una vera chiave di volta per poter ridurre i consumi. L’obiettivo di questa nuova tecnologia è infatti quella di ottimizzare, in questo caso gli apparecchi domestici, in modo tale da consumare esclusivamente lo stretto necessario, senza portare smisurati sprechi di risorse naturali che, se dovessero esaurirsi, metterebbero in grave pericolo il futuro dell’intera società.

\subsubsection{Il ruolo dell’energia in una Smart Home}
L’energia è un elemento fondamentale quando si parla di case intelligenti. I vari accorgimenti legati ad un impiego più razionale delle fonti energetiche sono strettamente correlate con un minor impatto ecologico. Hans-Peter Burkhard, Direttore del Centro per una politica aziendale ed economia sostenibile (CCRS) dell’Università di Zurigo, afferma che “per edifici sostenibili si interndono le case energeticamente efficienti” .
È un elemento rilevante poiché è sempre tra i primi motivi, oltre  alla possibilità di avere la propria abitazione sotto controllo ed una maggiore comodità, che spinge una persona a costruire proprio una Smart Home per il suo risparmio energetico che porta di conseguenza, come citato dal direttore dell’Università di Zurigo, ad un efficacia ed un rendimento energetico notevole.
L’utilizzo dell’energia che troviamo in queste case domotiche si sta integrando sempre più con il concetto di intelligenza artificiale. Si sta infatti evolvendo grazie alle nuove tecnologie ed il suo percorso è legato al fatto che determinati dispositivi sono in grado di misurare ed ottimizzare l’ambiente in cui viviamo.
Il termine utilizzato per determinare quest’energia è digital energy . Essa è stata sottoposta ad un’analisi dall’Energy e Strategy group del Politecnico di Milano ed indica “l’uso di tecnologie digitali sempre più avanzate lungo la filiera dell’energia ”. È un concetto basato su due fattori principali: la distribuzione e la misurazione. Entrambi gli aspetti passano attraverso lo sfruttamente dei sistemi di hardware e software, sistemi in grado di elaborare dati, monitorare l’elaborazione di questi dati che riguardano i consumi energetici. Questi sono defiiti i big data. I big data sono come “un ingente insieme di dati digitali che possono essere rapidamente processati da banche dati centralizzate ”. Grazie alle tecnologie digitali presenti al giorno d’oggi è possibile gestire questi dati reali immediatamente ed in ogni luogo. Ci sono cinque principali motivi che spingono a favore dell’utilizzo della digital energy :
\begin{enumerate}
\item  Decentralizzazione: attraverso reti interconnesse si possono offrire sistemi energetici pi\`{u} flessibili e soggetti a meno perdite.

\item  Comportamento dei clienti: i clienti sempre pi\`{u} esigenti, vogliono essere in grado di poter monitorare i propri consumi.

\item  Rischi tecnici: si pu\`{o} ridurre la durata di eventuali interruzioni di corrente elettrica o altre problematiche relative a questo problema in quanto la gestione \`{e} molto pi\`{u} efficace e veloce

\item  \textit{Cyber security: }questa digitalizzazione richiede attente valutazioni ai rischi legati alla sicurezza in termini informatici, importante quindi il fatto che siano reti protette con le tecnologie pi\`{u} affidabili e recenti in termini di innovazione.\textit{}

\item \textit{ }Norme e incentivi: le politiche energetiche insistono sulla sostenibilit\`{a} di tipo ambientale; aumento della produzione dalle fonti rinnovabili e riduzione di emissioni di sostanze nocive.
\end{enumerate}
Attravero le Smart Home e le nuove opportunità offerte dal continuo sviluppo della tecnologia, del web e dell’Internet of Things tutti gli elettrodomestici presenti in un’abitazione sono collegati alla rete per una migliore gestione degli ambienti domestici. Questi oggetti Smart, sempre più presenti nelle abitazioni, possiedono una potenzialità di calcolo incredibile che in maniera completamente automatica si personalizza e diventa maggiormente efficiente a secondo delle esigenze e delle preferenze di ogni abitante della casa. Grazie a questo, è fondamentale il fatto che nelle Smart Home, il consumo di energia può essere misurato durante l’arco di tutta la giornata  . I dati che si ricavano da queste analisi sono utili per chi abita nella casa e permette loro di essere più consapevoli dei propri consumi e limitare nel caso il consumo stesso, specialmente se questo dovesse essere smisurato. Ancora meglio sarebbe se i consumatori fossero consapevoli dei danni ambientali che la produzione dell’energia che la popolazione utilizza per svolgere le attivita quotidiane provoca.
\subsubsection{La direttiva europea sull’efficienza energetica}
Uscendo dal contesto svizzero, poiché l’utilizzo della domotica nelle abitazioni non è ancora così implementato, verrà riportata questa norma sull’efficienza energetica a livello europeo. La Commissione europea ha infatti avviato una consultazione per una revisione sulla legge riguardante il concetto di efficienza energetica. Si vuole infatti prestare la massima attenzione alle politiche energetiche-climatiche in sinergia con quelle di rilancio economico. Questa norma europea UNI EN 15232 “Prestazione energetica degli edifici – Incidenza dell’automazione, della regolazione e della gestione tecnica degli edifici” evidenzia come l’introduzione nelle abitazioni sistemi di controllo e di automazione comporta la riduzione del consumo energetico .
Entro il 2020 infatti gli stati dell’UE si impegnano a rispettare la “strategia europea 2020” che si pone principalmente tre grandi obiettivi.
\begin{enumerate}
\item  Diminuire del 20\% le emissioni di gas serra

\item  Aumentare del 20\% l'efficienza energetica

\item  Aumentare del 20\% l'utilizzo dell'energie rinnovabili
\end{enumerate}
Questo progetto viene incentivato grazie all’incoraggiamento dell’efficienza energetica e l’uso razionale delle risorse energetiche, la promozione delle fonti di energia innovative e rinnovabili che incoraggiano la diversificazione energetica.
\subsubsection{Le energie rinnovabi}
Dagli anni 2000 è diventato fondamentale riuscire a risparmiare energia per poterla riutilizzare. La tecnologia in continuo sviluppo e l’utilizzo di internet in maniera sempre più prorompente ci fa intuire che l’intera industria sta cambiando in questa direzione. 
Stiamo infatti attraversando una fase che possiamo chiamare rivoluzione energetica. L’europa potrebbe infatti riuscire a rendere prioritarie le fonti rinnvabili e l’efficienza energetica in modo tale che  preventivamente entro il 2020 non si utilizzino più il nucleare e le fonti fossibili. Come spiegato in precedenza, visto che le riserve di combustibili fossili si stanno esaurendo lentamente, garantiscono un approvvigionamente per circa ancora 50 anni . Risulta utile essere a conoscenza delle fonti energetiche che sono rinnovabili e parlare dei loro potenziali di riutilizzo poiché risultano essere fondamentali per l’intero funzionamento di un’abitazione, soprattutto di una Smart Home . La Svizzera dispone di un vettore energetico molto importante poiché molto ricco. Se si guarda ciò che potrebbe accadere sul lungo termine, la Svizzera appare sens’altro un paese con un grande potenziale di sviluppo per quanto riguarda le fonti energetiche rinnovabili. Esistono in realtà grandi prospettive della produzione di energia elettrica .
Abitando in una Smart Home, questo processo di poter riutilizzare l’energia avviene in maniera automatizzata. Per esempio se si parla di un impianto fotovoltaico, ossia un impianto che grazie alla tecnologia fotovoltaica presente al suo interno permette di produrre l’energia trafonrmando le radiazioni solari in elettricità senza l’utilizzo di nessuno combustibile e quindi riducendo le emissioni di sostanze tossiche , si può considerare un nuovo modo, inseguito negli ultimi anni, per affrontare il problema dell’approvvigionamento energetico senza danneggiare l’ambiente. Questa energia può essere usata per riscaldare o conseguentemente raffreddare gli ambienti.
L’energia solare è considerata una delle tecnologie rinnovabili più pulite e soprattutto più sicure. È un impianto che consente ad un proprietario di poter approvvgionarsi da se ed essere quindi autosufficiente senza dover ricorrere a terzi. Oltre che avere agevolazioni fiscali per l’installazione, un aumento del valore dell’intero immobile e risparmio economico, il fatto di installare un impianto fotovoltaico nella propria abitazione permette una riduzione delle emissioni nocive come anidride carbonica nell’aria poiché l’intero funzionamento avviene tramite la luce naturale del sole.

\subsubsection{Il monitoraggio dei consumi energetici}
Il fattore principale è rivolto al risparmio sui consumi domestici grazie ad un monitoraggio costante degli apparecchi presenti nell’abitazione. 
Infatti all’interno delle Smart Home è presente un nuovo dispositivo che consente di monitorare i consumi domestici. L’efficienza energetica passa infatti attraverso un controllo dei consumi degli elettrodomestici casalinghi grazie ad una gestione che è integrata direttamente nei sistemi. Questo monitoraggio è garantito dallo sviluppo del NED, questo dispositivo è uno smart meter comunemente chiamato “contatore telegestito” . Sviluppato e creato dalla PMI Midori dell’Incubatore 13P del Politecnico di Torino.Attraverso dei sensori è in grado di tenere sempre sotto controllo ed informare costantemente i proprietari della Smart Home dei consumi. Esso presuppone una maggiore valutazione della proprio spesa in quanto il consumatore sarà più informato e consapevole di ciò che accade. Questo contatore è in grado di connettersi automaticamente con tutti i dispositivi di casa senza l’obbligo di dover installare numerosi, costosi ed invasivi strumenti di misura . Questo apparecchio quindi segnala i consumi eccessivi e le eventuali anomalie elettriche, per poterle controllare meglio. Una volta collegato al quadro elettrico, tramite lo smartphone, più precisamente con l’uso di un’applicazione, è possibile scoprire in tempo reale l’utilizzo dell’energia e imparare a risparmiare sulla bolletta fino al 20\% ogni anno. Si è in grado finalmente di fornire informazioni rilevanti riguardanti tutti gli elettrodomestici presenti nell’abitazione .

\subsection{La sostenibilità ambientale delle Smart Home}
In questo capitolo si tratterà la sostenibilità a livello ambientale che questo metodo innovativo di costruire le abitazioni ha, toccando dei concetti chiave utili per comprendere meglio l’argomento in questione e trarre delle conclusioni poiché il tema della sostenibilità è prioritario e fondamentale per tutta l’umanità.
\subsubsection{Green economy}
Le abitazioni intelligenti puntano a promuovere e sostenere questo modello di economia.  “La green economy, in italiano economia verde, è una forma di economia in cui gli investimenti mirano a ridurre le emissioni di carbonio e l’inquinamento, ad aumentare l’efficienza energetica e delle risorse, a evitare la perdita di biodiversità e conservare l’ecosistema ”. Si può dire quindi che è un concetto volto a migliorare il benessere umano e l’equità sociale riducendo significativamente i rischi ambientali e le scarsità ecologiche .
I temi trattati dall’economia e quindi dalla crescita sostenibile sono al giorno d’oggi argomenti di forte attualità poiché tutti i consumatori e cittadini sono sempre più attenti alla qualità dell’ambiente. Anche se possono sembrare due elementi che non hanno niente in comune, economia e rispetto per l’ambiente non devono assolutamente essere visti come antagonisti bensì come obiettivi da perseguire in ugual maniera nel senso che un’abitazione che risulta maggiormente efficace per l’ambiente potrebbe incentivare i clienti a preferirla al posto che un’abitazione tradizionale che non può ancora automatizzare il suo funzionamento e ridurre i consumi. Questa politica economica può quindi scaturire un riscontro positivo per i risultati economici delle imprese del settore edile. È un concetto molto sentito soprattutto dove sono presenti molte abitazioni poiché vi è un incremento esponenziale di inquinamento, consumo delle risorse energetiche e naturali .

\subsubsection{L’edilizia sostenibile e domotica}
Il concetto di sviluppo sostenibile è definito come “sviluppo in grado di garantire il soddisfacimento dei bisogni attuali senza compromettere la possibilità delle generazioni future di far fronte ai loro bisogni ”.
Il concetto di “edilizia sostenibile” è stato introdotto e sviluppato già a partire dal 1970 per rispondere alla crisi energetica e rispettare aspetti sociali molto presenti e significativi da alcuni decenni. Si tratta dei problemi ambientali che la Terra sta subendo a causa di svariati fenomeni. Questa preoccupazione ha dato infatti inizio ad un nuovo modo di concepire il settore edile, prestando sempre più attenzione agli aspetti ecologici. Pur essendo un settore in rapida e costante evoluzione, solo oggi dopo quasi 50 anni, questo concetto si sta diffondendo diventando una realtà quotidiana. 
Oggigiorno la maggior parte delle persone si sofferma sulla qualità degli edifici in termini di involucro ed impianti senza studiare la parte più importante ossia la gestione della struttura stessa. Elemento questo strettamente legato allo stile di vita degli occupanti dell’abitazione .
Una delle sfide più importanti da affrontare a livello globale è proprio la riduzione del consumo di energia infatti negli ultimi anni si sta andando sempre più nella giusta direzione poiché la maggior parte dei nuovi edifici è già in partenza dotato di un livello di intelligenza maggiore rispetto alle abitazioni progettate e costruite in precedenza. Da non dimenticare il fatto che una Smart Home può essere in qualche maniera introdotta attraverso vari dispositivi in una casa già esistente da anni. Questo è concepibile per merito della tecnologia e l’Internet of Things.
Contemporaneamente all’aumento del livello di sostenibilità ambientale vi è anche quella economica .
Paradossalmente sono considerati quindi gli edifici “intelligenti” quelli più compatibili con l’ambiente.
Infatti il fattore dell’automazione si sposa perfettamente con la sostenibilità poiché sfruttando l’energia solare per riscaldare gli ambienti e anche grazie al fatto che quest’intelligenza artificiale inserita nei vari dispositivi è in grado di elaborare continuamente e in tempo reale un’elevatissima quantità di dati, consente ai proprietari delle Smart Home di tenere sempre sotto controllo i consumi e rispettivamente i costi, evitando cosi sprechi eccessivi .
L’edilizia sta puntando ininterrottamente sulle Smart Home. Smart è di fatto la parola chiave che caratterizza ed identifica le abitazioni in cui si abiterà in un breve futuro. Questo termine ci ricollega a più fattori perché uno stile di vita più facile per merito di un comfort maggiore e l’aumento della sicurezza non sono abbastanza per poter definire una casa Smart poiché quest’ultima rivolge anche un particolare interesse all’ottimizzazione dei consumi energetici .

\subsection{Considerazioni finali}
In conclusione le Smart Home permettono una riduzione del consumo energetico con conseguente diminuzione dell’impatto ecologico. Malgrado l’impossibilità di reperire dati concreti e precisi sulla differenza numerica del risparmio energetico tra una casa tradizionale ed una Smart Home, possiamo affermare con certezza che grazie al suo funzionamento interno automatizzato, all’utilizzo di energie rinnovabili ed al costante monitoraggio, analisi e controllo dei consumi, le case intelligenti sotto efficienti ed ecologiche sotto tutti gli aspetti. Lo studio e lo sviluppo dell’intelligenza artificiale sarà sicuramente ampliato nei decenni a venire.  L’efficienza ed il risparmio energetico rappresentano un chiaro e sostanziale valore per la società. L’accuratezza e la progettazione di un design moderno, che si riallaccia anche alla sostenibilità ambientale, può quindi fruttare dei grandi progetti. In questo caso si parte sempre però dall’approccio bioclimatico che la costruzione di queste abitazioni può avere poiché si presta molta attenzione all’orientamento dell’abitazione rispetto al sole e alla ventilazione naturale, all’attenzione per la riduzione del fabbisogno energetico per riscaldare e raffreddare i locali in cui la gente abita, il ricorso alle fonti rinnovabili citate in precedenza come il solare fotovoltaico ed infine all’efficienza degli impianti che permette la riduzione dei consumi .
L’estrema attualità e crescita dei temi legati all’inquinamento atmosferico e dall’efficienza energetica, stanno sensibilizzando sempre una fetta maggiore di cittadini che, diventando più attenti e sostenibili prestano maggiore cautela nella costruzione delle proprie abitazioni creando costantemente maggiore interesse nei confronti delle soluzioni abitative capaci di rispettare l’ambiente e migliorare in maniera rilevante la qualità della vita. “Il futuro è Smart ”.

\newpage
\section{Domotica in Ticino e aziende specializzate nel settore}
A cura di Ariel Signorotti
\subsection{Premessa}
In questo quarto e ultimo capitolo sulle Smart Home verranno analizzati tre casi reali, due aziende e un cliente di esse. La prima ditta e il cliente sono residenti sul suolo ticinese, mentre la seconda azienda è austriaca ed è specializzata unicamente nella produzione di componenti per Smart Home ed è la più importante nel settore della domotica. L’idea è quella di andare a toccare i principali punti trattati nei capitoli precedenti e capire come sono applicati alla piccola realtà ticinese, la quale è spesso vista come inferiore dal punto di vista dello sviluppo tecnologico dal resto della Svizzera. Le due aziende serviranno a capire quanto è forte economicamente questo mercato e se ha un futuro in Ticino, mentre l’ultimo caso avrà lo scopo di capire quanto realmente è vantaggiosa una casa intelligente rispetto a un’abitazione tradizionale basandosi sui dati reali forniti da un cliente. La struttura dei capitoletti consisterà in una breve contestualizzazione generale in cui verranno fornite le informazioni generiche del caso, dopodiché verranno trattati i temi legati alla tecnologia, sicurezza e privacy, aspetti energetici e infine i pregi e i difetti. Il capitolo terminerà con la conclusione in cui verrà brevemente riassunto il contenuto del capitolo e verrà svolta una riflessione. Lo scopo è quello di capire quanto sono realmente evoluti questi sistemi ad oggi e come sono percepiti dalla società, cercando quindi di smentire o confermare le critiche in modo più oggettivo possibile, che sono all’ordine del giorno quando si parla di Smart Home. È importante notare che il capitolo della Singenia Sagl e quello dell’esempio di Smart Home in Ticino sono tratti da due interviste e sono di conseguenza soggetti alle opinioni e stili di vita delle persone intervistate.
\subsection{Singenia Sagl}

\subsubsection{Informazioni generali}
Singenia Sagl è una piccola società a garanzia limitata fondata il 24 novembre 2010 da Jörg Baas , tuttora dirigente. Dopo la scuola obbligatoria, il signor Baas ha proseguito la sua formazione alla Scuola Cantonale di Commercio di Bellinzona, dopodiché ha conseguito un titolo di master in economia presso l’università di Friborgo. Dopo la laurea ha lavorato per breve tempo in un’azienda e in seguito si è messo in proprio fondando la Singenia Sagl, con la speranza di introdursi in un mercato che ai tempi dell’apertura era agli inizi e che si sarebbe sviluppato in modo positivo. L’azienda è stata finanziata interamente con il capitale proprio del proprietario. Ad oggi la ditta conta due dipendenti (i dipendenti non devono possedere una formazione specifica, dato che l’installazione viene fatta da elettricisti esterni all’azienda, l’importante è avere buone competenze nel campo della pianificazione e progettazione edile e conoscere bene tutti i prodotti offerti).
La mansione principale della ditta consiste nella consulenza, pianificazione, programmazione e realizzazione di impianti intelligenti per case e appartamenti, offrendo così un’integrazione completa di sistemi domotici. Nello svolgimento di questo percorso è di fondamentale importanza la collaborazione con altre aziende o artigiani specializzati come ad esempio elettricisti, finestristi, idraulici, installatori di impianti di riscaldamento, eccetera. La sede della ditta si trova a Giubiasco e al suo interno è stato installato un impianto domotico con scopo dimostrativo, oltre che vari tipi di prodotti in esposizione. Per il momento la Singenia opera solamente sul territorio ticinese, anche se eccezionalmente svolge piccoli mandati nel canton Grigioni. La concorrenza esiste ma non è particolarmente invasiva per quanto riguarda la ripartizione del mercato, difatti sul suolo ticinese è presente solamente la Ticino Domotica con sede a Lugano . I clienti variano da privati a aziendali e al momento non sono particolarmente numerosi dato che si tratta di una tecnologia poco pubblicizzata nella regione, anche se è da considerare che pian piano è un settore in crescita.
Lo scopo dell’azienda è il seguente: 
“Il commercio, l'importazione e l'esportazione, la promozione, lo sviluppo software e hardware, la consulenza, la progettazione e la produzione, l'installazione, l'assistenza e la manutenzione di sistemi di domotica atti al miglioramento della qualità della vita delle persone, quali l'illuminotecnica, la sicurezza, l'intrattenimento multimediale, la climatizzazione, il risparmio energetico, l'automazione di immobili, la comunicazione .”

\subsubsection{Tecnologie installate}
\input{caso/5.2.2.tex}
\subsubsection{Percezione di sicurezza e privacy}
Per quanto riguarda le tecnologie offerte alla clientela, la Singenia si affida principalmente a tre aziende specializzate nella produzione di componenti per impianti domotici: Loxone, Crestron (azienda americana, del New Jersey specializzata nella produzione di svariati prodotti digitali come telecomandi, impianti stereo e impianti domotici, non molto presente nel mercato svizzero ) e Control4 (grossa azienda americana nata a Salt Lake City presente in più di 100 nazioni tra cui la Svizzera e grande produttrice di impianti intelligenti per case ed edifici ). Il fornitore più importante è il primo, azienda austriaca che verrà presentata nel dettaglio nel secondo capitoletto. Queste aziende forniscono tutti i componenti in parti separate (ad esempio centralina, valvole, sensori, pulsantiere, eccetera), in modo che i collaboratori della Singenia siano in grado di progettare e realizzare un prodotto personalizzato per ogni cliente.
I sistemi installati vanno a toccare i seguenti aspetti della casa: illuminazione (luci e tapparelle), sicurezza (serrature, sistemi d’allarme e sensori antiincendio, antiallagamento e sensori termici per le placche), intrattenimento multimediale (televisione e stereo) e climatizzazione (riscaldamento, aria condizionata e finestre). 
Nella maggior parte dei casi (circa il 95\%) il sistema domotico viene installato quando lo stabile è già costruito o addirittura in fase di ristrutturazione, l’altra opzione è che viene già integrato nell’edificio durante la fase di costruzione. Quest’ultima è la migliore perché permette di progettare l’impianto elettrico già in funzione dell’impianto che la Singenia installerà in seguito, in caso contrario è necessario intervenire in modo più accentuato e di conseguenza più costoso.

\subsubsection{Aspetti energetici}
Dal punto di vista del risparmio energetico si possono individuare molteplici vantaggi, sia per quanto riguarda l’elettricità che l’acqua. Gli impianti domotici offerti dalla Singenia sono in grado di ottimizzare al massimo il risparmio energetico agendo su diversi impianti domestici: direttamente sul riscaldamento e sull’apertura/chiusura delle finestre, adattando così la temperatura di ogni singola stanza in base al fabbisogno in quel momento (ad esempio durante le ore di sonno riscalda la camera da letto e abbassa la temperatura in tutte le altre stanze, quando la casa è vuota abbassa la temperatura generale; ovviamente questi protocolli di tipo routinario si adeguano a eventuali cambiamenti grazie ai sensori che percepiscono la presenza di soggetti); può agire sull’impianto elettrico controllando così l’illuminazione sia interna che esterna utilizzando le tapparelle o le luci, evitando di sprecare elettricità laddove non è necessario. 
È molto interessante notare che l’efficienza massima di un sistema domotico si ottiene con l’abbinamento con i pannelli fotovoltaici, questo perché la Smart Home è in grado di capire quando la casa sta producendo elettricità oppure quanto la sta acquistando dalla rete di riferimento. Se ad esempio ho un’auto elettrica e voglio che sia carica per le 19 di sera in modo tale da poter uscire a cena, la casa gestirà la ricarica nel modo più intelligente ed efficiente possibile (anche facendo riferimento alle previsioni meteorologiche) ricaricandola durante le ore in cui il sole, nel limite del possibile, è presente per poter sfruttare l’impianto fotovoltaico. Sempre tenendo l’esempio dell’auto elettrica, l’impianto è in grado di caricare il veicolo durante le ore in cui l’elettricità costa meno, come ad esempio di notte (dando però sempre e comunque la precedenza all’impianto fotovoltaico). Lo stesso discorso vale per tutti gli elettrodomestici e dispositivi elettronici. Un altro importante risparmio di elettricità avviene grazie al controllo degli apparecchi in stand-by, dato che spesso e volentieri rimane una spia led accesa. Per risolvere questo problema, tutte le volte che gli occupanti dell’abitazione escono oppure vanno a dormire la casa provvede a spegnere tutto ciò che è rimasto acceso e che non serve in quel momento. Tutti questi aspetti, pensati per il risparmio energetico, sono regolabili e gestibili dall’utente; è importante sottolineare che l’impianto domotico metterà sempre il confort come priorità.
I risparmi monetari rispetto a un’abitazione tradizionale sono notevoli, tuttavia non è possibile stimare una cifra precisa perché è strettamente collegato alle preferenze di funzionamento impostate dal proprietario (ad esempio un utente che vuole una temperatura di 28 gradi in piscina tutto il giorno consumerà più risorse energetiche rispetto a uno che si accontenta di 25 gradi e la usa una volta al giorno) e dalla diversità delle abitazioni (ad esempio con o senza piscina).

\subsubsection{Pregi e difetti}
Il primo pregio che si apprezza vivendo in una Smart Home è sicuramente il comfort, seguito dal risparmio energetico e dalla sicurezza di cui si può beneficiare, temi trattati nei capitoletti precedenti. Al momento non sono previsti sussidi per chi costruisce una casa intelligente, ne dai cantoni, ne dalla confederazione (e secondo il parere del signor Baas non ci saranno, o almeno non nel futuro più prossimo, questo per due motivi: perché al giorno d’oggi chi costruisce una Smart Home lo fa più che altro per il comfort offerto da essa e in secondo luogo perché si tratta di una tecnologia relativamente nuova e non molto discussa come invece lo sono i pannelli fotovoltaici ); tuttavia spesso avendo un impianto domotico installato si raggiungono anche gli standard “Minergie”. Ottenere questa certificazione da invece accesso a sussidi e agevolazioni varie a dipendenza del luogo. I requisiti minimi per ottenere questo certificato sono i seguenti :
\begin{enumerate}
\item  Requisito principale: indice Minergie (nuove costruzioni: 55 kWh/(m2*a))

\item  Requisito supplementare sul fabbisogno termico per il riscaldamento per edifici nuovi (involucro edilizio): identico al MoPEC 2014

\item  Requisito supplementare sull'indice di energia termica finale senza PV: 35 kWh/(m2*a) per nuove costruzioni, 60 kWh/(m2*a) per ammodernamenti

\item  Produzione propria di elettricità almeno come richiesto dal MoPEC 2014 (10 W/m2 SRE)

\item  Ricambio dell'aria controllato e protezione termica estiva

\item  Nuove costruzioni senza combustibili fossili 

\item  Necessario un concetto di tenuta all'aria, ma nessuna misurazione obbligatoria

\item  Richiesto il monitoraggio dell'energia per edifici di superficie superiore a 2'000 m2 SRE 

\item  Semplici misure strutturali per l'idoneità alla mobilità elettrica degli edifici Minergie
\end{enumerate}
Dato che non ci sono problemi legati alla privacy dei dati grazie al sistema a “scatola chiusa”, al momento l’unico difetto che si può riscontrare dai servizi offerti dalla Singenia è il prezzo. Anche in questo caso è molto difficile stimare un prezzo medio perché il costo dell’impianto dipende da molteplici fattori: grandezza della casa, predisposizione del sistema elettrico, scelta degli accessori e delle funzioni (si possono personalizzare affinché rispecchino perfettamente i bisogni del cliente), vari servizi online (come la meteo), eccetera.
Il prezzo indicativo di un impianto mediamente completo per una casa unifamiliare si aggira intorno ai CHF 7’000-8'000, costo degli elettricisti escluso. Come spiegava il signor Baas durante l’intervista, i prodotti da loro offerti sono comparabili alla Volkswagen per le automobili: hanno un’ottima qualità e un buon livello tecnologico a un prezzo ragionevole. Ci sono poi altri prodotti molto più lussuosi e di conseguenza anche più costosi che però sono venduti da altre ditte specializzate .


\subsection{Loxone}

\subsubsection{Informazioni generali}
Loxone Electronics GmbH, come già accennato nel capitoletto precedente, è un’azienda austriaca specializzata nella progettazione e produzione di tutti i componenti necessari alla costruzione di un impianto domotico di ultima generazione. È un’azienda di grandi dimensioni fondata da due giovani imprenditori austriaci, Thomas Moser e Martin Öller. L’apertura risale al 2008 e ad oggi Loxone rifornisce molte nazioni in tutto il mondo, la maggior parte di queste possiedono anche diversi punti di distribuzione. L’idea di aprire l’azienda è nata dopo aver avuto l’intuizione di rendere le abitazioni completamente automatizzate ispirandosi ai robot tagliaerba e alle auto con guida autonoma. In realtà in parte questa possibilità esisteva già sul mercato, tuttavia le apparecchiature erano poco intuitive da utilizzare e soprattutto avevano un prezzo praticamente inaccessibile per la maggior parte delle persone. Si può quindi affermare che la fondazione di questa grossa ditta sia nata da una necessità personale, come spiega Martin Öller:
“Pensavamo che dovesse essere possibile creare un sistema semplice, intelligente e allo stesso tempo abbordabile dal punto di vista dei costi! Volevamo un sistema che di notte accendesse la luce in camera da letto, se hai la necessità di uscire, senza dover cercare l’interruttore. Volevamo un sistema che facesse sì che le tapparelle e le lamelle delle veneziane seguissero il corso del sole e che il riscaldamento riducesse la potenza quando la temperatura è sufficientemente alta…” (Martin Öller Fondatore e CEO Loxone, www.loxone.com).
È così nato il concetto che ha portato la Loxone al successo, la “Loxone Smart Home”, che inizialmente era una soluzione applicata alle case dei due fondatori, in seguito dopo essere stata rivista e perfezionata ha fatto il suo esordio sul mercato austriaco riscontrando un buon successo grazie alle seguenti caratteristiche spiegate da Thomas Moser:
“Loxone Smart Home non aumenta solo il comfort, l’efficienza energetica e la sicurezza. È un sistema previdente: ad esempio accende il riscaldamento per tempo per trovare la casa calda e confortevole al rientro. Senza che io debba intervenire. E se è necessario, è addirittura possibile controllare da lontano se il ferro da stiro è spento. Senza intaccare la privacy ovviamente.” (Thomas Moser, Fondatore e CEO Loxone, www.loxone.com).
Al momento l’azienda è la leader nel proprio settore e occupa all’incirca 250 dipendenti sparsi per il mondo, oltre che i rivenditori autorizzati (come ad esempio la Singenia Sagl, trattata nel capitoletto precedente). La loro strategia è quella di puntare sulla qualità del prodotto sempre mantenendo un prezzo accessibile alla maggior parte delle persone, oltre che puntare molto sulla sicurezza e sulla privacy (temi molto delicati quanto di parla di domotica). I principali clienti appartengono a diverse categorie: proprietari di case, di appartamenti, di hotel o aziende; tuttavia c’è un’idea che li accomuna: il fatto di vivere l’esperienza abitativa in modo moderno e confortevole. Un aspetto molto interessante da notare è che i prodotti sono fatti su misura per ogni cliente, finora sono stati implementati migliaia di progetti per molti clienti sparsi in tutto il mondo. La missione della Loxone è che tutti vivano in una Smart Home. 

\subsubsection{Tecnologie prodotte}
Come accennato in precedenza Loxone produce impianti domotici. Questi ultimi vengono venduti separatamente in modo da poter progettare la Smart Home fatta su misura per il cliente. Per poter fare ciò, al momento, l’azienda ha sul mercato all’incirca 150 prodotti sia software che hardware perfettamente coordinabili tra di loro. Qualche esempio di prodotto sono: sensori (di luminosità, umidità, temperatura, fumo, acqua, vento, …), software (sia per il funzionamento del prodotto che le varie applicazioni per Smartphone e altri dispositivi per la gestione del sistema domotico), impianti stereo, pulsantiere, centraline, orologi, piastre per cucine, luci, citofoni, eccetera.
Il cervello della Loxone Smart Home è il Loxone Miniserver, ovvero la centralina che riceve tutte le informazioni dai sensori e dai tasti e le elabora per poi decidere cosa fare, anche in modo autonomo in base alle abitudini degli abitanti della casa. È interessante notare che si tratta di un sistema centralizzato, quindi tutti i cablaggi passano dal server, permettendo così di avere un sistema coerente e lineare nel funzionamento. Questo piccolo server è in grado di lavorare senza l’ausilio di internet, al contrario della stragrande maggioranza dei prodotti simili sul mercato. Questa caratteristica permette di avere un livello di sicurezza dal punto di vista della protezione dei dati molto alto, questo tema era già stato parzialmente trattato nel capitolo della Singenia Sagl e verrà discusso in modo più approfondito nel prossimo capitoletto. 
Un altro prodotto molto interessante e innovativo, anch’esso progettato e sviluppato dalla marca austriaca, è la pulsantiera multifunzionale Loxone Touch. La novità di questo prodotto è che possiede cinque zone tattili dove sono presenti le varie funzioni, a loro volta differenziate a dipendenza del numero di volte che viene premuta la zona. L’idea è quella di avere una di queste pulsantiere in ogni stanza e che funzionano con la stessa logica, così da poter essere utilizzate per gestire tutto l’impianto domotico (illuminazione, riscaldamento, areazione, impianto d’intrattenimento, eccetera) in modo intuitivo. Le funzioni delle varie zone sono programmabili e personalizzabili dall’utente in ogni momento tramite l’applicazione, dalla quale è anche possibile interagire con la casa come con la pulsantiera. Un’altra funzionalità molto interessante è quella dello scenario, ovvero la possibilità di programmare un insieme di parametri in base a una determinata situazione. Un esempio potrebbe essere lo scenario “Relax”, dove si potrebbe rendere l’illuminazione più calda, le tapparelle alzate e musica rilassante. Lo scopo di queste pulsantiere intelligenti è quello di non dover utilizzare il telefono in continuazione per usufruire dei servizi della casa, cosa che invece accade negli altri sistemi domotici. La Loxone definisce questi ultimi come 2.0, mentre la Loxone Smart Home è considerata 3.0 proprio per questo motivo che garantisce un’esperienza abitativa decisamente più confortevole e tecnologica. 

\subsubsection{Sicurezza e privacy}
La Loxone Smart Home garantisce al cliente dei livelli di sicurezza e privacy molto elevati, soprattutto considerando il confronto con i prodotti della concorrenza che invece puntano maggiormente sulla connettività e altri servizi collegati al mondo del Web. Rinunciando a queste caratteristiche Loxone è comunque riuscita a diventare leader del mercato e anzi, a farsi apprezzare proprio per questo motivo. Affinché il sistema detto “a scatola chiusa” riesca a funzionare senza il collegamento alla rete internet è necessaria un’istallazione di un software molto complesso nel server e che sia in grado di elaborare in modo intelligente i dati che riceve. Il collegamento alla rete è necessario solamente per usufruire di alcuni servizi (ad esempio per il controllo dello stato della casa e l’azionamento di alcune funzioni) e per fare gli aggiornamenti del software. Tra i servizi precedentemente citati non sono compresi ad esempio l’apertura delle serrature e delle tapparelle. Il sistema è comunque molto difficile da attaccare perché la rete della casa dev’essere protetta da una password e tutto il sistema gode di una criptazione molto complesso. 
\subsubsection{Aspetti energetici}
Dato che i prodotti della Loxone sono impiegati dalla Singenia Sagl per installare impianti domotici ai clienti, per quanto riguarda gli aspetti energetici si può fare riferimento al capitolo dedicato a questo tema della Singenia.  Nel capitolo terzo e ultimo capitolo invece vengono citati dei numeri riguardo un caso sul suolo ticinese. Per quanto riguarda l’impegno per l’ambiente da parte dell’azienda stessa non ci sono informazioni specifiche, se non che tutti gli stabilimenti sono dotati di un sistema domotico prodotto dalla Loxone che ne garantisce il risparmio energetico e quindi anche la diminuzione dell’impatto ecologico.
\subsubsection{Pregi e difetti}
Oltre a comfort e sicurezza, i prodotti Loxone offrono un’ampia scelta di materiali e colori per ogni prodotto in modo da abbinarsi perfettamente agli stili delle abitazioni in cui verranno istallati. Un altro aspetto molto interessante è che l’azienda ha costruito delle case che fungono da showroom, ovvero case che integrano l’impianto domotico al loro interno e che il cliente può visitare prima di acquistare il prodotto per comprenderne al meglio il funzionamento anche attraverso la consulenza offerta dai collaboratori presenti. Questo progetto prende il nome di “The Loxone Experience Tour” e l’abitazione più grande si trova a Vienna. Secondo Loxone la Loxone Smart Home è la prima casa 3.0, ovvero che è in grado di prendere decisioni da sola e non più solamente attraverso la gestione via Smartphone dell’utente. Grazie a queste tecnologie la Loxone stima che in media ogni persona risparmia all’incirca 61'570 azioni all’anno (come ad esempio accendere la luce o aprire la porta quando arriva un ospite), ciò fa risparmiare all’utente molto tempo che può invece essere utilizzato per fare altro.
\subsection{Esempio di Smart Home in Ticino}
Come esempio di Smart Home in Ticino è stata scelta la casa di Jerry Lauber e Fabienne Lauber , clienti della Singenia e quindi anche della Loxone. Le abitudini dalla famiglia che abita in questa casa è riconducibile al profilo di cliente medio di aziende come la Singenia e la Loxone. Tuttavia è importante non dimenticare che le informazioni sono interamente ricavate da un’intervista fatta ai proprietari e sono di conseguenza soggettive e condizionate dal loro stile di vita, dalla programmazione del sistema e dal tipo di impianto che è stato installato, oltre che da impianti esterni che hanno comunque un impatto sul sistema domotico (come ad esempio i pannelli fotovoltaici). 
\subsubsection{Descrizione della casa}
La casa in questione è un’abitazione monofamiliare ed è ubicata nel bellinzonese in una zona abbastanza soleggiata tutto l’anno. È dotata di una piscina esterna e un impianto fotovoltaico che la rende praticamente autosufficiente (nonostante non possieda una batteria per accumulare l’energia elettrica prodotta in eccesso durante le ore di sole; quest’energia viene venduta all’azienda elettrica di riferimento e poi riacquistata nei momenti in cui i pannelli solari non sono in grado di produrre abbastanza elettricità da soddisfare il fabbisogno energetico in quel momento). Gli abitanti sono tre, due adulti e un bambino. La costruzione è terminata nel 2017 e l’impianto domotico è stato installato parallelamente alla fase di progettazione e di costruzione. Il suo prezzo è di approssimativamente CHF 25'000, di cui CHF 16'000 versati alla Singenia per i prodotti Loxone e per i servizi forniti, mentre la somma restante è stata spesa per il lavoro degli elettricisti (è comunque corretto notare che una parte di queste spese ci sarebbero state anche per un impianto elettrico tradizionale). La famiglia ha deciso di fare questo importante investimento per principalmente quattro motivi esposti nel rispettivo ordine d’importanza: passione e interesse per la tecnologia, comodità e abitabilità, per essere più ecologici e infine per risparmiare sulle bollette dell’elettricità in futuro. Sono venuti a conoscenza della possibilità di installare un impianto domotico per passione e interesse personale e hanno scelto la Singenia per svolgere il lavoro per conoscenze professionali. I clienti sono molto soddisfatti del lavoro svolto e della consulenza post-vendita offerta dalla ditta e affermano che l’impianto della Loxone ha soddisfatto a pieno le loro aspettative. 
\subsubsection{Tecnologie installate}
La casa è dotata di un sistema domotico parzialmente completo, ciò significa che non dispone di tutte le funzionalità potenzialmente offerte dalla Loxone ma possiede la predisposizione per averle tutte. Attualmente l’impianto domotico controlla le luci interne ed esterne, le tapparelle, l’impianto di climatizzazione (serpentine, termopompa e aria condizionata), le serrature, il sistema d’allarme e le telecamere interne ed esterne. L’impianto fotovoltaico, la piscina esterna e l’impianto multimediale invece hanno solamente la predisposizione (ciò significa che sono presenti tutti i cablaggi ma manca il pezzo finale per il funzionamento, ad esempio per la piscina manca solamente la pompa, che per una questione di costi hanno deciso di acquistare in futuro). I sensori più importanti installati sono quelli termici, luminosi, di movimento e all’esterno è presente un sensore che rileva la presenza di vento. Quest’ultimo collabora con i servizi meteo offerti dalla Loxone ed è in grado di salvaguardare le parti esterne più fragili della casa in caso di maltempo (ad esempio richiudendo la tenda parasole e le tapparelle). Un altro aspetto interessante è che la sensibilità di tutti i sensori può essere regolata in base alle esigenze specifiche degli abitanti. Il sistema più apprezzato dagli occupanti della casa è quello che gestisce le serpentine per il riscaldamento della casa, sistema che è molto preciso nel suo funzionamento ed è in grado di mantenere una temperatura omogenea e adeguata in tutte le stanze, condizione difficilmente ottenibili con le regolazioni manuali dei caloriferi.
Questa specifica casa possiede un sistema di accesso moto sofisticato e interessante dal profilo tecnologico ma anche della sicurezza (questo punto di vista verrà ripreso nel capitoletto seguente): ogni abitante oltre che possedere una classica chiave (fondamentale in caso di mancanza di corrente elettrica) è in possesso di un badge in grado di aprire le porte. Questo badge contiene le preferenze di utilizzo personalizzate di ogni persona, grazie a questo sistema la casa reagisce sempre in modo diverso in base a chi entra in casa e a chi invece esce, attivando anche l’allarme in modo automatico quando sono tutti fuori. Grazie a questa tecnologia è anche molto comodo ed economico (un badge costa all’incirca CHF 10) dare l’accesso ad altre persone con determinati vincoli: se ad esempio quando vado in vacanza e do l’accesso alla casa al mio vicino affinché possa annaffiarmi le piante, posso fare in modo di essere avvisato se questa persona si introduce in camera mia senza il mio consenso. Un altro esempio molto pratico è quello di dare l’accesso alla donna delle pulizie solamente in determinati orari di determinati giorni.

\subsubsection{Percezione di sicurezza e privacy}
Come già accennato nel capitoletto precedente, il sistema di accesso con il badge garantisce un livello ancora superiore di sicurezza rispetto a una chiave normale, questo perché il sistema domotico registra chi entra e chi esce, così da poter ricostruire i movimenti delle persone in caso di furti o danneggiamenti. La sicurezza è poi implementata dai sensori di movimento e alle telecamere. In caso di mancanza di corrente è comunque presente una batteria che garantisce il funzionamento dei sistemi di sicurezza. Grazie ai sensori di movimento la casa è in grado di capire quando ci sono delle intrusioni, questo perché sa quando la casa dovrebbe essere vuota grazie ai badge. Nel caso in cui nella casa dovesse abitare un animale domestico, i sensori possono essere regolati nella sensibilità alla percezione, così da evitare allarmi inutili. Il sistema di sicurezza è poi dotato da una rete di telecamere che riprendono sia gli spazi interni che quelli esterni. In caso di un’eventuale intrusione il sistema d’allarme avverte istantaneamente i telefoni degli abitanti della casa mostrando su di essi le riprese delle telecamere in tempo reale, lasciando così decidere all’utente se chiamare o meno la polizia (evitando così chiamate inutili in caso di falso allarme). È inoltre possibile impostare delle azioni che il sistema domotico attua per simulare la presenza di persone in casa, in questo caso si tratta di prevenzione. 
Per quanto riguarda i servizi collegati al web, questa casa ha elaborato un sistema praticamente inviolabile che utilizza tre reti diverse: una per l’impianto domotico, una per gli utenti della casa e una per gli ospiti. In questo modo la sicurezza è molto elevata (forse uno dei sistemi d’allarme più sicuri e sofisticati sul mercato) e la privacy è garantita non solo dalle due barriere di sicurezza offerta dalla Loxone discusse precedentemente, ma bensì tre: password della rete wireless, crittografazione dei dati e impossibilità di accedere ai dati nemmeno se si entra nella rete “normale” della casa. 

\subsubsection{Aspetti energetici rilevati}
Purtroppo la casa è nata già come Smart Home, di conseguenza non è possibile capire quanto si risparmia monetariamente grazie a questo impianto domotico. I proprietari però dichiarano che rispetto all’appartamento in cui vivevano prima, a parità di abitanti e stile ci vita, utilizzano molta meno elettricità, nonostante la casa sia più grande e possieda fattori di consumo energetico importanti (come ad esempio la piscina e l’aria condizionata). Le spese per la corrente elettrica sono praticamente meno della metà rispetto a prima, ma questo è dovuto anche alla presenza dell’impianto fotovoltaico, oltre che all’impianto domotico che sicuramente fa la sua parte (nonostante non sia ancora collegato alla piscina e ai pannelli solari). Sulle bollette dell’acqua non hanno riscontrato risparmi, questo perché il sistema installato non è stato progettato per lavorare nell’ambito idraulico. 
Entrando in casa si nota subito che il sistema domotico punta molto al risparmio energetico, spegnendo le luci che non vengono utilizzate in modo quasi immediato (grazie all’ausilio dei sensori di movimento che sono molti e distribuiti in modo molto strategico) e lavorando anche sulla quantità di luci da accendere per illuminare una zona e sull’intensità della luce. Sempre per la luce utilizza molto le radiazioni solari, regolando spesso le tapparelle. Oltre a lavorare molto sulla luminosità, agisce anche in modo importante sull’impianto di climatizzazione regolando i parametri in base alle abitudini della famiglia.
Una piccola svista della Loxone, sempre parlando di risparmio energetico, si trova nella centralina e nelle valvole che controllano il riscaldamento che hanno dei piccoli led che rimangono perennemente accesi. Questa è una piccola pecca che purtroppo va contro il concetto di risparmio minuzioso di energia. È comunque importante ricordare che l’impianto domotico per funzionare utilizza energia che altrimenti non verrebbe consumata, senza dimenticare però che grazie ad esso c’è un grande risparmio. Questo piccolo problema è quindi praticamente irrilevante rispetto ai benefici di risparmio, quindi il bilancio complessivo rimane comunque molto positivo.

\subsubsection{Pregi e difetti percepiti dal cliente}
Il maggiore pregio percepito dagli utenti della casa nell’anno e mezzo trascorso nella Smart Home sono sicuramente il comfort e la percezione di sicurezza, oltre che l’aspetto ecologico ed economico (inteso come risparmi sui costi dell’elettricità). Oltre a questi aspetti che sono già stati analizzati e approfonditi nei capitoli precedenti, ce ne sono molti altri: non è necessaria una manutenzione particolare, la ditta installatrice è molto presente nella fase post-vendita e offre un servizio veloce e di qualità. Molte regolazioni possono essere fatte dalla Singenia direttamente dall’ufficio, senza recarsi alla casa del cliente (la famiglia ha deciso di lasciare l’accesso momentaneo ai sistemi di controllo della casa al signor Baas per eventuali piccole regolazioni). Un altro pregio molto apprezzato dai proprietari è il fatto di essere indipendenti dalla Loxone, nel senso che il sistema domotico è perfettamente compatibile con prodotti anche di altre marche, lasciando così la scelta aperta e non vincolante dei prodotti. Se ad esempio vogliono installare un impianto stereo che sia controllato dall’impianto domotico per poter usufruire di tutti i comfort correlati ad esso non sono costretti ad acquistare il prodotto della Loxone ma possono sceglierne uno della marca che preferiscono senza perdere prestazioni.
Un altro aspetto molto interessante è la possibilità di aggiungere pulsantiere a piacimento dopo la costruzione della casa senza la necessità di ampliare la rete di cablaggi che convergono verso la centralina. Basta acquistare la pulsantiera e posizionarla dove di piacimento, dopodiché essa si collegherà al server tramite delle apposite frequenze radio. Un esempio fatto durante l’intervista è stato il seguente: molto frequentemente la famiglia viene visitata da un parente e spesso capita che l’unica persona presente in casa sia nell’ufficio al secondo piano a lavorare. Per questioni di comodità è stato installato un pulsante che apre la porta al piano terra in questo specifico locale, pulsante che non era stato previsto durante la fase di progettazione della casa e che avrebbe richiesto un importante intervento per essere aggiunto se non ci fosse stato questo sistema. 
Nonostante gli utenti della casa siano pienamente soddisfatti dell’importante investimento fatto per installare l’impianto domotico, hanno rilevato qualche difetto, anche se non particolarmente limitativo. Il primo punto è che per poter usufruire a pieno delle potenzialità del sistema e per fare in modo che funzioni senza interferire sulla routine degli utenti ci vuole parecchio tempo. Ancora oggi, dopo un anno e mezzo, ci sono delle piccole modifiche da fare. Un esempio molto banale era il piccolo robot aspirapolvere che azionava tutte le luci della casa quando puliva. Questo problema è stato risolto ricalibrando i sensori di movimento facendo in modo che non lo riconoscessero come se fosse una persona. Il tempo di assestamento è quindi abbastanza elevato, questo perché la famiglia ha dovuto imparare a usare il sistema e il sistema ha dovuto capire quali erano le abitudini degli abitanti della casa (in parte in modo automatico, in parte con delle modifiche fatte dall’ufficio della Singenia e infine con le piccole modifiche fatte direttamente dal telefono dei proprietari). Quest’ultimo aspetto ci porta al secondo difetto: nonostante la gestione del sistema sia molto indipendente dal cellulare rispetto ai prodotti della concorrenza, il telefono resta comunque uno strumento importante per poter usufruire di alcuni servizi come ad esempio la scelta degli scenari o di alcune configurazioni particolari. Il terzo e ultimo difetto riguarda la difficoltà di utilizzo per gli ospiti, specialmente per gli anziani. Questo problema però è relativo e dipende da quali servizi specifici decidono di usare gli ospiti (le luci ad esempio sono automatiche e quindi non richiedono conoscenze specifiche, lo stesso non vale per la regolazione manuale delle tapparelle che invece richiede la conoscenza della zona della pulsantiera da premere). Il problema più grande legato a queste tecnologie rimane comunque il prezzo che, per quanto garantisca dei risparmi energetici e quindi anche monetari, rimane ancora piuttosto elevato e difficilmente ammortizzabile.

\subsection{Considerazioni finali}
Come si è potuto notare, il mercato della domotica in Ticino non è particolarmente diffuso e conosciuto, questo anche a causa della piccola dimensione delle due aziende che operano in questo settore che appunto a causa della loro grandezza non possono fare importanti investimenti nell’ambito pubblicitario e promozionale. Il futuro pare comunque essere molto prosperoso, difatti già solo guardando gli ultimi anni le vendite sono aumentate in modo notevole, anche grazie al continuo progresso tecnologico. Come spiegava il signor Jörg Baas nell’intervista, nel resto della Svizzera la domotica è molto più conosciuta e ha un mercato molto più interessante sul profilo dei profitti. È però anche convinto che il Ticino, anche se un po’ in ritardo, raggiungerà i livelli nazionali . Per quanto riguarda i problemi legati alla sicurezza e alla privacy, come abbiamo visto, sono molto relativi e dipendono da sistema a sistema ma anche dalle precauzioni che il cliente adotta. I risparmi energetici sono garantiti, lo stesso vale per il comfort e il livello tecnologico che è molto alto. Abbiamo anche visto che i difetti, sempre prendendo come punto di riferimento la Loxone Smart Home, sono davvero minimi e quasi trascurabili. Quindi alla fine sta al cliente valutare, basandosi sulla sua disponibilità finanziaria e sulle sue priorità, se il costo di un impianto domotico vale tutti questi privilegi e se è in grado di trascurarne i difetti.


\newpage
\section{Conclusione}
Per concludere il lavoro, è importante far prevalere alcuni concetti che sono emersi durante la lettura dei vari capitoli. In particolar modo, si cerca di rispondere alla domanda: le Smart Home sono le case del futuro? 
Da quanto è emerso, si può sicuramente dire che le Smart Home hanno delle buone possibilità di essere quella che oggi chiamiamo “casa”. S’intende dire che è possibile che un giorno, bene o male tutte le case saranno Smart oppure saranno comunque convertite in un secondo momento. Il mercato è in crescita e la richiesta di sicurezza aumenta. Le persone rinunciano sempre di più inconsciamente alla privacy nella propria abitazione e dunque le Smart Home sono sicuramente destinate a persistere. 
Bisogna d’altro canto dire, però, che il mercato delle Smart Home nonostante si stia sviluppando sempre di più sul territorio (in particolar modo quello svizzero), non è ancora molto consolidato, dato che sono ancora relativamente poche le case completamente Smart oggi giorno.

\newpage
\section{Fonti}
\subsection{Bibliografia}
\noindent CARR N., (2008), Il lato oscuro della rete. Libertà, sicurezza, privacy, Rizzoli

\noindent CHESSA A., (2018), Smart Data, Egea

\noindent PIANO M., Energie rinnovabili e domotica: controlli ed ecosostenibilità nelle ZEB (Zero Energy Building), risparmio energetico, ESCO (Energy Service COmpany), F. Angeli

\noindent QUARANTA G., La domotica per l'efficienza energetica delle abitazioni, Maggioli editori

\noindent RICOTTI P., Sostenibilità e Green Economy, Quarto Settore. Competitività, Strategie e e Valore Aggiunto per le imprese del quarto millennio, F. Angeli

\noindent TRISCIUOGLIO, D. (2009). Introduzione alla domotica, Tecniche nuove

\noindent VANDI S. (2013-2014), Smart Home: Stato dell'arte della tecnolgia




\subsection{Interviste}
\noindent Fabienne Lauber, cliente della Singenia della Loxone, Gnosca 18 gennaio 2018

\noindent Jorg Baas, direttore e fondatore, Singenia Sagl, Giubiasco 15 novembre 2018
\subsection{Webgrafia}


\noindent A10, www.a10networks.com

\noindent ADMIN, www.admin.ch 

\noindent ADMIN, www.m4.ti.ch

\noindent AGENDA DIGITALE EU, www.agendadigitale.eu 

\noindent ALTERNATIVA SOSTENIBILE, www.alternativasostenibile.it

\noindent ARCHITETTURA E DOMOTICA, www.architetturaedomotica.it

\noindent AVG, www.avg.com 

\noindent BBC, www.bbc.com 

\noindent BRUNOSAETTA INTERNET E DIRITTO, www.brunosaetta.it 

\noindent BTICINO, www.bticino.com 

\noindent BUCAP, www.bucap.it 

\noindent BUSINESS INSIDER, www.businessinsider.com 

\noindent CLICLAVORO, www.cliclavoro.gov.it

\noindent CLOUDFLARE, www.cloudflare.com 

\noindent COMPUTERWORLD, www.cwi.it 

\noindent CONTROL4, www.control4.com

\noindent CORCOM, www.corrierecomunicazioni.it 

\noindent CORDIS, www.cordis.europa.eu 

\noindent CRESTRON,www.crestron.com

\noindent DATA PROTECTION LAW, www.dataprotectionlaw.it 

\noindent DIGITALTRENDS, www.digitaltrends.com 

\noindent DIRITTO.IT, www.diritto.it 

\noindent DIZIONARIO ITALIANO, www.dizionari.corriere.it 

\noindent DOGIARO, www.blog.dogiaro.com 

\noindent DOMOTICA PREZZI, www.domoticaprezzi.it

\noindent DOMOTICA.IT, www.domotica.it

\noindent EDILO.CH, www.edilo.ch 

\noindent EKKO, www.ekkolaprivacy.com 

\noindent ELDOMTRADE, www.eldomtrade.it 

\noindent ELETTRICO MAGAZINE, www.elettricomagazine.it 

\noindent ELETTRONICA NEWS, www.elettronicanews.it 

\noindent ELLEGI TECNOLOGIE, www.ellegisrl.biz

\noindent ENERBRAIN, www.enerbrain.ch

\noindent ENERCOM LUCE E GAS, www.enercomsrl.it 

\noindent ENERGY BOX, www.energybox.ch

\noindent ENERGY INTELLIGENCE, www.energyintelligence.it

\noindent ESET, www.eset.com 

\noindent EUROPEAN COMMISSION, www.ec.europa.eu 

\noindent EXPLAIN THAT STUFF, www.explainthatstuff.com 

\noindent FASTWEB, www.fastweb.it 

\noindent FORBES, www.forbes.com 

\noindent G2 CROWD, www.g2crowd.com 

\noindent GDPR EXPLAIED, www.gdprexplained.eu 

\noindent GEAK, www.cece.ch

\noindent GUARDIAN, www.theguardian.com 

\noindent HANGER BOCCHIOTTI, www.hanger-bocchiotti.it

\noindent HARTL HAUS, www.hartlhaus.at

\noindent HOME\&SMART, www.homeandsmart.de

\noindent HOW STUFF WORK, www.home.howstuffworks.com

\noindent HUBSPOT, www.blog.hubspot.com

\noindent IL SOLE 24 ORE, www.ilsole24ore.com 

\noindent INCENTIVI FOTOVOLTAICO, www.incentivifotovoltaico.name

\noindent INSTRORE, www.instoremag.it

\noindent INSURANCE UP, www.insuranceup.it 

\noindent INTELLIGENZA ARTIFICIALE, www.intelligenzaartificiale.it

\noindent INTEREL TRADING, www.iterel-trading.eu 

\noindent INTERNET 4 THINGS, www.internet4things.it 

\noindent INVESTOPEDIA, www.investopedia.com 

\noindent IOT AGENDA, www.internetofthingsagenda.tech 

\noindent IOT EVOLUTION, www.iotevolutionworld.com 

\noindent IOT FOR ALL, www.iotforall.com 

\noindent JSTOR, www.jstor.org 

\noindent KASPESPERY LAB, www.kaspespery.com 

\noindent KEY4BIZ, www.key4biz.it 

\noindent L'ESPRESSO, www.espresso.repubblica.it  

\noindent LA STAMPA, www.lastampa.it

\noindent LEGA AMBIENTE, www.legambiente.it

\noindent LIBERO TECNOLOGIA, www.tecnologia.libero.it 

\noindent LOXONE, www.loxone.com

\noindent LUISS GIUDO CARLI, www.tesi.luiss.it

\noindent MEDIUM, www.medium.com 

\noindent MINERGIE, www.minergie.ch

\noindent NETPHAROS, www.netpharos.it 

\noindent OSSERVATORI, www.osservatori.net

\noindent OSSERVATORI.NET DIGITAL INNOVATION, www.blog.osservatori.net

\noindent PCMAG UK, www.uk.pcmag.com 

\noindent PG CASA, www.pgcasa.it

\noindent PMI.IT, www.pmi.it 

\noindent PREDICTING OUR FUTURE, www.predictingourfuture.com 

\noindent PRIVACY ITALIA, www.privacyitalia.eu 

\noindent PROTEZIONE DATI PERSONALI, www.protezionedatipersonali.it  

\noindent REGISTRO DI COMMERCIO, ti.chregister.ch.

\noindent REPUBBLICA, www.repubblica.it

\noindent RETE ECOLOGICA, www.retecologica.it

\noindent RIDBLE, www.ridble.com 

\noindent RISTRUTTURAZIONE EDILIZIA, www.ristrutturazionedilizia.com

\noindent SCIENZA GIOVANE, www.scienzagiovane.unibo.it

\noindent SCIENZA IN RETE, www.scienzainrete.it

\noindent SCIENZA VERDE, www.scienzaverde.it

\noindent SECSOLUTION SECURITY ONLINE MAGAZINE, www.secsolution.com 

\noindent SECURELIST, www.securelist.com 

\noindent SINGENIA, www.singenia.ch

\noindent SMART-NOTIZIE, www.smart-notizie.it 

\noindent SMARTHOMEUSA, www.smarthomeusa.com 

\noindent STATISTA, www.statista.com 

\noindent SUVA, www.suva.ch 

\noindent SVIZZERA ENERGIA, www.svizzeraenergia.ch

\noindent SWISSCOM, www.swisscom.ch 

\noindent TECHOPEDIA, www.techopedia.com 

\noindent TECNOANDROID, www.tecnoandroid.it 

\noindent THE NEW YORK TIMES, www.nytimes.com 

\noindent THE NEXT WEB, www.thenextweb.com 

\noindent TICINO DOMOTICA, www.ticinodomotica.ch

\noindent TICINO ENERGIA, www.ticinoenergia.ch

\noindent TOM'S Guide, www.tomsguide.com 


\noindent TRECCANI, www.treccani.it

\noindent VENTUREBEAT, www.venturebeat.com 

\noindent WE LIVE SECURITY, www.welivesecurity.com 

\noindent WEBCASA24, www.webcasa24.ch

\noindent WEBNEWS, www.webnews.it 

\noindent WIKIPEDIA, www.wikipedia.org 

\noindent WIRED, www.wired.it 

\noindent YOUTUBE, www.youtube.com



\end{document}