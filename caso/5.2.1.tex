Singenia Sagl è una piccola società a garanzia limitata fondata il 24 novembre 2010 da Jörg Baas , tuttora dirigente. Dopo la scuola obbligatoria, il signor Baas ha proseguito la sua formazione alla Scuola Cantonale di Commercio di Bellinzona, dopodiché ha conseguito un titolo di master in economia presso l’università di Friborgo. Dopo la laurea ha lavorato per breve tempo in un’azienda e in seguito si è messo in proprio fondando la Singenia Sagl, con la speranza di introdursi in un mercato che ai tempi dell’apertura era agli inizi e che si sarebbe sviluppato in modo positivo. L’azienda è stata finanziata interamente con il capitale proprio del proprietario. Ad oggi la ditta conta due dipendenti (i dipendenti non devono possedere una formazione specifica, dato che l’installazione viene fatta da elettricisti esterni all’azienda, l’importante è avere buone competenze nel campo della pianificazione e progettazione edile e conoscere bene tutti i prodotti offerti).
La mansione principale della ditta consiste nella consulenza, pianificazione, programmazione e realizzazione di impianti intelligenti per case e appartamenti, offrendo così un’integrazione completa di sistemi domotici. Nello svolgimento di questo percorso è di fondamentale importanza la collaborazione con altre aziende o artigiani specializzati come ad esempio elettricisti, finestristi, idraulici, installatori di impianti di riscaldamento, eccetera. La sede della ditta si trova a Giubiasco e al suo interno è stato installato un impianto domotico con scopo dimostrativo, oltre che vari tipi di prodotti in esposizione. Per il momento la Singenia opera solamente sul territorio ticinese, anche se eccezionalmente svolge piccoli mandati nel canton Grigioni. La concorrenza esiste ma non è particolarmente invasiva per quanto riguarda la ripartizione del mercato, difatti sul suolo ticinese è presente solamente la Ticino Domotica con sede a Lugano . I clienti variano da privati a aziendali e al momento non sono particolarmente numerosi dato che si tratta di una tecnologia poco pubblicizzata nella regione, anche se è da considerare che pian piano è un settore in crescita.
Lo scopo dell’azienda è il seguente: 
“Il commercio, l'importazione e l'esportazione, la promozione, lo sviluppo software e hardware, la consulenza, la progettazione e la produzione, l'installazione, l'assistenza e la manutenzione di sistemi di domotica atti al miglioramento della qualità della vita delle persone, quali l'illuminotecnica, la sicurezza, l'intrattenimento multimediale, la climatizzazione, il risparmio energetico, l'automazione di immobili, la comunicazione .”
