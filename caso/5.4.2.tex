La casa è dotata di un sistema domotico parzialmente completo, ciò significa che non dispone di tutte le funzionalità potenzialmente offerte dalla Loxone ma possiede la predisposizione per averle tutte. Attualmente l’impianto domotico controlla le luci interne ed esterne, le tapparelle, l’impianto di climatizzazione (serpentine, termopompa e aria condizionata), le serrature, il sistema d’allarme e le telecamere interne ed esterne. L’impianto fotovoltaico, la piscina esterna e l’impianto multimediale invece hanno solamente la predisposizione (ciò significa che sono presenti tutti i cablaggi ma manca il pezzo finale per il funzionamento, ad esempio per la piscina manca solamente la pompa, che per una questione di costi hanno deciso di acquistare in futuro). I sensori più importanti installati sono quelli termici, luminosi, di movimento e all’esterno è presente un sensore che rileva la presenza di vento. Quest’ultimo collabora con i servizi meteo offerti dalla Loxone ed è in grado di salvaguardare le parti esterne più fragili della casa in caso di maltempo (ad esempio richiudendo la tenda parasole e le tapparelle). Un altro aspetto interessante è che la sensibilità di tutti i sensori può essere regolata in base alle esigenze specifiche degli abitanti. Il sistema più apprezzato dagli occupanti della casa è quello che gestisce le serpentine per il riscaldamento della casa, sistema che è molto preciso nel suo funzionamento ed è in grado di mantenere una temperatura omogenea e adeguata in tutte le stanze, condizione difficilmente ottenibili con le regolazioni manuali dei caloriferi.
Questa specifica casa possiede un sistema di accesso moto sofisticato e interessante dal profilo tecnologico ma anche della sicurezza (questo punto di vista verrà ripreso nel capitoletto seguente): ogni abitante oltre che possedere una classica chiave (fondamentale in caso di mancanza di corrente elettrica) è in possesso di un badge in grado di aprire le porte. Questo badge contiene le preferenze di utilizzo personalizzate di ogni persona, grazie a questo sistema la casa reagisce sempre in modo diverso in base a chi entra in casa e a chi invece esce, attivando anche l’allarme in modo automatico quando sono tutti fuori. Grazie a questa tecnologia è anche molto comodo ed economico (un badge costa all’incirca CHF 10) dare l’accesso ad altre persone con determinati vincoli: se ad esempio quando vado in vacanza e do l’accesso alla casa al mio vicino affinché possa annaffiarmi le piante, posso fare in modo di essere avvisato se questa persona si introduce in camera mia senza il mio consenso. Un altro esempio molto pratico è quello di dare l’accesso alla donna delle pulizie solamente in determinati orari di determinati giorni.
