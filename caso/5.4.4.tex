Purtroppo la casa è nata già come Smart Home, di conseguenza non è possibile capire quanto si risparmia monetariamente grazie a questo impianto domotico. I proprietari però dichiarano che rispetto all’appartamento in cui vivevano prima, a parità di abitanti e stile ci vita, utilizzano molta meno elettricità, nonostante la casa sia più grande e possieda fattori di consumo energetico importanti (come ad esempio la piscina e l’aria condizionata). Le spese per la corrente elettrica sono praticamente meno della metà rispetto a prima, ma questo è dovuto anche alla presenza dell’impianto fotovoltaico, oltre che all’impianto domotico che sicuramente fa la sua parte (nonostante non sia ancora collegato alla piscina e ai pannelli solari). Sulle bollette dell’acqua non hanno riscontrato risparmi, questo perché il sistema installato non è stato progettato per lavorare nell’ambito idraulico. 
Entrando in casa si nota subito che il sistema domotico punta molto al risparmio energetico, spegnendo le luci che non vengono utilizzate in modo quasi immediato (grazie all’ausilio dei sensori di movimento che sono molti e distribuiti in modo molto strategico) e lavorando anche sulla quantità di luci da accendere per illuminare una zona e sull’intensità della luce. Sempre per la luce utilizza molto le radiazioni solari, regolando spesso le tapparelle. Oltre a lavorare molto sulla luminosità, agisce anche in modo importante sull’impianto di climatizzazione regolando i parametri in base alle abitudini della famiglia.
Una piccola svista della Loxone, sempre parlando di risparmio energetico, si trova nella centralina e nelle valvole che controllano il riscaldamento che hanno dei piccoli led che rimangono perennemente accesi. Questa è una piccola pecca che purtroppo va contro il concetto di risparmio minuzioso di energia. È comunque importante ricordare che l’impianto domotico per funzionare utilizza energia che altrimenti non verrebbe consumata, senza dimenticare però che grazie ad esso c’è un grande risparmio. Questo piccolo problema è quindi praticamente irrilevante rispetto ai benefici di risparmio, quindi il bilancio complessivo rimane comunque molto positivo.
