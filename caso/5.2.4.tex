Dal punto di vista del risparmio energetico si possono individuare molteplici vantaggi, sia per quanto riguarda l’elettricità che l’acqua. Gli impianti domotici offerti dalla Singenia sono in grado di ottimizzare al massimo il risparmio energetico agendo su diversi impianti domestici: direttamente sul riscaldamento e sull’apertura/chiusura delle finestre, adattando così la temperatura di ogni singola stanza in base al fabbisogno in quel momento (ad esempio durante le ore di sonno riscalda la camera da letto e abbassa la temperatura in tutte le altre stanze, quando la casa è vuota abbassa la temperatura generale; ovviamente questi protocolli di tipo routinario si adeguano a eventuali cambiamenti grazie ai sensori che percepiscono la presenza di soggetti); può agire sull’impianto elettrico controllando così l’illuminazione sia interna che esterna utilizzando le tapparelle o le luci, evitando di sprecare elettricità laddove non è necessario. 
È molto interessante notare che l’efficienza massima di un sistema domotico si ottiene con l’abbinamento con i pannelli fotovoltaici, questo perché la Smart Home è in grado di capire quando la casa sta producendo elettricità oppure quanto la sta acquistando dalla rete di riferimento. Se ad esempio ho un’auto elettrica e voglio che sia carica per le 19 di sera in modo tale da poter uscire a cena, la casa gestirà la ricarica nel modo più intelligente ed efficiente possibile (anche facendo riferimento alle previsioni meteorologiche) ricaricandola durante le ore in cui il sole, nel limite del possibile, è presente per poter sfruttare l’impianto fotovoltaico. Sempre tenendo l’esempio dell’auto elettrica, l’impianto è in grado di caricare il veicolo durante le ore in cui l’elettricità costa meno, come ad esempio di notte (dando però sempre e comunque la precedenza all’impianto fotovoltaico). Lo stesso discorso vale per tutti gli elettrodomestici e dispositivi elettronici. Un altro importante risparmio di elettricità avviene grazie al controllo degli apparecchi in stand-by, dato che spesso e volentieri rimane una spia led accesa. Per risolvere questo problema, tutte le volte che gli occupanti dell’abitazione escono oppure vanno a dormire la casa provvede a spegnere tutto ciò che è rimasto acceso e che non serve in quel momento. Tutti questi aspetti, pensati per il risparmio energetico, sono regolabili e gestibili dall’utente; è importante sottolineare che l’impianto domotico metterà sempre il confort come priorità.
I risparmi monetari rispetto a un’abitazione tradizionale sono notevoli, tuttavia non è possibile stimare una cifra precisa perché è strettamente collegato alle preferenze di funzionamento impostate dal proprietario (ad esempio un utente che vuole una temperatura di 28 gradi in piscina tutto il giorno consumerà più risorse energetiche rispetto a uno che si accontenta di 25 gradi e la usa una volta al giorno) e dalla diversità delle abitazioni (ad esempio con o senza piscina).
