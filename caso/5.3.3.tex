La Loxone Smart Home garantisce al cliente dei livelli di sicurezza e privacy molto elevati, soprattutto considerando il confronto con i prodotti della concorrenza che invece puntano maggiormente sulla connettività e altri servizi collegati al mondo del Web. Rinunciando a queste caratteristiche Loxone è comunque riuscita a diventare leader del mercato e anzi, a farsi apprezzare proprio per questo motivo. Affinché il sistema detto “a scatola chiusa” riesca a funzionare senza il collegamento alla rete internet è necessaria un’istallazione di un software molto complesso nel server e che sia in grado di elaborare in modo intelligente i dati che riceve. Il collegamento alla rete è necessario solamente per usufruire di alcuni servizi (ad esempio per il controllo dello stato della casa e l’azionamento di alcune funzioni) e per fare gli aggiornamenti del software. Tra i servizi precedentemente citati non sono compresi ad esempio l’apertura delle serrature e delle tapparelle. Il sistema è comunque molto difficile da attaccare perché la rete della casa dev’essere protetta da una password e tutto il sistema gode di una criptazione molto complesso. 