Come si è potuto notare, il mercato della domotica in Ticino non è particolarmente diffuso e conosciuto, questo anche a causa della piccola dimensione delle due aziende che operano in questo settore che appunto a causa della loro grandezza non possono fare importanti investimenti nell’ambito pubblicitario e promozionale. Il futuro pare comunque essere molto prosperoso, difatti già solo guardando gli ultimi anni le vendite sono aumentate in modo notevole, anche grazie al continuo progresso tecnologico. Come spiegava il signor Jörg Baas nell’intervista, nel resto della Svizzera la domotica è molto più conosciuta e ha un mercato molto più interessante sul profilo dei profitti. È però anche convinto che il Ticino, anche se un po’ in ritardo, raggiungerà i livelli nazionali . Per quanto riguarda i problemi legati alla sicurezza e alla privacy, come abbiamo visto, sono molto relativi e dipendono da sistema a sistema ma anche dalle precauzioni che il cliente adotta. I risparmi energetici sono garantiti, lo stesso vale per il comfort e il livello tecnologico che è molto alto. Abbiamo anche visto che i difetti, sempre prendendo come punto di riferimento la Loxone Smart Home, sono davvero minimi e quasi trascurabili. Quindi alla fine sta al cliente valutare, basandosi sulla sua disponibilità finanziaria e sulle sue priorità, se il costo di un impianto domotico vale tutti questi privilegi e se è in grado di trascurarne i difetti.

