Il maggiore pregio percepito dagli utenti della casa nell’anno e mezzo trascorso nella Smart Home sono sicuramente il comfort e la percezione di sicurezza, oltre che l’aspetto ecologico ed economico (inteso come risparmi sui costi dell’elettricità). Oltre a questi aspetti che sono già stati analizzati e approfonditi nei capitoli precedenti, ce ne sono molti altri: non è necessaria una manutenzione particolare, la ditta installatrice è molto presente nella fase post-vendita e offre un servizio veloce e di qualità. Molte regolazioni possono essere fatte dalla Singenia direttamente dall’ufficio, senza recarsi alla casa del cliente (la famiglia ha deciso di lasciare l’accesso momentaneo ai sistemi di controllo della casa al signor Baas per eventuali piccole regolazioni). Un altro pregio molto apprezzato dai proprietari è il fatto di essere indipendenti dalla Loxone, nel senso che il sistema domotico è perfettamente compatibile con prodotti anche di altre marche, lasciando così la scelta aperta e non vincolante dei prodotti. Se ad esempio vogliono installare un impianto stereo che sia controllato dall’impianto domotico per poter usufruire di tutti i comfort correlati ad esso non sono costretti ad acquistare il prodotto della Loxone ma possono sceglierne uno della marca che preferiscono senza perdere prestazioni.
Un altro aspetto molto interessante è la possibilità di aggiungere pulsantiere a piacimento dopo la costruzione della casa senza la necessità di ampliare la rete di cablaggi che convergono verso la centralina. Basta acquistare la pulsantiera e posizionarla dove di piacimento, dopodiché essa si collegherà al server tramite delle apposite frequenze radio. Un esempio fatto durante l’intervista è stato il seguente: molto frequentemente la famiglia viene visitata da un parente e spesso capita che l’unica persona presente in casa sia nell’ufficio al secondo piano a lavorare. Per questioni di comodità è stato installato un pulsante che apre la porta al piano terra in questo specifico locale, pulsante che non era stato previsto durante la fase di progettazione della casa e che avrebbe richiesto un importante intervento per essere aggiunto se non ci fosse stato questo sistema. 
Nonostante gli utenti della casa siano pienamente soddisfatti dell’importante investimento fatto per installare l’impianto domotico, hanno rilevato qualche difetto, anche se non particolarmente limitativo. Il primo punto è che per poter usufruire a pieno delle potenzialità del sistema e per fare in modo che funzioni senza interferire sulla routine degli utenti ci vuole parecchio tempo. Ancora oggi, dopo un anno e mezzo, ci sono delle piccole modifiche da fare. Un esempio molto banale era il piccolo robot aspirapolvere che azionava tutte le luci della casa quando puliva. Questo problema è stato risolto ricalibrando i sensori di movimento facendo in modo che non lo riconoscessero come se fosse una persona. Il tempo di assestamento è quindi abbastanza elevato, questo perché la famiglia ha dovuto imparare a usare il sistema e il sistema ha dovuto capire quali erano le abitudini degli abitanti della casa (in parte in modo automatico, in parte con delle modifiche fatte dall’ufficio della Singenia e infine con le piccole modifiche fatte direttamente dal telefono dei proprietari). Quest’ultimo aspetto ci porta al secondo difetto: nonostante la gestione del sistema sia molto indipendente dal cellulare rispetto ai prodotti della concorrenza, il telefono resta comunque uno strumento importante per poter usufruire di alcuni servizi come ad esempio la scelta degli scenari o di alcune configurazioni particolari. Il terzo e ultimo difetto riguarda la difficoltà di utilizzo per gli ospiti, specialmente per gli anziani. Questo problema però è relativo e dipende da quali servizi specifici decidono di usare gli ospiti (le luci ad esempio sono automatiche e quindi non richiedono conoscenze specifiche, lo stesso non vale per la regolazione manuale delle tapparelle che invece richiede la conoscenza della zona della pulsantiera da premere). Il problema più grande legato a queste tecnologie rimane comunque il prezzo che, per quanto garantisca dei risparmi energetici e quindi anche monetari, rimane ancora piuttosto elevato e difficilmente ammortizzabile.
