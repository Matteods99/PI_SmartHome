Come già accennato nel capitoletto precedente, il sistema di accesso con il badge garantisce un livello ancora superiore di sicurezza rispetto a una chiave normale, questo perché il sistema domotico registra chi entra e chi esce, così da poter ricostruire i movimenti delle persone in caso di furti o danneggiamenti. La sicurezza è poi implementata dai sensori di movimento e alle telecamere. In caso di mancanza di corrente è comunque presente una batteria che garantisce il funzionamento dei sistemi di sicurezza. Grazie ai sensori di movimento la casa è in grado di capire quando ci sono delle intrusioni, questo perché sa quando la casa dovrebbe essere vuota grazie ai badge. Nel caso in cui nella casa dovesse abitare un animale domestico, i sensori possono essere regolati nella sensibilità alla percezione, così da evitare allarmi inutili. Il sistema di sicurezza è poi dotato da una rete di telecamere che riprendono sia gli spazi interni che quelli esterni. In caso di un’eventuale intrusione il sistema d’allarme avverte istantaneamente i telefoni degli abitanti della casa mostrando su di essi le riprese delle telecamere in tempo reale, lasciando così decidere all’utente se chiamare o meno la polizia (evitando così chiamate inutili in caso di falso allarme). È inoltre possibile impostare delle azioni che il sistema domotico attua per simulare la presenza di persone in casa, in questo caso si tratta di prevenzione. 
Per quanto riguarda i servizi collegati al web, questa casa ha elaborato un sistema praticamente inviolabile che utilizza tre reti diverse: una per l’impianto domotico, una per gli utenti della casa e una per gli ospiti. In questo modo la sicurezza è molto elevata (forse uno dei sistemi d’allarme più sicuri e sofisticati sul mercato) e la privacy è garantita non solo dalle due barriere di sicurezza offerta dalla Loxone discusse precedentemente, ma bensì tre: password della rete wireless, crittografazione dei dati e impossibilità di accedere ai dati nemmeno se si entra nella rete “normale” della casa. 
