In questo quarto e ultimo capitolo sulle Smart Home verranno analizzati tre casi reali, due aziende e un cliente di esse. La prima ditta e il cliente sono residenti sul suolo ticinese, mentre la seconda azienda è austriaca ed è specializzata unicamente nella produzione di componenti per Smart Home ed è la più importante nel settore della domotica. L’idea è quella di andare a toccare i principali punti trattati nei capitoli precedenti e capire come sono applicati alla piccola realtà ticinese, la quale è spesso vista come inferiore dal punto di vista dello sviluppo tecnologico dal resto della Svizzera. Le due aziende serviranno a capire quanto è forte economicamente questo mercato e se ha un futuro in Ticino, mentre l’ultimo caso avrà lo scopo di capire quanto realmente è vantaggiosa una casa intelligente rispetto a un’abitazione tradizionale basandosi sui dati reali forniti da un cliente. La struttura dei capitoletti consisterà in una breve contestualizzazione generale in cui verranno fornite le informazioni generiche del caso, dopodiché verranno trattati i temi legati alla tecnologia, sicurezza e privacy, aspetti energetici e infine i pregi e i difetti. Il capitolo terminerà con la conclusione in cui verrà brevemente riassunto il contenuto del capitolo e verrà svolta una riflessione. Lo scopo è quello di capire quanto sono realmente evoluti questi sistemi ad oggi e come sono percepiti dalla società, cercando quindi di smentire o confermare le critiche in modo più oggettivo possibile, che sono all’ordine del giorno quando si parla di Smart Home. È importante notare che il capitolo della Singenia Sagl e quello dell’esempio di Smart Home in Ticino sono tratti da due interviste e sono di conseguenza soggetti alle opinioni e stili di vita delle persone intervistate.