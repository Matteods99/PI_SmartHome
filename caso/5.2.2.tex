Per quanto riguarda le tecnologie offerte alla clientela, la Singenia si affida principalmente a tre aziende specializzate nella produzione di componenti per impianti domotici: Loxone, Crestron (azienda americana, del New Jersey specializzata nella produzione di svariati prodotti digitali come telecomandi, impianti stereo e impianti domotici, non molto presente nel mercato svizzero ) e Control4 (grossa azienda americana nata a Salt Lake City presente in più di 100 nazioni tra cui la Svizzera e grande produttrice di impianti intelligenti per case ed edifici ). Il fornitore più importante è il primo, azienda austriaca che verrà presentata nel dettaglio nel secondo capitoletto. Queste aziende forniscono tutti i componenti in parti separate (ad esempio centralina, valvole, sensori, pulsantiere, eccetera), in modo che i collaboratori della Singenia siano in grado di progettare e realizzare un prodotto personalizzato per ogni cliente.
I sistemi installati vanno a toccare i seguenti aspetti della casa: illuminazione (luci e tapparelle), sicurezza (serrature, sistemi d’allarme e sensori antiincendio, antiallagamento e sensori termici per le placche), intrattenimento multimediale (televisione e stereo) e climatizzazione (riscaldamento, aria condizionata e finestre). 
Nella maggior parte dei casi (circa il 95\%) il sistema domotico viene installato quando lo stabile è già costruito o addirittura in fase di ristrutturazione, l’altra opzione è che viene già integrato nell’edificio durante la fase di costruzione. Quest’ultima è la migliore perché permette di progettare l’impianto elettrico già in funzione dell’impianto che la Singenia installerà in seguito, in caso contrario è necessario intervenire in modo più accentuato e di conseguenza più costoso.
