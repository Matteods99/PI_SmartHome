Loxone Electronics GmbH, come già accennato nel capitoletto precedente, è un’azienda austriaca specializzata nella progettazione e produzione di tutti i componenti necessari alla costruzione di un impianto domotico di ultima generazione. È un’azienda di grandi dimensioni fondata da due giovani imprenditori austriaci, Thomas Moser e Martin Öller. L’apertura risale al 2008 e ad oggi Loxone rifornisce molte nazioni in tutto il mondo, la maggior parte di queste possiedono anche diversi punti di distribuzione. L’idea di aprire l’azienda è nata dopo aver avuto l’intuizione di rendere le abitazioni completamente automatizzate ispirandosi ai robot tagliaerba e alle auto con guida autonoma. In realtà in parte questa possibilità esisteva già sul mercato, tuttavia le apparecchiature erano poco intuitive da utilizzare e soprattutto avevano un prezzo praticamente inaccessibile per la maggior parte delle persone. Si può quindi affermare che la fondazione di questa grossa ditta sia nata da una necessità personale, come spiega Martin Öller:
“Pensavamo che dovesse essere possibile creare un sistema semplice, intelligente e allo stesso tempo abbordabile dal punto di vista dei costi! Volevamo un sistema che di notte accendesse la luce in camera da letto, se hai la necessità di uscire, senza dover cercare l’interruttore. Volevamo un sistema che facesse sì che le tapparelle e le lamelle delle veneziane seguissero il corso del sole e che il riscaldamento riducesse la potenza quando la temperatura è sufficientemente alta…” (Martin Öller Fondatore e CEO Loxone, www.loxone.com).
È così nato il concetto che ha portato la Loxone al successo, la “Loxone Smart Home”, che inizialmente era una soluzione applicata alle case dei due fondatori, in seguito dopo essere stata rivista e perfezionata ha fatto il suo esordio sul mercato austriaco riscontrando un buon successo grazie alle seguenti caratteristiche spiegate da Thomas Moser:
“Loxone Smart Home non aumenta solo il comfort, l’efficienza energetica e la sicurezza. È un sistema previdente: ad esempio accende il riscaldamento per tempo per trovare la casa calda e confortevole al rientro. Senza che io debba intervenire. E se è necessario, è addirittura possibile controllare da lontano se il ferro da stiro è spento. Senza intaccare la privacy ovviamente.” (Thomas Moser, Fondatore e CEO Loxone, www.loxone.com).
Al momento l’azienda è la leader nel proprio settore e occupa all’incirca 250 dipendenti sparsi per il mondo, oltre che i rivenditori autorizzati (come ad esempio la Singenia Sagl, trattata nel capitoletto precedente). La loro strategia è quella di puntare sulla qualità del prodotto sempre mantenendo un prezzo accessibile alla maggior parte delle persone, oltre che puntare molto sulla sicurezza e sulla privacy (temi molto delicati quanto di parla di domotica). I principali clienti appartengono a diverse categorie: proprietari di case, di appartamenti, di hotel o aziende; tuttavia c’è un’idea che li accomuna: il fatto di vivere l’esperienza abitativa in modo moderno e confortevole. Un aspetto molto interessante da notare è che i prodotti sono fatti su misura per ogni cliente, finora sono stati implementati migliaia di progetti per molti clienti sparsi in tutto il mondo. La missione della Loxone è che tutti vivano in una Smart Home. 
