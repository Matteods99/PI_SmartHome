La casa in questione è un’abitazione monofamiliare ed è ubicata nel bellinzonese in una zona abbastanza soleggiata tutto l’anno. È dotata di una piscina esterna e un impianto fotovoltaico che la rende praticamente autosufficiente (nonostante non possieda una batteria per accumulare l’energia elettrica prodotta in eccesso durante le ore di sole; quest’energia viene venduta all’azienda elettrica di riferimento e poi riacquistata nei momenti in cui i pannelli solari non sono in grado di produrre abbastanza elettricità da soddisfare il fabbisogno energetico in quel momento). Gli abitanti sono tre, due adulti e un bambino. La costruzione è terminata nel 2017 e l’impianto domotico è stato installato parallelamente alla fase di progettazione e di costruzione. Il suo prezzo è di approssimativamente CHF 25'000, di cui CHF 16'000 versati alla Singenia per i prodotti Loxone e per i servizi forniti, mentre la somma restante è stata spesa per il lavoro degli elettricisti (è comunque corretto notare che una parte di queste spese ci sarebbero state anche per un impianto elettrico tradizionale). La famiglia ha deciso di fare questo importante investimento per principalmente quattro motivi esposti nel rispettivo ordine d’importanza: passione e interesse per la tecnologia, comodità e abitabilità, per essere più ecologici e infine per risparmiare sulle bollette dell’elettricità in futuro. Sono venuti a conoscenza della possibilità di installare un impianto domotico per passione e interesse personale e hanno scelto la Singenia per svolgere il lavoro per conoscenze professionali. I clienti sono molto soddisfatti del lavoro svolto e della consulenza post-vendita offerta dalla ditta e affermano che l’impianto della Loxone ha soddisfatto a pieno le loro aspettative. 