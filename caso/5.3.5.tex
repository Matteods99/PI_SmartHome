Oltre a comfort e sicurezza, i prodotti Loxone offrono un’ampia scelta di materiali e colori per ogni prodotto in modo da abbinarsi perfettamente agli stili delle abitazioni in cui verranno istallati. Un altro aspetto molto interessante è che l’azienda ha costruito delle case che fungono da showroom, ovvero case che integrano l’impianto domotico al loro interno e che il cliente può visitare prima di acquistare il prodotto per comprenderne al meglio il funzionamento anche attraverso la consulenza offerta dai collaboratori presenti. Questo progetto prende il nome di “The Loxone Experience Tour” e l’abitazione più grande si trova a Vienna. Secondo Loxone la Loxone Smart Home è la prima casa 3.0, ovvero che è in grado di prendere decisioni da sola e non più solamente attraverso la gestione via Smartphone dell’utente. Grazie a queste tecnologie la Loxone stima che in media ogni persona risparmia all’incirca 61'570 azioni all’anno (come ad esempio accendere la luce o aprire la porta quando arriva un ospite), ciò fa risparmiare all’utente molto tempo che può invece essere utilizzato per fare altro.