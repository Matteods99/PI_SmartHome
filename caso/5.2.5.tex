Il primo pregio che si apprezza vivendo in una Smart Home è sicuramente il comfort, seguito dal risparmio energetico e dalla sicurezza di cui si può beneficiare, temi trattati nei capitoletti precedenti. Al momento non sono previsti sussidi per chi costruisce una casa intelligente, ne dai cantoni, ne dalla confederazione (e secondo il parere del signor Baas non ci saranno, o almeno non nel futuro più prossimo, questo per due motivi: perché al giorno d’oggi chi costruisce una Smart Home lo fa più che altro per il comfort offerto da essa e in secondo luogo perché si tratta di una tecnologia relativamente nuova e non molto discussa come invece lo sono i pannelli fotovoltaici ); tuttavia spesso avendo un impianto domotico installato si raggiungono anche gli standard “Minergie”. Ottenere questa certificazione da invece accesso a sussidi e agevolazioni varie a dipendenza del luogo. I requisiti minimi per ottenere questo certificato sono i seguenti :
\begin{enumerate}
\item  Requisito principale: indice Minergie (nuove costruzioni: 55 kWh/(m2*a))

\item  Requisito supplementare sul fabbisogno termico per il riscaldamento per edifici nuovi (involucro edilizio): identico al MoPEC 2014

\item  Requisito supplementare sull'indice di energia termica finale senza PV: 35 kWh/(m2*a) per nuove costruzioni, 60 kWh/(m2*a) per ammodernamenti

\item  Produzione propria di elettricità almeno come richiesto dal MoPEC 2014 (10 W/m2 SRE)

\item  Ricambio dell'aria controllato e protezione termica estiva

\item  Nuove costruzioni senza combustibili fossili 

\item  Necessario un concetto di tenuta all'aria, ma nessuna misurazione obbligatoria

\item  Richiesto il monitoraggio dell'energia per edifici di superficie superiore a 2'000 m2 SRE 

\item  Semplici misure strutturali per l'idoneità alla mobilità elettrica degli edifici Minergie
\end{enumerate}
Dato che non ci sono problemi legati alla privacy dei dati grazie al sistema a “scatola chiusa”, al momento l’unico difetto che si può riscontrare dai servizi offerti dalla Singenia è il prezzo. Anche in questo caso è molto difficile stimare un prezzo medio perché il costo dell’impianto dipende da molteplici fattori: grandezza della casa, predisposizione del sistema elettrico, scelta degli accessori e delle funzioni (si possono personalizzare affinché rispecchino perfettamente i bisogni del cliente), vari servizi online (come la meteo), eccetera.
Il prezzo indicativo di un impianto mediamente completo per una casa unifamiliare si aggira intorno ai CHF 7’000-8'000, costo degli elettricisti escluso. Come spiegava il signor Baas durante l’intervista, i prodotti da loro offerti sono comparabili alla Volkswagen per le automobili: hanno un’ottima qualità e un buon livello tecnologico a un prezzo ragionevole. Ci sono poi altri prodotti molto più lussuosi e di conseguenza anche più costosi che però sono venduti da altre ditte specializzate .

