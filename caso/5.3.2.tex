Come accennato in precedenza Loxone produce impianti domotici. Questi ultimi vengono venduti separatamente in modo da poter progettare la Smart Home fatta su misura per il cliente. Per poter fare ciò, al momento, l’azienda ha sul mercato all’incirca 150 prodotti sia software che hardware perfettamente coordinabili tra di loro. Qualche esempio di prodotto sono: sensori (di luminosità, umidità, temperatura, fumo, acqua, vento, …), software (sia per il funzionamento del prodotto che le varie applicazioni per Smartphone e altri dispositivi per la gestione del sistema domotico), impianti stereo, pulsantiere, centraline, orologi, piastre per cucine, luci, citofoni, eccetera.
Il cervello della Loxone Smart Home è il Loxone Miniserver, ovvero la centralina che riceve tutte le informazioni dai sensori e dai tasti e le elabora per poi decidere cosa fare, anche in modo autonomo in base alle abitudini degli abitanti della casa. È interessante notare che si tratta di un sistema centralizzato, quindi tutti i cablaggi passano dal server, permettendo così di avere un sistema coerente e lineare nel funzionamento. Questo piccolo server è in grado di lavorare senza l’ausilio di internet, al contrario della stragrande maggioranza dei prodotti simili sul mercato. Questa caratteristica permette di avere un livello di sicurezza dal punto di vista della protezione dei dati molto alto, questo tema era già stato parzialmente trattato nel capitolo della Singenia Sagl e verrà discusso in modo più approfondito nel prossimo capitoletto. 
Un altro prodotto molto interessante e innovativo, anch’esso progettato e sviluppato dalla marca austriaca, è la pulsantiera multifunzionale Loxone Touch. La novità di questo prodotto è che possiede cinque zone tattili dove sono presenti le varie funzioni, a loro volta differenziate a dipendenza del numero di volte che viene premuta la zona. L’idea è quella di avere una di queste pulsantiere in ogni stanza e che funzionano con la stessa logica, così da poter essere utilizzate per gestire tutto l’impianto domotico (illuminazione, riscaldamento, areazione, impianto d’intrattenimento, eccetera) in modo intuitivo. Le funzioni delle varie zone sono programmabili e personalizzabili dall’utente in ogni momento tramite l’applicazione, dalla quale è anche possibile interagire con la casa come con la pulsantiera. Un’altra funzionalità molto interessante è quella dello scenario, ovvero la possibilità di programmare un insieme di parametri in base a una determinata situazione. Un esempio potrebbe essere lo scenario “Relax”, dove si potrebbe rendere l’illuminazione più calda, le tapparelle alzate e musica rilassante. Lo scopo di queste pulsantiere intelligenti è quello di non dover utilizzare il telefono in continuazione per usufruire dei servizi della casa, cosa che invece accade negli altri sistemi domotici. La Loxone definisce questi ultimi come 2.0, mentre la Loxone Smart Home è considerata 3.0 proprio per questo motivo che garantisce un’esperienza abitativa decisamente più confortevole e tecnologica. 
